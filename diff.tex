\documentclass[fleqn,10pt]{wlscirep}
%DIF LATEXDIFF DIFFERENCE FILE
%DIF DEL old_main.tex   Tue Apr  7 15:05:05 2020
%DIF ADD main.tex       Sun Sep 20 21:10:20 2020
\usepackage[utf8]{inputenc}
\usepackage[T1]{fontenc}
\usepackage[algoruled,lined]{algorithm2e}
\usepackage{subcaption}

\newcommand{\R}{\mathbb{R}}
\newtheorem{theorem}{Theorem}
\newtheorem{lemma}[theorem]{Lemma}

%DIF 11-14d11
%DIF < \newcommand{\figureacknowledgement}{Source code and package information for
%DIF < generating this figure are discussed in the acknowledgements, and are available
%DIF < at~\cite{nikoleta_glynatsi_2019}.}
%DIF < 
%DIF -------
\title{A theory of mind: Best responses to memory-one strategies. The limitations
of extortion and restricted memory}

%DIF 18c14
%DIF < \author[1,*]{Nikoleta E. Glynatsi}
%DIF -------
\author[1,2,*]{Nikoleta E. Glynatsi} %DIF > 
%DIF -------
\author[1]{Vincent A. Knight}
\affil[1]{Cardiff University, School of Mathematics, Cardiff, CF24 4AG, United Kingdom}
%DIF 21a17
\affil[2]{Max Planck Institute for Evolutionary Biology, Pl\"{o}n, 24 306, Germany} %DIF > 
%DIF -------

%DIF 22c19
%DIF < \affil[*]{glynatsine@cardiff.ac.uk}
%DIF -------
\affil[*]{glynatsi@evolbio.mpg.de} %DIF > 
%DIF -------

% \affil[+]{these authors contributed equally to this work}

%\keywords{Keyword1, Keyword2, Keyword3}

\begin{abstract}
Memory-one strategies are a set of Iterated Prisoner's Dilemma strategies
that have been praised for their mathematical tractability and performance
%DIF 31-33c28-30
%DIF < against single opponents. This manuscript investigates a theory of mind: \textit{best
%DIF < response} memory-one strategies, as a multidimensional
%DIF < optimisation problem. We add to the literature that has shown that
%DIF -------
against single opponents. This manuscript investigates %DIF > 
\textit{best response} memory-one strategies with a theory of mind for %DIF > 
their opponents. The results add to the literature that has shown that %DIF > 
%DIF -------
extortionate play is not always optimal by showing that optimal play is
often not extortionate.
%DIF 36c33
%DIF < We
%DIF -------
They %DIF > 
%DIF -------
also provide evidence that memory-one strategies suffer from their limited
memory in multi agent interactions and can be out performed by
optimised strategies with longer memory.
%DIF 40a37-42
We have developed a theory that has allowed to explore the entire %DIF > 
space of memory-one strategies. The framework presented is suitable to %DIF > 
study memory-one strategies in the Prisoner's Dilemma, but also %DIF > 
in evolutionary processes such as the Moran process. %DIF > 
Furthermore, results on the stability of defection in populations of %DIF > 
memory-one strategies are also obtained. %DIF > 
%DIF -------
\end{abstract}
%DIF PREAMBLE EXTENSION ADDED BY LATEXDIFF
%DIF UNDERLINE PREAMBLE %DIF PREAMBLE
\RequirePackage[normalem]{ulem} %DIF PREAMBLE
\RequirePackage{color}\definecolor{RED}{rgb}{1,0,0}\definecolor{BLUE}{rgb}{0,0,1} %DIF PREAMBLE
\providecommand{\DIFadd}[1]{{\protect\color{blue}\uwave{#1}}} %DIF PREAMBLE
\providecommand{\DIFdel}[1]{{\protect\color{red}\sout{#1}}}                      %DIF PREAMBLE
%DIF SAFE PREAMBLE %DIF PREAMBLE
\providecommand{\DIFaddbegin}{} %DIF PREAMBLE
\providecommand{\DIFaddend}{} %DIF PREAMBLE
\providecommand{\DIFdelbegin}{} %DIF PREAMBLE
\providecommand{\DIFdelend}{} %DIF PREAMBLE
\providecommand{\DIFmodbegin}{} %DIF PREAMBLE
\providecommand{\DIFmodend}{} %DIF PREAMBLE
%DIF FLOATSAFE PREAMBLE %DIF PREAMBLE
\providecommand{\DIFaddFL}[1]{\DIFadd{#1}} %DIF PREAMBLE
\providecommand{\DIFdelFL}[1]{\DIFdel{#1}} %DIF PREAMBLE
\providecommand{\DIFaddbeginFL}{} %DIF PREAMBLE
\providecommand{\DIFaddendFL}{} %DIF PREAMBLE
\providecommand{\DIFdelbeginFL}{} %DIF PREAMBLE
\providecommand{\DIFdelendFL}{} %DIF PREAMBLE
%DIF LISTINGS PREAMBLE %DIF PREAMBLE
\RequirePackage{listings} %DIF PREAMBLE
\RequirePackage{color} %DIF PREAMBLE
\lstdefinelanguage{DIFcode}{ %DIF PREAMBLE
%DIF DIFCODE_UNDERLINE %DIF PREAMBLE
  moredelim=[il][\color{red}\sout]{\%DIF\ <\ }, %DIF PREAMBLE
  moredelim=[il][\color{blue}\uwave]{\%DIF\ >\ } %DIF PREAMBLE
} %DIF PREAMBLE
\lstdefinestyle{DIFverbatimstyle}{ %DIF PREAMBLE
	language=DIFcode, %DIF PREAMBLE
	basicstyle=\ttfamily, %DIF PREAMBLE
	columns=fullflexible, %DIF PREAMBLE
	keepspaces=true %DIF PREAMBLE
} %DIF PREAMBLE
\lstnewenvironment{DIFverbatim}{\lstset{style=DIFverbatimstyle}}{} %DIF PREAMBLE
\lstnewenvironment{DIFverbatim*}{\lstset{style=DIFverbatimstyle,showspaces=true}}{} %DIF PREAMBLE
%DIF END PREAMBLE EXTENSION ADDED BY LATEXDIFF

\begin{document}

\flushbottom
\maketitle
% * <john.hammersley@gmail.com> 2015-02-09T12:07:31.197Z:
%
%  Click the title above to edit the author information and abstract
%
\thispagestyle{empty}

\section*{Introduction}

The Prisoner's Dilemma (PD) is a two player game used in understanding the
evolution of cooperative behaviour, formally introduced in~\cite{Flood1958}.
Each player has two options, to cooperate (C) or to defect (D). The decisions
are made simultaneously and independently. The normal form representation of the
game is given by:

\begin{equation}\label{equ:pd_definition}
    S_p =
    \begin{pmatrix}
        R & S  \\
        T & P
    \end{pmatrix}
    \quad
    S_q =
    \begin{pmatrix}
        R & T  \\
        S & P
    \end{pmatrix}
\end{equation}

where \(S_p\) represents the utilities of the row player and \(S_q\) the
utilities of the column player. The payoffs, \((R, P, S, T)\), are constrained
by \(T > R > P > S\) and \(2R > T + S\), and the most common values used in the
literature are \((R, P, S, T) = (3, 1, 0, 5)\)~\cite{Axelrod1981}.
The numerical experiments of our manuscript are carried out using these
payoff values.
The PD is a one shot game, however, it is commonly studied in a manner where the
history of the interactions matters. The repeated form of the game is called the
Iterated Prisoner's Dilemma (IPD).

Memory-one strategies are a set of IPD strategies that have been
studied thoroughly in the literature~\cite{Nowak1990, Nowak1993}, however, they have gained
most of their attention when a certain subset of memory-one strategies was
introduced in~\cite{Press2012}, the zero-determinant strategies (ZDs). In~\cite{Stewart2012} it
was stated that ``Press and Dyson have fundamentally changed the viewpoint on
the Prisoner's Dilemma''.
A special case of ZDs are extortionate strategies that choose their actions so that a linear relationship is forced
between the players' score ensuring that they will always
receive at least as much as their opponents. ZDs are
indeed mathematically unique and are proven to be robust in pairwise
interactions, however, their true effectiveness in tournaments and
evolutionary dynamics has been questioned~\cite{adami2013, Hilbe2013b,
Hilbe2013, Hilbe2015, Knight2018, Harper2015}.

\DIFdelbegin \DIFdel{In~\mbox{%DIFAUXCMD
\cite{Press2012} }\hspace{0pt}%DIFAUXCMD
the authors stated that ``Only a player with a theory of
mind about his opponent can do better, in which case Iterated Prisoner's Dilemma
is an Ultimatum Game''. }\DIFdelend The purpose of this work is to \DIFdelbegin \DIFdel{investigate the first
part of
this sentence, more specifically, to identify the best response strategy
}\DIFdelend \DIFaddbegin \DIFadd{reinforce the literature on the limitations of
extortionate strategies by considering a new approach. More specifically, by
considering best response memory-one strategies }\DIFaddend with a theory of mind of
\DIFdelbegin \DIFdel{a given group of opponents. The outcomes of our work
reinforce known results, namely that memory-one strategies must be forgiving to
be evolutionarily stable~\mbox{%DIFAUXCMD
\cite{Stewart2013, Stewart2016} }\hspace{0pt}%DIFAUXCMD
and that longer-memory
strategies have a certain form of advantage over short memory
strategies~\mbox{%DIFAUXCMD
\cite{Hilbe2017, Pan2015}}\hspace{0pt}%DIFAUXCMD
}\DIFdelend \DIFaddbegin \DIFadd{their opponents. There are several works in the literature that have considered
strategies with a theory of mind~\mbox{%DIFAUXCMD
\cite{Han2011, De2013, Devaine2014, Han2012,
Press2012, Stewart2012}}\hspace{0pt}%DIFAUXCMD
. These works defined ``theory of mind'' as intention
recognition~\mbox{%DIFAUXCMD
\cite{Han2011, Han2012, De2013, Devaine2014} }\hspace{0pt}%DIFAUXCMD
and as the ability of a
strategy to realise that their actions can influence
opponents~\mbox{%DIFAUXCMD
\cite{Stewart2012}}\hspace{0pt}%DIFAUXCMD
. Compared to these works, theory of mind is defined
here as the ability of a strategy to know the behaviour of their opponents
and alter their own behaviour in response to that}\DIFaddend .

\DIFdelbegin \DIFdel{In particular, this work presents }\DIFdelend \DIFaddbegin \DIFadd{We present }\DIFaddend a closed form algebraic expression for the utility of a
memory-one strategy against a given set of opponents \DIFdelbegin \DIFdel{, }\DIFdelend \DIFaddbegin \DIFadd{and }\DIFaddend a compact method of
identifying it's best response to that given set of opponents\DIFdelbegin \DIFdel{essentially: a theory
of mind}\DIFdelend .
The aim is to evaluate whether a best response memory-one
strategy behaves in a zero-determinant way which in turn indicates whether it
can be extortionate. We do this using a linear algebraic approach presented
in~\cite{Knight2019}. This is done in tournaments with \DIFdelbegin \DIFdel{and without self interactions}\DIFdelend \DIFaddbegin \DIFadd{two opponents}\DIFaddend .
Moreover, we introduce a framework that allows the comparison of
an optimal memory-one strategy and an optimised strategy which has a larger
memory.

To illustrate the analytical results obtained in this manuscript a number of
numerical experiments are run. The source code for these experiments has been
written in a sustainable manner~\cite{Benureau2018}.
It is open source (\url{https://github.com/Nikoleta-v3/Memory-size-in-the-prisoners-dilemma})
and tested which ensures the validity of the results. It has also been archived
and can be found at~\DIFdelbegin \DIFdel{\mbox{%DIFAUXCMD
\cite{nikoleta_glynatsi_2019}}\hspace{0pt}%DIFAUXCMD
}\DIFdelend \DIFaddbegin \DIFadd{\mbox{%DIFAUXCMD
\cite{opt_mo2019}}\hspace{0pt}%DIFAUXCMD
}\DIFaddend .

\section*{Methods}
One specific advantage of memory-one strategies is their mathematical
tractability. They can be represented completely as an element of \(\R^{4}_{[0, 1]}\). This
originates from~\cite{Nowak1989} where it is stated that if a strategy is
concerned with only the outcome of a single turn then there are four possible
`states' the strategy could be in; both players cooperated (\(CC\)),
the first player cooperated whilst the second player defected (\(CD\)),
the first player defected whilst the second player cooperated (\(DC\)) and
both players defected (\(DD\)).
Therefore, a memory-one strategy can be denoted by the probability vector of
cooperating after each of these states; \(p=(p_1, p_2, p_3, p_4) \in \R_{[0,1]}
^ 4\).

In~\cite{Nowak1989} it was shown that it is not necessary to simulate the play
of a strategy $p$ against a memory-one opponent $q$. Rather this exact behaviour
can be modeled as a stochastic process, and more specifically as a Markov chain
whose corresponding transition matrix \(M\) is
given by Equation~(\ref{eq:transition_matrix}). The long run steady state probability
vector \(v\), which is the solution to \(v M = v\), can be
combined with the payoff matrices of Equation~(\ref{equ:pd_definition}) to give the expected
payoffs for each player. More specifically, the utility for a memory-one
strategy \(p\) against an opponent \(q\), denoted as \(u_q(p)\), is given by
Equation~(\ref{eq:press_dyson_utility}).

\begin{equation}\label{eq:transition_matrix}
    \resizebox{.5\hsize}{!}{$M = [[p_1*q_1 p_1*(-q_1 + 1) q_1*(-p_1 + 1) (-p_1 + 1)*(-q_1 + 1)]
 [p_2*q_3 p_2*(-q_3 + 1) q_3*(-p_2 + 1) (-p_2 + 1)*(-q_3 + 1)]
 [p_3*q_2 p_3*(-q_2 + 1) q_2*(-p_3 + 1) (-p_3 + 1)*(-q_2 + 1)]
 [p_4*q_4 p_4*(-q_4 + 1) q_4*(-p_4 + 1) (-p_4 + 1)*(-q_4 + 1)]]$}
\end{equation}


\begin{equation}\label{eq:press_dyson_utility}
    u_q(p) = v \cdot (R, S, T, P).
\end{equation}

This manuscript has explored the form of \(u_q(p)\), to the \DIFaddbegin \DIFadd{best of the }\DIFaddend authors knowledge no
previous work has done this, and Theorem~\ref{theorem_one} states that \(u_q(p)\) is given by a ratio
of two quadratic forms~\cite{kepner2011}.

\begin{theorem}\label{theorem_one}
    The expected utility of a memory-one strategy \(p\in\mathbb{R}_{[0,1]}^4\)
    against a memory-one opponent \(q\in\mathbb{R}_{[0,1]}^4\), denoted
    as \(u_q(p)\), can be written as a ratio of two quadratic forms:

    \begin{equation}\label{eq:optimisation_quadratic}
    u_q(p) = \frac{\frac{1}{2}pQp^T + cp + a}{\frac{1}{2}p\bar{Q}p^T + \bar{c}p + \bar{a}},
    \end{equation}
    where \(Q, \bar{Q}\) \(\in \R^{4\times4}\) are square matrices defined by the
    transition probabilities of the opponent \(q_1, q_2, q_3, q_4\) as follows:

    \begin{center}
    \begin{equation}
    \resizebox{0.9\linewidth}{!}{\arraycolsep=2.5pt%
    \boldmath\(
    Q = \left[\begin{matrix}0 & - \left(q_{1} - q_{3}\right) \left(P q_{2} - P - T q_{4}\right) & \left(q_{1} - q_{2}\right) \left(P q_{3} - S q_{4}\right) & \left(q_{1} - q_{4}\right) \left(S q_{2} - S - T q_{3}\right)\\- \left(q_{1} - q_{3}\right) \left(P q_{2} - P - T q_{4}\right) & 0 & \left(q_{2} - q_{3}\right) \left(P q_{1} - P - R q_{4}\right) & - \left(q_{3} - q_{4}\right) \left(R q_{2} - R - T q_{1} + T\right)\\\left(q_{1} - q_{2}\right) \left(P q_{3} - S q_{4}\right) & \left(q_{2} - q_{3}\right) \left(P q_{1} - P - R q_{4}\right) & 0 & \left(q_{2} - q_{4}\right) \left(R q_{3} - S q_{1} + S\right)\\\left(q_{1} - q_{4}\right) \left(S q_{2} - S - T q_{3}\right) & - \left(q_{3} - q_{4}\right) \left(R q_{2} - R - T q_{1} + T\right) & \left(q_{2} - q_{4}\right) \left(R q_{3} - S q_{1} + S\right) & 0\end{matrix}\right]\)},
    \end{equation}
    \begin{equation}\label{eq:q_bar_matrix}
    \resizebox{0.8\linewidth}{!}{\arraycolsep=2.5pt%
    \boldmath\(
    \bar{Q} =  \left[\begin{matrix}0 & - \left(q_{1} - q_{3}\right) \left(q_{2} - q_{4} - 1\right) & \left(q_{1} - q_{2}\right) \left(q_{3} - q_{4}\right) & \left(q_{1} - q_{4}\right) \left(q_{2} - q_{3} - 1\right)\\- \left(q_{1} - q_{3}\right) \left(q_{2} - q_{4} - 1\right) & 0 & \left(q_{2} - q_{3}\right) \left(q_{1} - q_{4} - 1\right) & \left(q_{1} - q_{2}\right) \left(q_{3} - q_{4}\right)\\\left(q_{1} - q_{2}\right) \left(q_{3} - q_{4}\right) & \left(q_{2} - q_{3}\right) \left(q_{1} - q_{4} - 1\right) & 0 & - \left(q_{2} - q_{4}\right) \left(q_{1} - q_{3} - 1\right)\\\left(q_{1} - q_{4}\right) \left(q_{2} - q_{3} - 1\right) & \left(q_{1} - q_{2}\right) \left(q_{3} - q_{4}\right) & - \left(q_{2} - q_{4}\right) \left(q_{1} - q_{3} - 1\right) & 0\end{matrix}\right]\)}.
    \end{equation}
    \end{center}

    \(c \text{ and } \bar{c}\) \(\in \R^{4 \times 1}\) are similarly defined by:

    \begin{equation}\label{eq:q_matrix_numerator}
    \resizebox{0.6\linewidth}{!}{\arraycolsep=2.5pt%
    \boldmath\(c = \left[\begin{matrix}- 5 q_{1} q_{4}\\5 q_{4} \left(q_{3} - 1\right)\\q_{4} \left(2 q_{2} + 1\right)\\5 q_{1} q_{4} - 2 q_{2} q_{4} - q_{2} - 5 q_{3} q_{4} + 5 q_{3} - 3 q_{4} + 1\end{matrix}\right]\),}
    \end{equation}
    \begin{equation}\label{eq:q_matrix_denominator}
    \resizebox{0.3\linewidth}{!}{\arraycolsep=2.5pt%
    \boldmath\(\bar{c} = \left[\begin{matrix}q_{1} \left(q_{2} - q_{4} - 1\right)\\- \left(q_{3} - 1\right) \left(q_{2} - q_{4} - 1\right)\\- q_{1} q_{2} + q_{2} q_{3} + q_{2} - q_{3} + q_{4}\\q_{1} q_{4} - q_{2} - q_{3} q_{4} + q_{3} - q_{4} + 1\end{matrix}\right]\),
    }
    \end{equation}
    and the constant terms \(a, \bar{a}\) are defined as \(a = 5 q_{4}\) and
    \(\bar{a} = - q_{2} + q_{4} + 1\).
\end{theorem}

The proof of Theorem~\ref{theorem_one} is given in the
Supplementary Information.
Theorem~\ref{theorem_one} can be extended to consider multiple
opponents. The IPD is commonly studied in tournaments and/or Moran Processes
where a strategy interacts with a number of opponents. The payoff of a player in
such interactions is given by the average payoff the player received against
each opponent. More specifically the expected utility of a memory-one strategy
against \(N\) opponents is given by:

\begin{align}\label{eq:tournament_utility}
       & \frac{1}{N} \sum\limits_{i=1} ^ {N} {u_q}^{(i)} (p) =
       \frac{\frac{1}{N} \sum\limits_{i=1} ^ {N} (\frac{1}{2} pQ^{(i)} p^T + c^{(i)} p + a^ {(i)})
       \prod\limits_{\tiny\begin{array}{l} j=1 \\ j \neq i \end{array}} ^
       N (\frac{1}{2} p\bar{Q}^{(j)} p^T + \bar{c}^{(j)} p + \bar{a}^ {(j)})}{\prod\limits_{i=1} ^ N (\frac{1}{2} p\bar{Q}^{(i)} p^T + \bar{c}^{(i)} p + \bar{a}^ {(i)})}.
\end{align}

Equation~(\ref{eq:tournament_utility}) is the average score (using Equation~(\ref{eq:optimisation_quadratic})) against the set of opponents.

Estimating the utility of a memory-one strategy against any number of opponents
without simulating the interactions is the main result used in the rest of this manuscript.
It will be used to obtain best response memory-one strategies in tournaments
\DIFdelbegin \DIFdel{with and without self interactions }\DIFdelend in order to explore the limitations of extortion
and restricted memory.

\section*{Results}

\DIFdelbegin \DIFdel{Here we }\DIFdelend \DIFaddbegin \DIFadd{The formulation as presented in Theorem~\ref{theorem_one} can be used to
}\DIFaddend define \textit{memory-one best response} strategies as a multi dimensional
optimisation problem given by:

\begin{equation}\label{eq:mo_tournament_optimisation}
    \begin{aligned}
    \max_p: & \ \sum_{i=1} ^ {N} {u_q}^{(i)} (p)
    \\
    \text{such that}: & \ p \in \R_{[0, 1]}
    \end{aligned}
\end{equation}

Optimising this particular ratio of quadratic forms is not trivial. It can be
verified empirically for the case of a single opponent that there exists at
least one point for which the definition of concavity does not hold.
The non concavity of \(u(p)\) indicates multiple local
optimal points. This is also intuitive. The best response against a cooperator,
\(q=(1, 1, 1, 1)\), is a defector \(p^*=(0, 0, 0, 0)\). The strategies
\(p=(\frac{1}{2}, 0, 0, 0)\) and \(p=(\frac{1}{2}, 0, 0, \frac{1}{2})\) are also
best responses. The approach taken here is to introduce a compact way of
constructing the discrete candidate set of all local optimal points, and evaluating
the objective function Equation~(\ref{eq:tournament_utility}). This gives the best
response memory-one strategy. The approach is given in
Theorem~\ref{memone_group_best_response}.

\begin{theorem}\label{memone_group_best_response}

    The optimal behaviour of a memory-one strategy player \(p^* \in \R_{[0, 1]} ^
    4\) against a set of \(N\) opponents \(\{q^{(1)}, q^{(2)}, \dots, q^{(N)} \}\)
    for \(q^{(i)} \in \R_{[0, 1]} ^ 4\) is given by:

    \[p^* = \textnormal{argmax}\sum\limits_{i=1} ^ N  u_q(p), \ p \in S_q.\]

    The set \(S_q\) is defined as all the possible combinations of:

    \DIFaddbegin {\scriptsize
    \DIFaddend \begin{equation}\label{eq:s_q_set}
        S_q =
        \left\{p \in \mathbb{R} ^ 4 \left|
            \begin{aligned}
                \bullet\quad p_j \in \{0, 1\} & \quad \text{and} \quad \frac{d}{dp_k}
                \sum\limits_{i=1} ^ N  u_q^{(i)}(p) = 0 \\
                & \quad \text{for all} \quad j \in J \quad \&  \quad k \in K  \quad \text{for all} \quad J, K \\
                & \quad \text{where} \quad J \cap K = \O \quad
                \text{and} \quad J \cup K = \{1, 2, 3, 4\}.\\
                \bullet\quad  p \in \{0, 1\} ^ 4
            \end{aligned}\right.
        \right\}.
    \end{equation}
    \DIFaddbegin }
\DIFaddend 

    Note that there is no immediate way to find the zeros of \(\frac{d}{dp}
    \sum\limits_{i=1} ^ N  u_q(p)\) where,

    \DIFaddbegin {\scriptsize
    \DIFaddend \begin{align}\label{eq:mo_tournament_derivative}
        \frac{d}{dp} \sum\limits_{i=1} ^ {N} {u_q}^{(i)} (p) & = \displaystyle\sum\limits_{i=1} ^ {N}
        \frac{\left(pQ^{(i)} + c^{(i)}\right) \left(\frac{1}{2} p\bar{Q}^{(i)} p^T + \bar{c}^{(i)} p + \bar{a}^ {(i)}\right)}{\left(\frac{1}{2} p\bar{Q}^{(i)} p^T + \bar{c}^{(i)} p + \bar{a}^ {(i)}\right)^ 2}
        - \frac{\left(p\bar{Q}^{(i)} + \bar{c}^{(i)}\right) \left(\frac{1}{2} pQ^{(i)} p^T + c^{(i)} p + a^ {(i)}\right)}{\left(\frac{1}{2} p\bar{Q}^{(i)} p^T + \bar{c}^{(i)} p + \bar{a}^ {(i)}\right)^ 2}
    \end{align}
    \DIFaddbegin }
\DIFaddend 

    For \(\frac{d}{dp} \sum\limits_{i=1} ^ N  u_q(p)\) to equal zero then:

    \DIFaddbegin {\scriptsize
    \DIFaddend \begin{align}\label{eq:polynomials_roots}
        \displaystyle\sum\limits_{i=1} ^ {N}
        \left(pQ^{(i)} + c^{(i)}\right) \left(\frac{1}{2} p\bar{Q}^{(i)} p^T + \bar{c}^{(i)} p + \bar{a}^ {(i)}\right)
        - \left(p\bar{Q}^{(i)} + \bar{c}^{(i)}\right) \left(\frac{1}{2} pQ^{(i)} p^T + c^{(i)} p + a^ {(i)}\right)
        & = 0, \quad {while} \\
        \displaystyle\sum\limits_{i=1} ^ {N} \frac{1}{2} p\bar{Q}^{(i)} p^T + \bar{c}^{(i)} p + \bar{a}^ {(i)} & \neq 0.
    \end{align}\DIFaddbegin }
\DIFaddend 

\end{theorem}

The proof of Theorem~\ref{memone_group_best_response} is given in the
Supplementary Information.
Finding best response memory-one strategies is analytically feasible using the
formulation of Theorem~\ref{memone_group_best_response} and resultant
theory~\cite{Jonsson2005}. However, for large systems building the resultant
becomes intractable. As a result, best responses will be estimated
heuristically using a numerical method, suitable for problems with local optima,
called Bayesian optimisation~\cite{Mokus1978}.

\DIFdelbegin \DIFdel{In several evolutionary settings
such as Moran Processes self interactions are key. Previous work has
identified interesting results such as the appearance of self recognition
mechanisms when training strategies using evolutionary algorithms in Moran
processes~\mbox{%DIFAUXCMD
\cite{Knight2018}}\hspace{0pt}%DIFAUXCMD
. This aspect of reinforcement learning can be done for
best response memory-one strategies by incorporating the strategy itself in the
objective function as shown in Equation~(\ref{eq:mo_tournament_optimisation}).
\(K\) is the number of self interactions that will take place.
}\DIFdelend \DIFaddbegin \subsection*{\DIFadd{Limitations of extortionate behaviour}}
\DIFaddend 

\DIFdelbegin \begin{displaymath}\DIFdel{\label{eq:mo_evolutionary_optimisation}
\begin{aligned}
\max_p: & \ \frac{1}{N} \sum\limits_{i=1} ^ {N} {u_q}^{(i)} (p) + Ku_p(p)
\\
\text{such that}: & \ p \in \R_{[0, 1]}
\end{aligned}
}\end{displaymath}%DIFAUXCMD
%DIFDELCMD < 

%DIFDELCMD < %%%
\DIFdel{For determining the memory-one best response with self interactions, an
algorithmic approach is considered, called }\textit{\DIFdel{best response dynamics}}%DIFAUXCMD
\DIFdel{. The
best response dynamics approach used in this manuscript is given by
Algorithm~\ref{algo:best_response_dynamics}.
}%DIFDELCMD < 

%DIFDELCMD < \begin{center}
%DIFDELCMD < \begin{minipage}{.55\textwidth}
%DIFDELCMD < \begin{algorithm}[H]
%DIFDELCMD <        %%%
\DIFdel{$p^{(t)}\leftarrow (1, 1, 1, 1)$\;
       }%DIFDELCMD < \While{$p^{(t)} \neq p ^{(t -1)}$}{
%DIFDELCMD <        $p^{(t + 1)} =  \text{argmax} \frac{1}{N} \sum\limits_{i=1} ^ {N} {u_q}^{(i)}
%DIFDELCMD <        (p^{(t)}) + Ku_{p^{(t)}}(p^{(t)})$\;
%DIFDELCMD <        }
%DIFDELCMD <        %%%
%DIFDELCMD < \caption{%
{%DIFAUXCMD
\DIFdel{Best response dynamics Algorithm}}
       %DIFAUXCMD
%DIFDELCMD < \label{algo:best_response_dynamics}
%DIFDELCMD < \end{algorithm}
%DIFDELCMD < \end{minipage}
%DIFDELCMD < \end{center}
%DIFDELCMD < 

%DIFDELCMD < %%%
\DIFdel{Using this approach it would be possible to create a memory-one best response
strategy that updates on every generation of a Moran process to recalculate the
optimal behaviour given the population. This extension of the ``theory of mind''
is an interesting avenue for future work.
}%DIFDELCMD < 

%DIFDELCMD < %%%
\DIFdelend In multi opponent settings,
where the payoffs matter, strategies trying to exploit their opponents will
suffer.
Compared to ZDs, best response memory-one strategies, which have a
theory of mind of their opponents, utilise their behaviour in order to gain the
most from their interactions. The question that arises then is whether best
response strategies are optimal because they behave in an extortionate way.

The results of this section use Bayesian optimisation to generate a data set of best response
memory-one strategies for \(N=2\) opponents.
The data set is available at~\cite{glynatsi2019}. It contains a total of 1000 trials
corresponding to 1000 different instances of a best response strategy in
tournaments with \DIFdelbegin \DIFdel{and without self interactions}\DIFdelend \DIFaddbegin \DIFadd{\(N=2\)}\DIFaddend . For each trial a set of 2 opponents is
randomly generated and the memory-one best response against them is found.
In order to investigate whether best responses
behave in an extortionate matter the SSE method described in ~\cite{Knight2019} is used. More
specifically,
in~\cite{Knight2019} the point \(x^*\), in the space of memory-one strategies,
that is
the nearest extortionate strategy to a given strategy \(p\) is
given by,

\begin{equation}\label{eqn:x_star_formula}
    x^* = {\left(C^{T}C\right)}^{-1}C^{T}\bar{p}
\end{equation}

where \(\bar{p}=(p_1 - 1, p_2 - 1, p_3, p_4)\) and

\begin{equation}\label{eq:definition_of_C}
    C =
    \begin{bmatrix}
        R - P & R- P \\
        S - P & T- P \\
        T - P & S- P \\
        0     & 0 \\
    \end{bmatrix}.
\end{equation}

Once this closest ZDs is found, the squared norm of the remaining error is referred to as sum of squared errors
of prediction (SSE):

\begin{equation}\label{eqn:x_SSError_formula}
    \text{SSE} = {\bar{p}} ^ T \bar{p} -
           \bar{p} C \left(C ^ T C \right) ^ {-1} C ^ T \bar{p}
         = {\bar{p}} ^ T \bar{p} - \bar{p} C x ^ *
\end{equation}

Thus, SSE is defined as how far a strategy is from behaving as a ZD. A
high SSE implies a non extortionate behaviour.
The \DIFdelbegin \DIFdel{distributions }\DIFdelend \DIFaddbegin \DIFadd{distribution }\DIFaddend of SSE for the best response in tournaments
(\(N=2\)) \DIFdelbegin \DIFdel{with and without self interactions (with \(K=1\)) are }\DIFdelend \DIFaddbegin \DIFadd{is }\DIFaddend given in
Figure~\ref{fig:sse_distributions}. Moreover, a statistical summary of the SSE
\DIFdelbegin \DIFdel{distributions }\DIFdelend \DIFaddbegin \DIFadd{distribution }\DIFaddend is given in Table~\ref{table:sserror_stats}.

\begin{table}[!htbp]
    \begin{center}
    \DIFdelbeginFL %DIFDELCMD < \resizebox{.8\columnwidth}{!}{%
%DIFDELCMD <     \begin{tabular}{lrrrrrrrrrrr}
%DIFDELCMD <         \toprule
%DIFDELCMD <         & mean & std  & 5\% & 50\% &  95\% & max & median & skew & kurt\\
%DIFDELCMD <         \midrule
%DIFDELCMD <     \textbf{Tournament without self interactions} & 0.34  & 0.40  & 0.028  & 0.17  &
%DIFDELCMD <     1.05  & 2.47  & 0.17  & 1.87 & 3.60 \\
%DIFDELCMD <     \textbf{Tournament with self interactions} & 0.17 & 0.23 & 0.01 &
%DIFDELCMD <     0.12 & 0.67 & 1.53 & 0.12 & 3.42 & 1.92 \\
%DIFDELCMD <         \bottomrule
%DIFDELCMD <     \end{tabular}}
%DIFDELCMD <     %%%
\DIFdelendFL \DIFaddbeginFL \resizebox{.5\columnwidth}{!}{%
    \begin{tabular}{lrrrrrrrrrrr}
    \toprule
    mean & std  & 5\% & 50\% &  95\% & max & median & skew & kurt\\
    \midrule
    0.34  & 0.40  & 0.028  & 0.17  & 1.05  & 2.47  & 0.17  & 1.87 & 3.60 \\
    \bottomrule
    \end{tabular}}
    \DIFaddendFL \end{center}
    \caption{SSE of best response memory-one when \(N=2\)}\label{table:sserror_stats}
\end{table}

\begin{figure}[!htbp]
    \DIFdelbeginFL %DIFDELCMD < \begin{subfigure}{0.47\textwidth}
%DIFDELCMD <         %%%
\DIFdelendFL \begin{center}
        \DIFdelbeginFL %DIFDELCMD < \includegraphics[width=\linewidth]{Figure_1_a.jpeg}
%DIFDELCMD <         %%%
\DIFdelendFL \DIFaddbeginFL \includegraphics[width=.5\linewidth]{Figure_1.pdf}
    \DIFaddendFL \end{center}
    \caption{SEE distribution for best response in tournaments \DIFdelbeginFL \DIFdelFL{without self interactions}\DIFdelendFL \DIFaddbeginFL \DIFaddFL{with \(N=2\)}\DIFaddendFL .}\DIFdelbeginFL %DIFDELCMD < \end{subfigure}\hfill
%DIFDELCMD <     \begin{subfigure}{0.47\textwidth}
%DIFDELCMD <         \begin{center}
%DIFDELCMD <             \includegraphics[width=\linewidth]{Figure_1_b.jpeg}
%DIFDELCMD <         \end{center}
%DIFDELCMD <         %%%
%DIFDELCMD < \caption{%
{%DIFAUXCMD
\DIFdelFL{SEE distribution for best response in tournaments with self interactions.}}
    %DIFAUXCMD
%DIFDELCMD < \end{subfigure}
%DIFDELCMD <     %%%
%DIFDELCMD < \caption{%
{%DIFAUXCMD
\DIFdelFL{SEE distributions for best response in tournaments without and with self interactions. }%DIFDELCMD < \figureacknowledgement%%%
}%DIFAUXCMD
\DIFdelendFL \label{fig:sse_distributions}
\end{figure}

For the best response in tournaments \DIFdelbegin \DIFdel{that do not include self interactions }\DIFdelend \DIFaddbegin \DIFadd{with \(N=2\) }\DIFaddend the
distribution of SSE is skewed to the left, indicating that the best response
does exhibit ZDs behaviour and so could be extortionate, however, the best
response is not uniformly a ZDs. A positive measure of skewness and kurtosis,
and a mean of 0.34 indicate a heavy tail to the right. Therefore, in several
cases the strategy is not trying to extort its opponents. \DIFdelbegin \DIFdel{In~\mbox{%DIFAUXCMD
\cite{Hilbe2018} }\hspace{0pt}%DIFAUXCMD
a similar behaviour is refereed to as the }\textit{\DIFdel{partner
strategy}}%DIFAUXCMD
\DIFdel{. The partner
strategy aims to share the payoff for mutual cooperation, but it is ready
to fight back when being exploited. The partner strategy was designed, but the
best responses which are defined by their opponents seem to exhibit the same
behaviour.
}%DIFDELCMD < 

%DIFDELCMD < %%%
\DIFdel{Similarly, when considering self interactions, the distribution of SSE for }\DIFdelend \DIFaddbegin \DIFadd{This highlights the importance of adaptability since }\DIFaddend the best response strategy \DIFdelbegin \DIFdel{has skewness and kurtosis that indicate a heavy tail to the
right. This indicates that evolutionary stable memory-one strategies need to
more adaptable than a ZDs, and aim for mutual cooperation as well as
exploitation which is in line with the results of \mbox{%DIFAUXCMD
\cite{Hilbe2018} }\hspace{0pt}%DIFAUXCMD
where their
strategy was designed to adapt and was shown to be evolutionary stable. The
findings of this work show that an optimal strategy acts in the same way.
}\DIFdelend \DIFaddbegin \DIFadd{against an opponent is rarely (if ever) a unique ZDs.
}\DIFaddend 

\DIFdelbegin \DIFdel{The difference between best responses in tournaments without and with self interactions
is further explored by Figure~\ref{fig:behaviour_violin_plots}.
Though, no statistically significant differences have been found, from
Figure~\ref{fig:behaviour_violin_plots}, it seems that the best
response that incorporate self interactions has a higher median $p_2$; which corresponds to the probability of cooperating
after receiving a defection.
Thus, they are more likely to forgive after
being tricked. This is due to the fact that they could be playing against
themselves, and they need to be able to forgive so that future cooperation can
occur.
}\DIFdelend \DIFaddbegin \subsection*{\DIFadd{Limitations of memory size}}
\DIFaddend 

\DIFdelbegin %DIFDELCMD < \begin{figure}[!htbp]
%DIFDELCMD <     \centering
%DIFDELCMD <     \includegraphics[width=.9\textwidth]{Figure_2}
%DIFDELCMD <     %%%
%DIFDELCMD < \caption{%
{%DIFAUXCMD
\DIFdelFL{Distributions of \(p^*\) for best responses in tournaments and
    evolutionary settings. The medians, denoted as \(\bar{p}^*\), for tournaments
    are \(\bar{p}^* = (0, 0, 0, 0)\), and for evolutionary settings
    \(\bar{p}^* = (0, 0.19, 0, 0)\). }%DIFDELCMD < \figureacknowledgement%%%
}
    %DIFAUXCMD
%DIFDELCMD < \label{fig:behaviour_violin_plots}
%DIFDELCMD < \end{figure}
%DIFDELCMD < 

%DIFDELCMD < %%%
\DIFdelend The other main finding presented in~\cite{Press2012} was that
short memory of the strategies was all that was needed.
We argue that the second limitation of ZDs in multi opponent
interactions is that of their restricted memory.
To demonstrate the effectiveness of memory in the IPD we explore a best response
longer-memory strategy against a given set of memory-one opponents,  and compare
it's performance to that of a memory-one best response.

In~\cite{Harper2017}, a strategy called \textit{Gambler} which makes
probabilistic decisions based on the opponent's \(n_1\) first moves, the
opponent's \(m_1\) last moves and the player's \(m_2\) last moves was
introduced. In this manuscript Gambler with parameters: $n_1 = 2, m_1 = 1$ and $m_2 = 1$ is used
as a longer-memory strategy.
By considering the opponent's first two moves, the opponents last move and the
player's last move, there are only 16 $(4 \times 2 \times 2)$ possible outcomes
that can occur, furthermore, Gambler also makes a probabilistic decision of
cooperating in the opening move. Thus, Gambler is a function \(f: \{\text{C,
D}\} \rightarrow [0, 1]_{\R}\). This can be hard coded as an element
of \([0, 1]_{\R} ^ {16 + 1}\), one probability for each outcome plus the opening
move. Hence, compared to Equation~(\ref{eq:mo_tournament_optimisation}), finding an
optimal Gambler is a 17 dimensional problem given by:

\begin{equation}\label{eq:gambler_optimisation}
    \begin{aligned}
    \max_p: & \ \sum_{i=1} ^ {N} {U_q}^{(i)} (f)
    \\
    \text{such that}: & \ f \in \R_{[0, 1]}^{17}
    \end{aligned}
\end{equation}

Note that Equation \DIFdelbegin \DIFdel{~}\DIFdelend (\ref{eq:tournament_utility}) can not be used here for the utility
of Gambler, and actual simulated players are used. This is done using~\cite{axelrodproject}
with 500 turns and 200 repetitions, moreover, Equation \DIFdelbegin \DIFdel{~}\DIFdelend (\ref{eq:gambler_optimisation})
is solved numerically using Bayesian optimisation.

Similarly to \DIFdelbegin \DIFdel{previous sections}\DIFdelend \DIFaddbegin \DIFadd{the previous section}\DIFaddend , a large data set has been generated with
instances of an optimal Gambler and a memory-one best response, available
at~\cite{glynatsi2019}. Estimating a best response Gambler (17 dimensions) is
computational more expensive compared to a best response memory-one (4
dimensions). As a result, the analysis of this section is based on a total of
152 trials. As before, for each trial \(N=2\) random opponents have been selected.

The ratio between Gambler's utility and the best response memory-one strategy's utility has been calculated and its distribution in
given in Figure~\ref{fig:utilities_gambler_mem_one}.
It is evident from Figure~\ref{fig:utilities_gambler_mem_one} that
Gambler always performs as well as the best response memory-one strategy and often performs better. There are
no points where the ratio value is less than 1, thus Gambler never performed less
than the best response memory-one strategy and in places outperforms it.
However, against two memory-one opponents Gambler's performance is better than
the optimal memory-one strategy. This is evidence that in the case of multiple
opponents, having a
shorter memory is limiting.

\begin{figure}[!htbp]
    \centering
    \DIFdelbeginFL %DIFDELCMD < \includegraphics[width=.5\textwidth]{Figure_3}
%DIFDELCMD <     %%%
\DIFdelendFL \DIFaddbeginFL \includegraphics[width=.5\textwidth]{Figure_2.pdf}
    \DIFaddendFL \caption{The ratio between the utilities of Gambler and best response memory-one
    strategy for 152 different pair of opponents.\DIFdelbeginFL %DIFDELCMD < \figureacknowledgement%%%
\DIFdelendFL }\label{fig:utilities_gambler_mem_one}
\end{figure}

\DIFaddbegin \subsection*{\DIFadd{Dynamic best response player}}

\DIFadd{In several evolutionary settings
such as Moran Processes self interactions are key. Previous work has
identified interesting results such as the appearance of self recognition
mechanisms when training strategies using evolutionary algorithms in Moran
processes~\mbox{%DIFAUXCMD
\cite{Knight2018}}\hspace{0pt}%DIFAUXCMD
. This aspect of reinforcement learning can be done for
best response memory-one strategies, as presented in this manuscript, by incorporating the strategy itself in the
objective function as shown in Equation~(\ref{eq:mo_tournament_optimisation}).
Where \(K\) is the number of self interactions that will take place.
}

\begin{equation}\DIFadd{\label{eq:mo_evolutionary_optimisation}
\begin{aligned}
\max_p: & \ \frac{1}{N} \sum\limits_{i=1} ^ {N} {u_q}^{(i)} (p) + Ku_p(p)
\\
\text{such that}: & \ p \in \R_{[0, 1]}
\end{aligned}
}\end{equation}

\DIFadd{For determining the memory-one best response with self interactions, an
algorithmic approach called }\textit{\DIFadd{best response dynamics}} \DIFadd{is proposed. The
best response dynamics approach used in this manuscript is given by
Algorithm~\ref{algo:best_response_dynamics}.
}

\begin{center}
\begin{minipage}{.55\textwidth}
\begin{algorithm}[H]
       \DIFadd{$p^{(t)}\leftarrow (1, 1, 1, 1)$\;
       }\While{$p^{(t)} \neq p ^{(t -1)}$}{
       $p^{(t + 1)} =  \text{argmax} \frac{1}{N} \sum\limits_{i=1} ^ {N} {u_q}^{(i)}
       (p^{(t)}) + Ku_{p^{(t)}}(p^{(t)})$\;
       }
       \caption{\DIFadd{Best response dynamics Algorithm}}
       \label{algo:best_response_dynamics}
\end{algorithm}
\end{minipage}
\end{center}

\DIFadd{To investigate the effectiveness of this approach, more formally a Moran process
will be considered. If a population of \(n\) total individuals of two types is
considered, with \(K\) individuals of the first type and \(n-K\) of the second
type.
The probability that the individuals of the first type will take over the
population (the fixation probability) is denoted by \(x_K\) and is known to
be~\mbox{%DIFAUXCMD
\cite{nowak2006evolutionary}}\hspace{0pt}%DIFAUXCMD
:
}

\[ \DIFadd{x_K = \frac{ 1 + \sum_{j=1}^{K-1}\prod_{i=1}^j\gamma_i }{ 1 +
\sum_{j=1}^{n-1}\prod_{i=1}^j\gamma_i } }\]

\DIFadd{where:
}

\[ \DIFadd{\gamma_i = \frac{ p_{K, K - 1} }{ p_{K, K + 1} }. }\]

\DIFadd{To evaluate the formulation proposed here the best response player (taken to be
the first type of individual in our population)  will be allowed to act
dynamically: adjusting their probability vector at every generation. In essence
using the theory of mind to find the best response to not only the opponent but
also the distribution of the population. Thus for every value of \(K\) there is
a different best response player.
}

\DIFadd{Considering the dynamic best response player as a vector
\(p\in\mathbb{R}^4_{[0, 1]}\) and the opponent as a vector
\(q\in\mathbb{R}^4_{[0, 1]}\), the transition probabilities depend on
the payoff matrix \(A ^ {(K)}\) where:
}

\begin{itemize}
    \item \DIFadd{\(A ^ {(K)}_{11}=u_{p}(p)\) is the long run utility of the best response player against itself.
    }\item \DIFadd{\(A ^ {(K)}_{12}=u_{q}(p)\) is the long run utility of the best response player against the opponent.
    }\item \DIFadd{\(A ^ {(K)}_{11}=u_{p}(q)\) is the long run utility of the opponent against the best response player.
    }\item \DIFadd{\(A ^ {(K)}_{11}=u_{q}(q)\) is the long run utility of the opponent against itself.
}\end{itemize}

\DIFadd{The matrix \(A ^ {(K)}\) is calculated using Equation~(\ref{eq:optimisation_quadratic}).
For every value of \(K\) the best response dynamics algorithm
(Algorithm~\ref{algo:best_response_dynamics})
is used to
calculate the best response player.
}

\DIFadd{The total utilities/fitnesses for each player can be written down:
}

\[\DIFadd{f_1^{(K)} = (K - 1) A_{11}^{(K)} + (n - K)A_{12}^{(K)}}\]

\[\DIFadd{f_2^{(K)} = (K) A_{21}^{(K)} + (n - K - 1)A_{22}^{(K)}}\]

\DIFadd{where \(f_1^{(K)}\) is the fitness of the best response player, and \(f_2^{(K)}\) is
the fitness of the opponent.
}

\DIFadd{Using this:
}

\[ \DIFadd{p_{K, K - 1} = \frac{ (n - K)f_2^{(K)} }{ Kf_1^{(K)}+(n - K)f_2^{(K)} } \frac{ K }{ n } }\]

\DIFadd{and:
}

\[ \DIFadd{p_{K, K + 1} = \frac{ Kf_1^{(K)} }{ Kf_1^{(K)}+(n - K)f_2^{(K)} } \frac{ (n - K) }{ n } }\]

\DIFadd{which are all that are required to calculate \(x_K\).
}

\DIFadd{Figure~\ref{fig:dynamic_moran_process_results} shows the results of an analysis
of \(x_K\) for dynamically updating players.
This is obtained over 182
Moran process against 122 randomly selected opponents.
For each Moran process the fixation probabilities for \(K\in\{1, 2, 3\}\) are
collected.
As well as recording \(x_K\), \(\tilde x_K\) is measured where \(\tilde
x_K\) represents the fixation probability of the best response player
calculated for a given \(K\) but not allowing it to dynamically update as the
population changes. The ratio \(\frac{x_K}{\tilde x_K}\) is included in the
Figure. This is done to be able to compare to a high performing strategy that
has a theory of mind of the opponent but not of the population density.
The ratio shows a relatively high performance compared to a non
dynamic best response strategy. The mean ratio over all values of \(K\) and all
experiments is \(1.044\).
In some cases this dynamic updating results in a 25\% increase in the
absorption probability.
}

\DIFadd{As denoted before it is clear that the best response strategy in general does not have a
low SSE (only 25\% of the data is below .923 and the average is .454) this is
further compounded by the ratio being above one showing that in many
cases the dynamic strategy benefits from its ability to adapt.
This indicates that memory-one strategies that perform well in Moran processes need to
more adaptable than a ZDs, and aim for mutual cooperation as well as
exploitation which is in line with the results of~\mbox{%DIFAUXCMD
\cite{Hilbe2018} }\hspace{0pt}%DIFAUXCMD
where their
strategy was designed to adapt and was shown to be evolutionary stable. The
findings of this work show that an optimal strategy acts in the same way.
}

\begin{figure}[!hbtp]
    \centering
    \includegraphics[width=.9\textwidth]{Figure_3.pdf}
    \caption{\DIFaddFL{Results for the best response player in a dynamic Moran process.
    The ratio is taken as the ratio of \(x_k\) of the dynamically updating
    player to the fixation probability of a best response player that does not
    update as the population density changes.}}\label{fig:dynamic_moran_process_results}
\end{figure}


\DIFaddend \section*{Discussion}

This manuscript has considered \textit{best response} strategies in the IPD game, and
more specifically, \textit{memory-one best responses}. It has proven that:

\begin{itemize}
    \item The utility
          of a memory-one strategy against a set of memory-one opponents can be written as a sum
          of ratios of quadratic forms (Theorem~\ref{theorem_one}).
    \item There is a compact way of identifying a memory-one best response to a
        group of opponents through a search over a discrete set
        (Theorem~\ref{memone_group_best_response}).
\end{itemize}

\DIFdelbegin \DIFdel{Note that Theorem~\ref{memone_group_best_response} which }\DIFdelend \DIFaddbegin \DIFadd{There is one further theoretical result that can be obtained from Theorem~\ref{theorem_one},
which allows the identification of
environments for which cooperation cannot occur (Details are in the
Supplementary Information). Moreover,
Theorem~\ref{memone_group_best_response} }\DIFaddend does not only have game theoretic
novelty, but also the mathematical novelty of solving quadratic ratio
optimisation problems where the quadratics are non concave.

\DIFdelbegin \DIFdel{Moreover Theorem~\ref{theorem_one}, allows us to
obtain a condition for which in an environment of }\DIFdelend \DIFaddbegin \DIFadd{The empirical results of the manuscript investigated the behaviour of }\DIFaddend memory-one
\DIFdelbegin \DIFdel{opponents
defection is the stable choice, based only on the coefficients of the opponents,
as stated in Lemma~\ref{lemma:stability_of_defection}.
}%DIFDELCMD < 

%DIFDELCMD < \begin{lemma}\label{lemma:stability_of_defection}
%DIFDELCMD <     %%%
\DIFdel{In a tournament of \(N\) players \(\{q^{(1)}, q^{(2)}, \dots, q^{(N)} \}\)
    for \(q^{(i)} \in \R_{[0, 1]} ^ 4\)
    defection is stable if the transition probabilities of the
    opponents satisfy conditions Equation~(\ref{eq:defection_condition_one}) and Equation~(\ref{eq:defection_condition_two}). }%DIFDELCMD < 

%DIFDELCMD <     %%%
\begin{displaymath}\DIFdel{\label{eq:defection_condition_one}
        \sum_{i=1} ^ N (c^{(i)} \bar{a}^{(i)} - \bar{c}^{(i)} a^{(i)}) \leq 0
    }\end{displaymath}%DIFAUXCMD
%DIFDELCMD < 

%DIFDELCMD <     %%%
\DIFdel{while,
}%DIFDELCMD < 

%DIFDELCMD <     %%%
\begin{displaymath}\DIFdel{\label{eq:defection_condition_two}
        \sum_{i=1} ^ N \bar{a}^{(i)} \neq 0
    }\end{displaymath}%DIFAUXCMD
%DIFDELCMD < \end{lemma}
%DIFDELCMD < 

%DIFDELCMD < %%%
\DIFdel{The proof of Lemma~\ref{lemma:stability_of_defection} is given in the
Supplementary Information and a numerical simulation
demonstrating the result is given
in Figure~\ref{fig:stability_of_defection}.
}%DIFDELCMD < 

%DIFDELCMD < \begin{figure}[!htbp]
%DIFDELCMD <     \centering
%DIFDELCMD <     \includegraphics[width=.4\linewidth]{Figure_4}
%DIFDELCMD <     %%%
%DIFDELCMD < \caption{%
{%DIFAUXCMD
\DIFdelFL{A. For \(q_{1}=(0.22199, 0.87073, 0.20672, 0.91861)\),
    $q_{2}=(0.48841, 0.61174, 0.76591, 0.51842)$ and
    $q_{3}=(0.2968, 0.18772, 0.08074, 0.73844)$, Equation~(\ref{eq:defection_condition_one}) and
    Equation~(\ref{eq:defection_condition_two}) hold and Defector takes over the
    population. B. For $q_{1}=(0.96703, 0.54723, 0.97268, 0.71482)$,
    $q_{2}=(0.69773, 0.21609, 0.97627, 0.0062)$ and
    $q_{3}=(0.25298, 0.43479, 0.77938, 0.19769)$, Equation~(\ref{eq:defection_condition_one}) fails
    and Defector does not take over the population.
    These results have been obtained by using~\mbox{%DIFAUXCMD
\cite{axelrodproject} }\hspace{0pt}%DIFAUXCMD
an open
    source research framework for the study of the IPD. }%DIFDELCMD < \figureacknowledgement%%%
}%DIFAUXCMD
%DIFDELCMD < \label{fig:stability_of_defection}
%DIFDELCMD < \end{figure}
%DIFDELCMD < 

%DIFDELCMD < %%%
\DIFdelend \DIFaddbegin \DIFadd{strategies and their limitations. }\DIFaddend The empirical results have shown that the
performance \DIFdelbegin \DIFdel{and the evolutionary
stability }\DIFdelend of memory-one strategies rely on adaptability and not on extortion,
and that memory-one strategies' performance is limited by their memory in cases
where they interact with multiple opponents. \DIFdelbegin %DIFDELCMD < 

%DIFDELCMD < %%%
\DIFdel{These results were mainly to investigate the
behaviour of }\DIFdelend \DIFaddbegin \DIFadd{These relied on two bespoke
data sets of 1000 and 152 pairs of }\DIFaddend memory-one \DIFdelbegin \DIFdel{strategies and their limitations.
A large data set which
contained best responses in tournaments whilst including or not self
interactions for \(N=2\)
}\DIFdelend \DIFaddbegin \DIFadd{opponents equivalently, archived
at~\mbox{%DIFAUXCMD
\cite{glynatsi2019}}\hspace{0pt}%DIFAUXCMD
.
}

\DIFadd{A further set of results for Moran processes with a dynamically updating best
response player }\DIFaddend was generated and is archived in~\DIFdelbegin \DIFdel{\mbox{%DIFAUXCMD
\cite{glynatsi2019}}\hspace{0pt}%DIFAUXCMD
.
This allowed us to investigate their respective behaviours, and
whether it was extortionate acts that made them the most favorable strategies.
It was shown that it was not extortionbut adaptability that allowed
the strategies to gain the most from their interactions. In settingswith self interactions
there
is some evidence that it is more likely to forgive after being tricked}\DIFdelend \DIFaddbegin \DIFadd{\mbox{%DIFAUXCMD
\cite{nikoleta_glynatsi_2020}}\hspace{0pt}%DIFAUXCMD
.
This confirmed the previous results which is that high performance
requires adaptability and not extortion}\DIFaddend . \DIFaddbegin \DIFadd{It also provides a framework for future
stability of optimal behaviour in evolutionary settings.
}\DIFaddend 

\DIFdelbegin \DIFdel{All the empirical results presented in this manuscript have been for the case of $N=2$}\DIFdelend \DIFaddbegin \DIFadd{In the interactions we have considered here the players do not make mistakes; their
actions were executed with perfect accuracy. Mistakes, however, are relevant in
the reasearch of repeated games~\mbox{%DIFAUXCMD
\cite{Boyd1989, Imhof2007, Nowak1993, Wu1995}}\hspace{0pt}%DIFAUXCMD
}\DIFaddend . In
future work we would consider \DIFdelbegin \DIFdel{larger values of $N$, however, we
believe that for larger values of $N$ the results that have been presented here
would only be more evident.
In addition, we would investigate potential
theoretical results for the best responses dynamics algorithm
discussed. Another interesting avenue }\DIFdelend \DIFaddbegin \DIFadd{interactions with ``noise''. Noise can be
incoroporated into our formulation and it can be shown that the utility remains a ratio
of quadratic forms (Details see the Supplementary Information).
Another avenue of investigation }\DIFaddend would be to \DIFdelbegin \DIFdel{study the Moran process with a dynamically updating best response }\DIFdelend \DIFaddbegin \DIFadd{understand if and/or when an
evolutionary trajectory leads to a best response strategy}\DIFaddend .

By specifically exploring the entire space of memory-one strategies to identify
the best strategy for a variety of situations, this work adds to the literature
casting doubt
on the effectiveness of ZDs, highlights the importance of adaptability and provides
a framework for the continued understanding of these important questions.

\bibliography{bibliography}

\section*{Acknowledgements}

A variety of software libraries have been used in this work:

\begin{itemize}
    \item The Axelrod library for IPD simulations~\cite{axelrodproject}.
    \item The Scikit-optimize library for an implementation of Bayesian optimisation~\cite{tim_head_2018_1207017}.
    \item The Matplotlib library for visualisation~\cite{hunter2007matplotlib}.
    \item The SymPy library for symbolic mathematics~\cite{sympy}.
    \item The Numpy library for data manipulation~\cite{walt2011numpy}.
\end{itemize}

\section*{Author contributions statement}

N.G. and V.K. conceived the idea. N.G. conducted the experiments, N.G. and V.K.
analysed the results. All authors reviewed the manuscript.

\section*{Additional information}

\textbf{Competing interests.} The author(s) declare no competing interests.

\end{document}