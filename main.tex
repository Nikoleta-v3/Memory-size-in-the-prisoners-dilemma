\documentclass[10pt]{article}

% Manage page layout
\usepackage[margin=2.5cm, includefoot, footskip=30pt]{geometry}
\pagestyle{plain}
\setlength{\parindent}{0em}
\setlength{\parskip}{1em}
\renewcommand{\baselinestretch}{1}

%%%%%%%PACKAGES HERE%%%%%%%
\usepackage{tikz}
\usepackage{amsmath}
\usepackage{amssymb}
\usepackage{amsthm}
\usepackage{graphicx}
\usepackage{subcaption}
\usepackage{standalone}
\usepackage{booktabs}
\usepackage{algorithm, setspace}
\usepackage[noend]{algpseudocode}

\makeatletter
\def\BState{\State\hskip-\ALG@thistlm}
\makeatother

\newcommand{\R}{\mathbb{R}}
\newtheorem{theorem}{Theorem}
\usetikzlibrary{decorations.pathmorphing, decorations.pathreplacing, angles,
                quotes, calc, er, positioning}

\newtheorem{lemma}[theorem]{Lemma}
\def\arraystretch{1.5}

\title{Stability of defection, optimisation of strategies and the limits of
       memory in the Prisoner's Dilemma.}
\author{Nikoleta E. Glynatsi \and Vincent A. Knight}
\date{}

\begin{document}

\maketitle

\begin{abstract}
    This manuscript builds upon a framework provided in 1989 to study best
    responses in the well known memory one strategies of the Iterated Prisoner's
    Dilemma. The aim of this work is to construct a compact way of identifying
    best responses of short memory strategies and to show their limitations in
    multi-opponent interactions. A number of theoretic results are presented.
    %TODO: Expand when we get results
\end{abstract}

\section{Introduction}\label{section:introduction}

The Prisoner's Dilemma (PD) is a two player game used in understanding the
evolution of co-operative behaviour, formally introduced in~\cite{Flood1958}.
Each player has two options, to cooperate (C) or to defect (D). The decisions
are made simultaneously and independently. The normal form representation of the
game is given by:

\begin{equation}\label{equ:pd_definition}
    S_p =
    \begin{pmatrix}
        R & S  \\
        T & P
    \end{pmatrix}
    \quad
    S_q =
    \begin{pmatrix}
        R & T  \\
        S & P
    \end{pmatrix}
\end{equation}

where \(S_p\) represents the utilities of the row player and \(S_q\) the
utilities of the column player. The payoffs, \((R, P, S, T)\), are constrained
by equations~(\ref{eq:pd_constrain_one}) and~(\ref{eq:pd_constrain_two}).
Constraint~(\ref{eq:pd_constrain_one}) ensures that
defection dominates cooperation and constraint~(\ref{eq:pd_constrain_two})
ensures that there is a dilemma; the sum of the utilities for both players is
better when both choose to cooperate. The most common values used in the literature are
\((3, 1, 0, 5)\)~\cite{Axelrod1981}.


\begin{equation}\label{eq:pd_constrain_one}
    T > R > P > S
\end{equation}

\begin{equation}\label{eq:pd_constrain_two}
    2R > T + S
\end{equation}

The PD is a one shot game, however it is commonly studied in a manner where the
history of the interactions matter. The repeated form of the game is called the
Iterated Prisoner's Dilemma (IPD) and in the 1980s, following the work
of~\cite{Axelrod1980a, Axelrod1980b} it attracted the attention of the
scientific community. In~\cite{Axelrod1980a} and~\cite{Axelrod1980b}, the first
well known computer tournaments of the IPD were performed. A total of 13 and 63
strategies were submitted, respectively, in the form of computer code and
competed against each other in a round robin tournament. The contestant competed
against each other, a copy of themselves and a random strategy and the winner
was decided on the average score a strategy achieved (not the total number of
wins). The contestants were given access to the entire history of a match,
however, how many turns of history a strategy would incorporate, the memory
size, was a result of the particular strategic decisions made by the author.

The winning strategy of both tournaments was the strategy called Tit for Tat.
Tit for Tat is a strategy which starts by cooperating and then mimics the last
move of it's opponent, specifically, is a strategy that considers only the
previous move of the opponent, these type of strategies are called
\textit{reactive}~\cite{Nowak1989}. Reactive strategies are a subset of so
called \textit{memory one} strategies. Memory one strategies similarly are only
concerned with the previous turn, however, they incorporate both players' recent
moves. Several successful memory one and reactive strategies are found in the
literature, such as Pavlov~\cite{Nowak1993} and Generous Tit For
Tat~\cite{Nowak1990}.

In 2012, Press and Dyson published~\cite{Press2012} and created a small shock in
the game theoretic community,~\cite{Stewart2012} stated that ``Press and Dyson
have fundamentally changed the viewpoint on the Prisoner's Dilemma''. They
introduced a set of memory one strategies that chose their actions so that a
linear relationship is forced between their score and that of the opponent.
These strategies are called zero determinate.

\begin{itemize}
    \item One specific advantage
    of memory one strategies is their mathematical tractability, as described in
    Section~\ref{section:utility} they can be represented completely as an vector of
    \(\R^{4}\)
    \item They are robust when it comes to pairwise interactions
    \item They are mathematically good.
\end{itemize}

Several works have questioned the effectiveness of such strategies.

The purpose of this work is to consider a given memory one strategy in a similar
fashion to~\cite{Press2012}, however whilst~\cite{Press2012} found a way for a
player to manipulate a given opponent, this work will consider a
multidimensional optimisation approach to identify the best response to a group
of opponents. The aim is to produce a compact method of identifying these best
response memory one strategy against opponents. In later parts of the paper the
best response strategies are compared to extortionate and complex ones.

These are achieved by comparing the performance of an optimal memory one
strategy, for a given environment, with the performance of a more complex
strategy that has a larger memory. One particular benefit of this analysis is
the identification of conditions for which defection is a best response. Thus,
identifying environments for which cooperation can not occur.

\section{The utility}\label{section:utility}

In~\cite{Nowak1989} it is stated that if a strategy is concerned with only the
outcome of a single turn then there are four possible `states' the strategy
could be in; \(CC, CD, DC,CC\). Therefore, a memory one strategy can denoted by
the probability of vector of cooperating after each of these states; \(p=(p_1, p_2, p_3,
p_4) \in \R_{[0,1]} ^ 4\).
In an IPD match two memory one strategies are moving from state to state at each
turn with a given probability. This exact behaviour can be modelled as a stochastic
process, and more specifically as a Markov chain (Figure~\ref{fig:markov_chain}).
The corresponding transition matrix \(M\) of Figure~\ref{fig:markov_chain} is
given by:

\begin{figure}
    \begin{minipage}{0.35\textwidth}
        \includestandalone[width=\textwidth]{tex/markov_chain}
        \caption{markov}
        \label{fig:markov_chain}
    \end{minipage}
    \begin{minipage}{0.45\textwidth}
    \begin{equation*}
        M = [[p_1*q_1 p_1*(-q_1 + 1) q_1*(-p_1 + 1) (-p_1 + 1)*(-q_1 + 1)]
 [p_2*q_3 p_2*(-q_3 + 1) q_3*(-p_2 + 1) (-p_2 + 1)*(-q_3 + 1)]
 [p_3*q_2 p_3*(-q_2 + 1) q_2*(-p_3 + 1) (-p_3 + 1)*(-q_2 + 1)]
 [p_4*q_4 p_4*(-q_4 + 1) q_4*(-p_4 + 1) (-p_4 + 1)*(-q_4 + 1)]]
    \end{equation*}
    \end{minipage}
\end{figure}

The long run steady state probability \(v\) is the solution to \(v M = v\). The
stationary vector \(v\) can be combined with the payoff matrices of
equation~(\ref{equ:pd_definition}) and the expected payoffs for each player
can be estimated without simulating the actual interactions. More
specifically, the utility for a memory one strategy \(p\) against an opponent \(q\),
denoted as \(u_q(p)\), is defined by,

\begin{equation}\label{eq:press_dyson_utility}
    u_q(p) = v \times (R, P, S, T).
\end{equation}

In Theorem~\ref{theorem:quadratic_form_u}, the first theoretical results of
the manuscript is presented, that is that \(u_q(p)\) is given by a ratio of
two quadratic forms~\cite{kepner2011}. To the authors knowledge this is the
first work that has been done on the form of \(u_q(p)\).

\begin{theorem}\label{theorem:quadratic_form_u}
    The expected utility of a memory one strategy \(p\in\mathbb{R}_{[0,1]}^4\)
    against a memory one opponent \(q\in\mathbb{R}_{[0,1]}^4\), denoted
    as \(u_q(p)\), can be written as a ratio of two quadratic forms:

    \begin{equation}\label{eq:optimisation_quadratic}
    u_q(p) = \frac{\frac{1}{2}pQp^T + cp + a}
                {\frac{1}{2}p\bar{Q}p^T + \bar{c}p + \bar{a}},
    \end{equation}
    where \(Q, \bar{Q}\) \(\in \R^{4\times4}\) are matrices defined by the
    transition probabilities of the opponent \(q_1, q_2, q_3, q_4\) as follows:

    \begin{center}
    \begin{equation}
    \resizebox{0.9\linewidth}{!}{\arraycolsep=2.5pt%
    \boldmath\(
    Q = \left[\begin{matrix}0 & - \left(q_{1} - q_{3}\right) \left(P q_{2} - P - T q_{4}\right) & \left(q_{1} - q_{2}\right) \left(P q_{3} - S q_{4}\right) & \left(q_{1} - q_{4}\right) \left(S q_{2} - S - T q_{3}\right)\\- \left(q_{1} - q_{3}\right) \left(P q_{2} - P - T q_{4}\right) & 0 & \left(q_{2} - q_{3}\right) \left(P q_{1} - P - R q_{4}\right) & - \left(q_{3} - q_{4}\right) \left(R q_{2} - R - T q_{1} + T\right)\\\left(q_{1} - q_{2}\right) \left(P q_{3} - S q_{4}\right) & \left(q_{2} - q_{3}\right) \left(P q_{1} - P - R q_{4}\right) & 0 & \left(q_{2} - q_{4}\right) \left(R q_{3} - S q_{1} + S\right)\\\left(q_{1} - q_{4}\right) \left(S q_{2} - S - T q_{3}\right) & - \left(q_{3} - q_{4}\right) \left(R q_{2} - R - T q_{1} + T\right) & \left(q_{2} - q_{4}\right) \left(R q_{3} - S q_{1} + S\right) & 0\end{matrix}\right]\)},
    \end{equation}
    \begin{equation}\label{eq:q_bar_matrix}
    \resizebox{0.8\linewidth}{!}{\arraycolsep=2.5pt%
    \boldmath\(
    \bar{Q} =  \left[\begin{matrix}0 & - \left(q_{1} - q_{3}\right) \left(q_{2} - q_{4} - 1\right) & \left(q_{1} - q_{2}\right) \left(q_{3} - q_{4}\right) & \left(q_{1} - q_{4}\right) \left(q_{2} - q_{3} - 1\right)\\- \left(q_{1} - q_{3}\right) \left(q_{2} - q_{4} - 1\right) & 0 & \left(q_{2} - q_{3}\right) \left(q_{1} - q_{4} - 1\right) & \left(q_{1} - q_{2}\right) \left(q_{3} - q_{4}\right)\\\left(q_{1} - q_{2}\right) \left(q_{3} - q_{4}\right) & \left(q_{2} - q_{3}\right) \left(q_{1} - q_{4} - 1\right) & 0 & - \left(q_{2} - q_{4}\right) \left(q_{1} - q_{3} - 1\right)\\\left(q_{1} - q_{4}\right) \left(q_{2} - q_{3} - 1\right) & \left(q_{1} - q_{2}\right) \left(q_{3} - q_{4}\right) & - \left(q_{2} - q_{4}\right) \left(q_{1} - q_{3} - 1\right) & 0\end{matrix}\right]\)}.
    \end{equation}
    \end{center}

    \(c \text{ and } \bar{c}\) \(\in \R^{4 \times 1}\) are similarly defined by:

    \begin{equation}\label{eq:q_matrix_numerator}
    \resizebox{0.3\linewidth}{!}{\arraycolsep=2.5pt%
    \boldmath\(c = \left[\begin{matrix}- 5 q_{1} q_{4}\\5 q_{4} \left(q_{3} - 1\right)\\q_{4} \left(2 q_{2} + 1\right)\\5 q_{1} q_{4} - 2 q_{2} q_{4} - q_{2} - 5 q_{3} q_{4} + 5 q_{3} - 3 q_{4} + 1\end{matrix}\right]\),}
    \end{equation}
    \begin{equation}\label{eq:q_matrix_denominator}
    \resizebox{0.3\linewidth}{!}{\arraycolsep=2.5pt%
    \boldmath\(\bar{c} = \left[\begin{matrix}q_{1} \left(q_{2} - q_{4} - 1\right)\\- \left(q_{3} - 1\right) \left(q_{2} - q_{4} - 1\right)\\- q_{1} q_{2} + q_{2} q_{3} + q_{2} - q_{3} + q_{4}\\q_{1} q_{4} - q_{2} - q_{3} q_{4} + q_{3} - q_{4} + 1\end{matrix}\right]\).
    }
    \end{equation}
    and \(a = 5 q_{4}\) and
    \(\bar{a} = - q_{2} + q_{4} + 1\).
\end{theorem}

The proof of Theorem~\ref{theorem:quadratic_form_u} is given in Appendix. %TODO write proof

Numerical simulations have been carried out to validate the formulation of
\(u_q(p)\) as a quadratic ratio, a data set is available at. Two examples are
graphically represented in Figure~\ref{fig:analytical_simulated} and show that
the formulation successfully captures the simulated behaviour. The simulated
utility, which is denoted as \(U_q(p)\), has been calculated
using~\cite{axelrodproject}, an open source research framework for the study of
the IPD. The project is described in~\cite{Knight2016}.

\begin{figure}[!htbp]
\begin{center}
    \begin{subfigure}{0.45\textwidth}
        \includegraphics[width=\linewidth]{img/validation_against_player_one.pdf}
    \end{subfigure}
    \begin{subfigure}{0.45\textwidth}
        \includegraphics[width=\linewidth]{img/validation_against_player_two.pdf}
    \end{subfigure}
\end{center}

\caption{Differences between simulated and analytical results for
         \(q = (\frac{1}{3}, \frac{1}{3}, \frac{1}{3}, q_4)\).}
\label{fig:analytical_simulated}
\end{figure}

Theorem~\ref{theorem:quadratic_form_u} can be extended to consider multiple
opponents. The IPD is commonly studied in tournaments and/or Moran Processes
where a strategy interacts with a number of opponents. The payoff of a player is
then given by the average payoffs the against each opponent. More specifically
the expected utility of a memory one strategy against a \(N\) number of
opponents is given by Theorem~\ref{theorem:tournament_utility}.

\begin{theorem}\label{theorem:tournament_utility}
    The expected utility of a memory one strategy \(p\in\mathbb{R}_{[0,1]}^4\)
    against a group of opponents \(q^{(1)}, q^{(2)}, \dots, q^{(N)}\), denoted
    as \(\frac{1}{N} \sum\limits_{i=1} ^ {N} {u_q}^{(i)} (p)\) is given by:

    \begin{equation}\label{eq:tournament_utility}
        \frac{1}{N} \sum\limits_{i=1} ^ {N} {u_q}^{(i)} (p) = \frac{1}{N}
        \frac{\sum\limits_{i=1} ^ {N} (\frac{1}{2} pQ^{(i)} p^T + c^{(i)} p + a^ {(i)})
        \prod\limits_{\tiny\begin{array}{l} j=1 \\ j \neq i \end{array}} ^
        N (\frac{1}{2} p\bar{Q}^{(i)} p^T + \bar{c}^{(i)} p + \bar{a}^ {(i)})}
        {\prod\limits_{i=1} ^ N (\frac{1}{2} p\bar{Q}^{(i)} p^T + \bar{c}^{(i)} p + \bar{a}^ {(i)})}.
    \end{equation}
\end{theorem}

As an illustration, Theorem~\ref{theorem:tournament_utility} is used to
calculate the theoretical payoffs of several memory one strategies against a set
of 10 opponents. The opponents used are the memory one strategies for the
tournament conducted in~\cite{Stewart2012}. This tournament is also simulated as
before The names and a small explanation of the strategic rules are given in
Appendix~\ref{appendix:tables}. The values of \(\sum\limits_{i=1} ^ {N} u_{q
^{(i)}} (p)\) and \(\sum\limits_{i=1} ^ {N} U_{q ^{(i)}} (p)\) match
(Table~\ref{table:list_stewart_plotkin}),

\begin{table}
        \begin{center}
        \resizebox{.5\textwidth}{!}{\begin{tabular}{clcc}
        \toprule
        {}&  Name & Memory one representation & Reference \\
        \midrule
        1  & Cooperator           & \((1, 1, 1, 1)\) & \cite{Axelrod1981} \\
        2  & Defector             & \((0, 0, 0, 0)\) & \cite{Axelrod1981}\\
        3  & Random               & \((\frac{1}{2}, \frac{1}{2}, \frac{1}{2},
        \frac{1}{2})\) & \cite{Axelrod1981} \\
        4  & Tit for Tat          & \((1, 0, 1, 0)\) & \cite{Axelrod1981}\\
        5  & Grudger              & \((1, 0, 0, 0)\) & \cite{Li2011} \\
        6  & Generous Tit for Tat & \((1, \frac{1}{3}, 1, \frac{1}{3})\) & \cite{Nowak1990}\\
        7  & Win Stay Lose Shift  & \((1, 0, 0, 1)\) & \cite{Nowak1993} \\
        8  & ZDGTFT2              & \((1, \frac{1}{8}, 1, \frac{1}{4})\) &\cite{Stewart2012}\\
        9  & ZDExtort2            & \((\frac{8}{9}, \frac{1}{2}, \frac{1}{3},
        0)\) & \cite{Stewart2012}\\
        10 & Hard Joss            & \((\frac{9}{10}, 0, \frac{9}{10}, 0)\) &
        \cite{Stewart2012} \\
        \bottomrule
    \end{tabular}}
    \caption{List of strategies used in the tournament described in~\cite{Stewart2012}.}
    \label{table:list_stewart_plotkin}
    \end{center}
\end{table}



\begin{table}[htbp]
    \begin{center}
    \begin{tabular}{lllllll}
\toprule
{} & $p_1$ & $p_2$ & $p_3$ & $p_4$ & $\frac{1}{10} \sum \limits_{i = 1}^{10} u_q^{(i)} (p)$ & $\frac{1}{10} \sum\limits_{i = 1} ^{10} U_q^{(i)} (p)$ \\
\midrule
0  &   1/5 &   1/5 &   1/5 &   1/5 &                                            2.07675 &                                            2.07173 \\
1  &   1/5 &   1/5 &   1/5 &   2/5 &                                            2.12906 &                                            2.12359 \\
2  &   1/5 &   1/5 &   1/5 &   3/5 &                                            2.15484 &                                            2.15064 \\
3  &   1/5 &   1/5 &   1/5 &   4/5 &                                            2.17088 &                                            2.16632 \\
4  &   1/5 &   1/5 &   2/5 &   1/5 &                                            2.05783 &                                            2.05034 \\
5  &   1/5 &   1/5 &   2/5 &   2/5 &                                            2.10907 &                                            2.10407 \\
6  &   1/5 &   1/5 &   2/5 &   3/5 &                                            2.13769 &                                            2.13023 \\
7  &   1/5 &   1/5 &   2/5 &   4/5 &                                            2.14697 &                                            2.14472 \\
8  &   1/5 &   1/5 &   3/5 &   1/5 &                                            2.06018 &                                            2.05372 \\
9  &   1/5 &   1/5 &   3/5 &   2/5 &                                            2.10985 &                                             2.1047 \\
10 &   1/5 &   1/5 &   3/5 &   3/5 &                                             2.1285 &                                            2.12679 \\
11 &   1/5 &   1/5 &   3/5 &   4/5 &                                             2.1431 &                                              2.138 \\
12 &   1/5 &   1/5 &   4/5 &   1/5 &                                            2.09328 &                                            2.08564 \\
13 &   1/5 &   1/5 &   4/5 &   2/5 &                                            2.13008 &                                            2.12497 \\
14 &   1/5 &   1/5 &   4/5 &   3/5 &                                            2.14007 &                                            2.13836 \\
\bottomrule
\end{tabular}

    \end{center}
    \caption{Results of memory one strategies against the strategies in Table~\ref{table:list_stewart_plotkin}.}
    \label{table:list_stewart_plotkin}
\end{table}

% TODO This needs to be either a single number: the utility in the tournament
% **OR** parametrize the strategies (change 0 <= p_1 <= 1) and get a 1
% dimensional plot against one of those parameters.

Note that the utility against a group of strategies can not be captured by
the utility against the mean opponent.

\begin{equation}\label{eq:tournament_hypothesis}
    \frac{1}{N} \sum_{i=1} ^ {N} {u_q}^{(i)} (p) \neq
      u_{\frac {1}{N} \sum\limits_{i=1} ^ N q^{(i)}}(p).
\end{equation}

This is illustrated in Figure~\ref{fig:hypothesis}.

\begin{figure}[!htbp]
    \begin{center}
            \includegraphics[width=\linewidth]{img/mean_vs_average_heatmap.pdf}
    \end{center}
    \caption{Plotting the difference of the average utility against ten
    opponents versus the utility against the average of the ten strategies.}
    \label{fig:hypothesis}
\end{figure}

The analytical formulation of Theorem~\ref{theorem:tournament_utility} will be
used in the following sections to explore the best response to memory one
strategies.

Furthermore, the iterated prisoner's dilemma is commonly stydied in evolutioary
settings~\cite{}. In such settings self interaction are taken into account. In
tournaments as well, as described before, in axelrod tournament the straetgies
compitected against a copy of themselves. However in evolutionary settings
being good against your self can lead to evolutionary robystness. Thus utility
can now be re written as.


\section{Stability of defection}

In this section the stability of defection is explored. Defection is
known to be the dominant action in the PD and it can be proven to be the dominant
strategy for the IPD for given environments. Even so, several works have proven
that cooperation emerges in the IPD and many studies focus on the
emergence of cooperation. In this manuscript we try to provide a condition for when
defection is the best response in the IPD, thus when it is known that
cooperation is not dominant.

Initially, let us consider equation~(\ref{eq:mo_derivative}) for \(p = (0, 0, 0,
0)\),

\begin{equation}\label{eq:derivative_of_quadratic_zero}
    \begin{aligned}
     \frac{du}{dp_{| p=(0, 0, 0, 0)}} & = && \frac{c \bar{a} - \bar{c}a}
      {\bar{a}^2} .\\
    \end{aligned}
\end{equation}

The numerator \(\bar{c}a - c\bar{a}\) is given by,

\[\input{tex/defection_matrix.txt}\]

and the denominator \(\bar{a} ^ 2 = (-q_2 + q_4 + 1) ^ 2\), which is always positive. In order
for defection to be the best response the derivative must have a negative
sign at the point \(p = (0, 0, 0, 0)\). That means that the utility is only
decreasing after \(p = (0, 0, 0, 0)\).

Because \(\bar{a} ^ 2\) is always positive the sign of the derivative is given by \(\bar{c}a - c\bar{a}\).
More specifically from equations,

\begin{equation}\label{eq:defection_condition_one}
    \input{tex/defection_condition_one.txt}
\end{equation}
\begin{equation}\label{eq:defection_condition_two}
    \input{tex/defection_condition_two.txt}
\end{equation}

Both signs of the partial derivatives must be negative in order for the overall
function to be decreasing ensuring defection is a best response.
The signs of equations (\ref{eq:defection_condition_one}) and (\ref{eq:defection_condition_two})
vary. There are cases that they have the same sign and cases that they do not,

For a tournament setting we substitute \(p = (0, 0, 0, 0)\) in
equation~(\ref{eq:mo_tournament_derivative}) which gives:

\begin{equation}
\sum_{i=1} ^ N (c^{(i)T} \bar{a}^{(i)} - \bar{c}^{(i)T} a^{(i)})
\prod\limits_{\tiny\begin{array}{l} j=1 \\ j \neq i \end{array}} ^ N (\bar{a}^{(i)})^2
\end{equation}

The second term \(\prod\limits_{\tiny\begin{array}{l} j=1 \\ j \neq i
\end{array}} ^ N (\bar{a}^{(i)})^2\) is always positive. However the sign of the
first term \(\sum_{i=1} ^ N (c^{(i)T} \bar{a}^{(i)} - \bar{c}^{(i)T} a^{(i)})\)
can vary based on the transition probabilities of the opponents, as discussed
above. A condition that must hold in order for defection to be stable in a
tournament is that each term of the sum must be negative. The results are
exhibited in Lemma~\ref{lemma:stability_of_defection}.

\begin{lemma}\label{lemma:stability_of_defection}
    In a tournament of \(N\) players where \(q^{(i)} = (q_{1}^{(i)}, q_{2}^{(i)}, q_{3}^{(i)}, q_{4}^{(i)})\)
    defection is a best response if the transition probabilities of the
    opponents satisfy the condition:

    \begin{equation}
        \sum_{i=1} ^ N (c^{(i)T} \bar{a}^{(i)} - \bar{c}^{(i)T} a^{(i)}) \leq 0
    \end{equation}
\end{lemma}

Moreover lets us consider a constrained version of the problem once again. Lets us
assume that in an pairwise interaction the opponent is a reactive player \(q=(q_1, q_2, q_1, q_2)\).
By substituting \(q_3=q_1\) and \(q_4=q_2\) equations (\ref{eq:defection_condition_one})
and (\ref{eq:defection_condition_two}) are now re written as follow,

\[\left[\begin{matrix}- q_{2} \left(4 q_{1} - 5 q_{2} - 1\right)\\
\left(q_{2} - 1\right) \left(4 q_{1} - 5 q_{2} - 1\right)\end{matrix}\right]\]

\begin{lemma}
Defection is the best responses of a memory one player \(p\) against a reactive
player \(q\) if the transition probabilities of the opponent satisfy the
condition:

\begin{equation}
    4 q_{1} - 5 q_{2} - 1 > 0
\end{equation}
\end{lemma}


\section{Best responses to memory one players}\label{section:best_response_mem_one}

Identifying the \textbf{best response} to a group of memory one strategies  will
be considered as a multi dimensional optimisation problem, where a memory one
strategy \(p\) aims to optimise
\( \frac{1}{N} \sum u_{q ^{(i)}} (p)\) against a set opponents
\(\{q^{(1)}, q^{(2)}, \dots, q^{(N)} \}\).

The decision variable is the vector \(p\) and the solitary constraint is
that \(p \in \R^4_{[0, 1]} \).
The optimisation problem is given by~(\ref{eq:mo_tournament_optimisation}).

\begin{equation}\label{eq:mo_tournament_optimisation}
    \begin{aligned}
    \max_p: & \ \sum_{i=1} ^ {N} {u_q}^{(i)} (p)
    \\
    \text{such that}: & \ p \in \R_{[0, 1]}
    \end{aligned}
\end{equation}

Optimising this particular ratio of quadratic forms is not trivial.
It can be verified empirically for the case of a single opponent that there
exist at
least one point for which the definition of concavity does not hold.

There is some work on the optimisation on non concave ratios of of quadratic forms
\cite{Beck2009, Hongyan2014}, however in these both the numerator and the denominator
of the fractional problem were concave which is not true for here.
These results are established in Theorem~\ref{theorem:concavity}.

\begin{theorem}\label{theorem:concavity}
    The utility of a player \(p\) against an opponent \(q\), \(u_q (p)\) given by
    (\ref{eq:optimisation_quadratic}), is not concave. Furthermore neither the
    numeration or the denominator of (\ref{eq:optimisation_quadratic}), are concave.
\end{theorem}

\begin{proof}
    A function \(f(x)\) is said to be concave on an interval \([a, b]\) if, for any
    points \(x_1\) and \(x_2 \in [a, b]\), the function \(-f(x)\) is convex on that
    interval.

    A function \(f(x)\) is convex on an interval \([a, b]\) if for any two
    points \(x_1\) and \(x_2\) in \([a, b]\) and any \(\lambda\) where \(0 < \lambda < 1\),

    \begin{equation}\label{def:convex}
    f (\lambda x_1 + (1 - \lambda )x_2 ) \leq \lambda f (x_1 ) + (1 - \lambda )f (x_2 ).
    \end{equation}

    Let \(f\) be
    \(u_{(\frac{1}{3}, \frac{1}{3}, \frac{1}{3}, \frac{1}{3})}\) it can be shown
    that
    for \(x_1 = (\frac{1}{4}, \frac{1}{2}, \frac{1}{5} , \frac{1}{2}),
    x_2 = (\frac{8}{10}, \frac{1}{2}, \frac{9}{10} , \frac{7}{10})\) and
    \(\lambda=\frac{1}{10}\)
    condition (\ref{def:convex}) does not hold as: \(1.49 \geq 1.48\).

    In~\cite{Anton2014} it is stated that a quadratic form will
    be concave if and only if it's symmetric matrix is negative semi definite.
    A matrix\(A\) is semi-negative definite if:

    \begin{equation}\label{def:semi_negative}
    |A|_i \leq 0 \text{ for } i \text{ is odd and } |A|_i \geq 0  \text{ for } i
    \text{ is even.}
    \end{equation}

    For both \(Q\) and \(\bar{Q}\) it is exhibited that for \(i=2\) (odd):
    % TODO What do you mean by "exhibited"?

    \[|Q|_2 = - \left(q_{1} - q_{3}\right)^{2} \left(q_{2} - 5 q_{4} - 1\right)^{2},\]
    \[|\bar{Q}|_2 =- \left(q_{1} - q_{3}\right)^{2} \left(q_{2} - q_{4} - 1\right)^{2}\]

    both determinants are negative, thus the concavity condition
    (\ref{def:semi_negative})
    fails for both quadratic forms.
\end{proof}

The non concavity of \(u(p)\) indicates multiple local optimal points.
The approach taken here is to introduce a compact way of constructing
the candidate set of all local optimal points. Once
the set is defined the point that maximises (\ref{eq:tournament_utility})
corresponds to the best response strategy, this approach transforms the
continuous optimisation problem in to a discrete problem.

The problem considered is a bounded because \(p \in \R^4_{[0, 1]}\).
The candidate solutions will exist either at the boundaries
of the feasible solution space, or within that space. The method of Lagrange
Multipliers~\cite{bertsekas2014} and Karush-Kuhn-Tucker
conditions~\cite{Giorgi2016} are based on this. The Karush-Kuhn-Tucker
conditions are used here because the constraints are inequalities.

The above discussion leads to Lemma~\ref{lemma:memone_group_best_response} which
presents the best response to a group of opponents.

\begin{lemma}\label{lemma:memone_group_best_response}

    The optimal behaviour of a memory one strategy player
    \(p^* \in \R_{[0, 1]} ^ 4\)
    against a set of \(N\) opponents \(\{q^{(1)}, q^{(2)}, \dots, q^{(N)} \}\)
    for \(q^{(i)} \in \R_{[0, 1]} ^ 4\) is established by:

    \[p^* = \textnormal{argmax}(\sum\limits_{i=1} ^ N  u_q(p)), \ p \in S_q,\]

    where the set \(S_q\) is defined as

    \[
        S_q =
        \left\{p \in \mathbb{R} ^ 4\;|\;
            \begin{array}{c}
                p_i \in{0, 1}\text{ or }F_i(p_i) = 0\\
                \prod\limits_{i=1} ^ N Q_{D}^{(i)} \neq 0
            \end{array}
            % TODO We should definitely talk this one over.
            \text{ for all } i\in\{1, 2, 3, 4\}
        \right\}
    \]


    where,
    % TODO I am not sure if F is a good choice
    \begin{equation}\label{eq:group_derivative_numerator_condition}
    F=(\sum\limits_{i=1} ^ {N} Q_{N}^{(i)'} \prod_{\substack{j=1 \\ j \neq i}} ^ N Q_{D}^{(i)}
     + \sum\limits_{i=1} ^ {N} Q_{D}^{(i)'} \sum_{\substack{j=1 \\ j \neq i}} ^ {N} Q_{N}^{(i)}
    \prod_{\substack{j=1 \\ j \neq \{i, j\}}} ^ N Q_{D}^{(i)}) \times
    \prod\limits_{i=1} ^ N Q_{D}^{(i)} - (\sum\limits_{i=1} ^ {N} Q_{D}^{(i)'}
    \prod_{\substack{j=1 \\ j \neq i}} ^ N Q_{D}^{(i)}) \times
    (\sum\limits_{i=1} ^ {N} Q_{N}^{(i)} \prod_{\substack{j=1 \\ j \neq i}} ^ N Q_{D}^{(i)})
    \end{equation}

    and,
    \begin{align*}
        Q_{N}^{(i) } & = \frac{1}{2} pQ^{(i)} p^T + c^{(i)} p + a^ {(i)}, \\
        Q_{N}^{(i)'} & =  pQ^{(i)} + c^{(i)}, \\
        Q_{D}^{(i) } & = \frac{1}{2} p\bar{Q}^{(i)} p^T + \bar{c}^{(i)} p + \bar{a}^ {(i)}, \\
        Q_{D}^{(i)'} & =  p\bar{Q}^{(i)} + \bar{c}^{(i)}. \\
    \end{align*}
\end{lemma}

\begin{proof}
    The best response of a memory one strategy against a group of memory
    one strategies can captured by a candidate set of behaviours. This candidate
    set is constructed by considering behaviours where any or all of \(p_1, p_2, p_3, p_4\)
    are \(\in \{0, 1\}\) and the rest or all of \(p_1, p_2, p_3, p_4\) are given by
    roots of the partial derivatives.

    Note that for \(p_i \in \{0, 1\}\) we consider the roots of the partial derivatives
    for \(p_j \neq p_i\) for \(i,j \in [1, 4]\).

    The derivatives, \(\frac{d\sum u}{dp}\), are given by,

    {\scriptsize
    \begin{align}\label{eq:mo_tournament_derivative}
        \frac{d}{dp} \sum\limits_{i=1} ^ {N} {u_q}^{(i)} (p) & = \nonumber \\
        & =\frac{
        (\sum\limits_{i=1} ^ {N} Q_{N}^{(i)'} \prod_{\substack{j=1 \\ j \neq i}} ^ N Q_{D}^{(i)}
        + \sum\limits_{i=1} ^ {N} Q_{D}^{(i)'} \sum_{\substack{j=1 \\ j \neq i}} ^ {N} Q_{N}^{(i)}
       \prod_{\substack{j=1 \\ j \neq \{i, j\}}} ^ N Q_{D}^{(i)}) \times
       \prod\limits_{i=1} ^ N Q_{D}^{(i)} - (\sum\limits_{i=1} ^ {N} Q_{D}^{(i)'}y-vk
       \prod_{\substack{j=1 \\ j \neq i}} ^ N Q_{D}^{(i)}) \times
       (\sum\limits_{i=1} ^ {N} Q_{N}^{(i)} \prod_{\substack{j=1 \\ j \neq i}} ^ N Q_{D}^{(i)})}
        {(\prod\limits_{i=1} ^ N Q_{D}^{(i)})^{2}}
    \end{align}
    }

    For equation~\ref{eq:mo_tournament_derivative} to be zero, the numerator must fall
    to zero and the denominator can not nullified.

    One the candidate set is constructed each point is evaluated using equation
    (\ref{eq:tournament_utility}). The point with the maximum utility is selected.
\end{proof}

A special case of Lemma~\ref{lemma:memone_group_best_response} is for \(N=1\),
thus when a strategy plays against a single opponent. In this case the formulation
of Theorem~\ref{theorem:quadratic_form_u} is used and the best response is captured
by Lemma~\ref{lemma:memone_best_response}.

% TODO I wonder if moving this long list of specific results to an appendix
% would help focus the paper. (They could be an entire separate chapter in your
% thesis).
\begin{lemma}\label{lemma:memone_best_response}
    The optimal behaviour of a memory one strategy player \(p^* \in \R_{[0, 1]} ^ 4\)
    against a given opponent \(q \in \R_{[0, 1]} ^ 4\) is given by:

    \[p^* = \textnormal{argmax}(u_q(p)), \ p \in S_q,\]

    where the set \(S_q\) is defined as

    \[S_q = \{0, \bar{p}_i, 1 \}^4 \text{ for } i \in \R,\]

    where any \(\bar{p}\) satisfy condition (\ref{eq:cases_mem_one}). Note that now
    the numerators of the partial derivatives, (\ref{eq:group_derivative_numerator_condition}),
    are given by

    {\small
    \begin{equation}\label{eq:derivative_numerator_condition}
        (pQ + c) ( \frac{1}{2} p  \bar{Q}  p^T + \bar{c}  p + \bar{a})
        - (p\bar{Q} + \bar{c})( \frac{1}{2} p  Q  p^T + c p + a)
    \end{equation}}

    and (\ref{eq:group_derivative_denominator_condition}) is re-written as:

    {\small
    \begin{equation}\label{eq:derivative_denominator_condition}
        \frac{1}{2} p  \bar{Q}  p^T + \bar{c}  p + \bar{a} \neq 0
    \end{equation}}
\end{lemma}

\begin{proof} The best response of a memory one strategy against another memory
    one strategy can captured by a candidate set of behaviours. This candidate
    set is constructed by considering behaviours where any or all of \(p_1, p_2, p_3, p_4\)
    are \(\in \{0, 1\}\) and the rest or all of \(p_1, p_2, p_3, p_4\) are given by
    roots of the partial derivatives.

    Note that for \(p_i \in \{0, 1\}\) we consider the roots of the partial derivatives
    for \(p_j \neq p_i\) for \(i,j \in [1, 4]\).

    The derivatives, \(\frac{du}{dp}\), are given by,

    \begin{equation}\label{eq:mo_derivative}
        \frac{du_q (p)}{dp}  = \frac{(pQ + c) ( \frac{1}{2} p  \bar{Q}  p^T + \bar{c}  p + \bar{a})
        - (p\bar{Q} + \bar{c})( \frac{1}{2} p  Q  p^T + c p + a)}
          {( \frac{1}{2} p  \bar{Q}  p^T + \bar{c}  p + \bar{a})^2} \\
    \end{equation}


    For equation~\ref{eq:mo_tournament_derivative} to be zero, the numerator must fall
    to zero and the denominator can not be zero.
\end{proof}

Equations (\ref{eq:group_derivative_numerator_condition}) and
(\ref{eq:derivative_numerator_condition})
are systems of at most \(4\) polynomials and the degree of the polynomials is gradually increasing
every time an extra opponent is taken into account. Solving system of polynomials corresponds
to the calculation of a resultant and for large systems these quickly become intractable.
% TODO Add a reference here about resultants.
Because of that no further analytical consideration is given to problems described
here.

Theorems~\ref{} can also be used to identify if a strategy is a best response.
% TODO Add an example where we check if `extort-2` is a best response to the
% tournament of S&P.

Lemma~\ref{lemma:memone_group_best_response} and
Theorem~\ref{theorem:tournament_utility}
will now be used to give a list of particular best response results.


\section{Optimisation against memory one strategies}\label{section:optimisation_memone}

The results of the previous section allow for the quick characterisation of a
strategy and indeed a Nash equilibria.
Here the optimisation problem of (\ref{eq:mo_tournament_optimisation})
is maximized using Bayesian optimisation. Bayesian optimisation is a
global optimisation algorithm, introduced in~\cite{Mokus1978}, which has proven to
outperform many other popular algorithms~\cite{Jones2001}.

Bayesian optimisation constructs a probabilistic
model for \(f\) and then exploits this model to make decisions about where in the
bounded set to next evaluate the function. It relies on the prior information
and does not simply rely on local gradient and Hessian approximations.
This allows the algorithm to optimise a non concave function with relatively few
evaluations, at the cost of performing more computation to determine the next point
to try~\cite{snoek2012}.

The open source package~\cite{Head2018} offers an implementation of bayesian
optimisation and is used in this paper to compute a large number of best memory
one responses against sets of random memory one strategies. The implementation
of bayesian in~\cite{Head2018} allows us to perform the algorithm for a
different combination of parameters.  The parameters explored here are:

\begin{itemize}
    \item Number of calls, maximum number of calls to the objective function.
    \item Number of random starts, number of evaluations of the objective function
    with random points before approximating it.
\end{itemize}

The different parameters' combinations and their respective values are laid out in
Table~\ref{table:ba_opt}.

\begin{table}[htbp]
\begin{center}
\begin{tabular}{ccc}
    \toprule
    {} &  number of calls & number of random starts\\
    \midrule
    1 & 20 & 10 \\
    2 & 30 & 20 \\
    3 & 40 & 20 \\
    4 & 45 & 20 \\
    5 & 50 & 20 \\
    \bottomrule
\end{tabular}
\end{center}
\caption{Bayesian optimisation sets of parameters' values.}
\label{table:ba_opt}
\end{table}

A total of 9900 different memory one opponents were randomly generated and
Bayesian optimisation was used to find the optimal reactive strategy. The
results were compared to that of Lemma~\ref{lemma:reactive_best_response} which
is used to obtain the exact optimal.  The results of this comparison are
presented by Table~\ref{table:bayesian_excact_difference}.

\begin{table}[htbp]
    \begin{center}
        % TODO Fix the column names "Parameters"
    \input{tex/bayesian_proof.txt}
    \caption{Difference of \(u_q(p)\) for \(p \in \R_{[0, 1]} ^ 2\). The difference was
    calculated as exact \(u_q(p ^ *)\) minus Bayesian \(u_q(\tilde{p} ^ *)\).}
    \label{table:bayesian_excact_difference}
    \end{center}
\end{table}

The combination that was chosen to carry out the empirical trials is that
of 50 number of calls and 20 number of random starts. This set of parameters
was determined to be the most efficient using best reactive responses as an experimental
case.

% TODO Perhaps include a training plot as an example?

Bayesian optimisation finds the optimal
behaviour of reactive strategies. The very same set of parameters will now be
used to optimise
memory one strategies against single opponents and against
sets of \(N=2\) opponents. \(N=2\) was chosen because is the smallest \(N\) for
which there is a multi opponent interaction. For each \(N=1\) and \(N=2\) opponents a
total of 1022 best responses of memory one strategies have been captured. This data has
been archived in.
% TODO: add ZD after our paper is on pre-print?!

Note that another global optimization algorithm called differential evolution~\cite{Storn1997}
was also evaluated. Bayesian optimization was
chosen over differential evolution due to a lower computational cost and
comparable results.

\section{Limitation of memory}

The third and final part of this paper focuses on proving that short memory strategies
have limitations. Though it has been proven~\cite{Press2012} that there exists a set
of memory one strategies that can outperform any opponent, this was done only for
the case of \(N=1\). In this section we introduce several empirical results that
show that more complex strategies can indeed perform better in cases of \(N=2\).
This is achieved by comparing the performance of an optimised memory one strategy
to that of a trained long memory one.

The long memory strategies are trained using reinforcement learning through
Bayesian optimisation similarly to Section~\ref{section:optimisation_memone}.
The trained strategy used is a strategy called Gambler,
introduced and discussed in~\cite{Harper2017}, and the objective function
is the average performance in a tournament of 200 turns and 50 repetitions.
Thus, the player learns by playing a number of players, building a Bayesian
landscape of it's parameters and updating them accordingly.

\subsection{Gambler}

Several ways of representing IPD strategies have been used over the years.
In~\cite{Harper2017} several of those `archetypes' are presented
and used to train different successful strategies. One of the archetypes firstly
introduced in that paper is Gambler. Gambler is based on a lookup table and maps
the opponent's first \(n_1\) moves, the opponent's
last \(m_1\) moves, and the players last \(m_2\) moves to a probability of
cooperation.

Several variants of Gambler have been trained for this work
(Table~\ref{table:gambler}).

\begin{table}[htbp]
    \begin{center}
    \begin{tabular}{clllll}
        \toprule
        {}&  \(n_1\) & \(m_1\) & \(m_2\)\\
        \midrule
        1 & 1 & 1 & 2\\
        2 & 2 & 2 & 0\\
        3 & 2 & 2 & 1\\
        4 & 2 & 2 & 2\\
        5 & 4 & 4 & 4\\
        \bottomrule
    \end{tabular}
    \end{center}
    \caption{Variants of Gambler used.}
    \label{table:gambler}
\end{table}

\subsection{Empirical Results}

The performance of the memory one and Gamblers strategies are compared for cases of
\(N=1\) and \(N=2\). The following steps are taken:

\begin{enumerate}
    \item An \(N\) number of random opponents are generated: this gives the
        environment.
    \item Using (\ref{eq:mo_tournament_optimisation}) and Bayesian optimisation \(p^*\) 
s obtained for the environment.
    \item A Gambler type (each variant of Table~\ref{table:gambler}) is trained for the same environment.
    \item Both utilities are compared.
\end{enumerate}

A large data set containing the opponents as well as the optimised/trained behaviours
can be found in. %TODO archive
The number of experimental cases for each Gambler are displayed in Table~\ref{table:number_of_trials_per_gambler}.
Note that a number of 1022 trials corresponds to 1022 trials for \(N=1\) and 1022
trials for \(N=2\).

\begin{table}[htbp]
    \begin{center}
    \input{tex/gambler_number_of_trials.txt}
    \caption{Number of trials, for \(N=1\) and \(N=2\), for each Gambler instance.}
    \label{table:number_of_trials_per_gambler}
    \end{center}
\end{table}

The results are explored by studying the difference between
\(\frac{1}{N} \sum\limits_{i=1} ^ {N} {u_q}^{(i)} (p ^ *)\) and
\(\frac{1}{N} \sum\limits_{i=1} ^ {N} {U_q}^{(i)} (G)\) ,
where \(U(G)\) represents the utility of a Gambler. The results are shown in
Figure~\ref{fig:boxplots}.

For the cases of Gambler \(n_1=1, m_1=1, m_2=2\), \(n_1=2, m_1=2, m_2=0\) and
\(n_1=2, m_1=2, m_2=1\), though there are few edges cases, the difference distribution
is congregated around zero. For the rest of the Gambler's types there nce is mainly worse.
This could be a result of the Gamblers not being trained for long enough. Thus a larger
number of calls should be used. % Currently running more.

Furthermore, there is no significant difference between the distributions of
\(N=1\) and \(N=2\). This was checked by performing \(T-\)test for the means of two
samples. The calculated \(p-\) values are presented in Table~\ref{table:p_values}.

\begin{table}
    \begin{center}
    \begin{tabular}{llr}
        \toprule
        {} &       Gamblers &  \(p-\) values \\
        \midrule
        0 &  Gambler 1\_1\_2 &              0.242 \\
        1 &  Gambler 2\_2\_0 &              0.214 \\
        2 &  Gambler 2\_2\_1 &              0.179 \\
        3 &  Gambler 2\_2\_2 &              0.629 \\
        4 &  Gambler 4\_4\_4 &              0.141 \\
        \bottomrule
    \end{tabular}
    \caption{\(p-\) values for the means of \(N=1\) to \(N=2\) using \(T-\)tests.}
    \label{table:p_values}
    \end{center}
\end{table}

There appears to be no significant difference between complex and memory one strategies.
The difference in performance is mainly congregated around zero, and that is true for both
cases of \(N=1\) and \(N=2\). However, that there is indication that
complex strategies can outperform memory one strategies for \(N=2\).There
are cases that they have a difference in score of 0.5.

\begin{figure}
    \centering
    \begin{subfigure}{0.30\textwidth}
        \centering
        \includegraphics[width=\textwidth]{"img/Gambler 1_1_2_boxplot"}
    \end{subfigure}
    \begin{subfigure}{0.30\textwidth}
        \centering
        \includegraphics[width=\textwidth]{"img/Gambler 2_2_0_boxplot"}
    \end{subfigure}
    \begin{subfigure}{0.30\textwidth}
        \centering
        \includegraphics[width=\textwidth]{"img/Gambler 2_2_1_boxplot"}
    \end{subfigure}
    \begin{subfigure}{0.30\textwidth}
        \centering
        \includegraphics[width=\textwidth]{"img/Gambler 2_2_2_boxplot"}
    \end{subfigure}
    \begin{subfigure}{0.30\textwidth}
        \centering
        \includegraphics[width=\textwidth]{"img/Gambler 4_4_4_boxplot"}
    \end{subfigure}
    \caption{Difference between \(\frac{1}{N} \sum\limits_{i=1} ^ {N} {u_q}^{(i)} (p ^ *)\)
    and  \(\frac{1}{N} \sum\limits_{i=1} ^ {N} {U_q}^{(i)} G\) for \(N=1\) and \(N=2\).}
    \label{fig:boxplots}
\end{figure}

% TODO Let us talk about this.
\section{Discussion}

In this framework, memory one strategies for the well known game the IPD were
studied. These are strategies that utilize a single slot of memory to define their
next action. An analytical formulation for retrieving the payoffs of memory one
strategies against memory one strategies was used here. Though the analytical
formulation has been previously use, this manuscript is the first to prove that the payoff
of a such a player \(p\) has a compact form and proved that is a non concave
function. Furthermore, best memory one responses were exploit as an optimisation
problem of a ratio of quadratic forms.

We have managed to prove that for reactive and purely random strategies
that best responses can be captured analytically. This was done using using algebraic
approaches such as companion matrices and resultant theory. We investigated the
stability of defection and proved that environments for which cooperation will
never emerge can be recognised immediately by the transitions of the opponents.

Finally,  we generated a large date set of bests memory one responses for \(N=1\) and
\(N=2\). The limitations of memory were tried to be shown by comparing the performance
of best memory one strategies to that of more complex strategies. Though there are
indications that complex strategies indeed perform better, the significant of the
difference is in question. More experimental trials and exploration will be
carried out.

\appendix
%\section{Appendix Tables}\label{appendix:tables}

The memory one strategies used in the computer tournament described in~\cite{Stewart2012}
are given by Table~\ref{table:list_stewart_plotkin}.

\begin{table}[htbp]
    \begin{center}
        \resizebox{\textwidth}{!}{\begin{tabular}{clclll}
        \toprule
        {}&  Name & Memory one representation & Explanation \\
        \midrule
        1  & Cooperator           & \((1, 1, 1, 1)\) & Always chooses \(C\).\\
        2  & Defector             & \((0, 0, 0, 0)\) & Always chooses \(D\).\\
        3  & Random               & \((\frac{1}{2}, \frac{1}{2}, \frac{1}{2}, \frac{1}{2})\) & 
        Randomly chooses between \(C\) and \(D\) with a probability of 0.5.\\
        4  & Tit for Tat          & \((1, 0, 1, 0)\) & Start with a \(C\) and then mimics
        the opponent's last move.\\
        5  & Grudger              & \((1, 0, 0, 0)\) & Starts by cooperating however
        will defect if at any point the opponent has defected.\\
        6  & Generous Tit for Tat & \((1, \frac{1}{3}, 1, \frac{1}{3})\) & A more generous
        version of Tit for Tat.\\
        7  & Win Stay Lose Shift  & \((1, 0, 0, 1)\) & Starts with a \(C\) and then
        repeats it's previous move only if it was awarded with a payoff of \(R\) or \(T\).\\
        8  & ZDGTFT2              & \((1, \frac{1}{8}, 1, \frac{1}{4})\) & A generous
        zero determinant strategy introduced in~\cite{Stewart2012}\\
        9  & ZDExtort2            & \((\frac{8}{9}, \frac{1}{2}, \frac{1}{3}, 0)\) & An
        extortionate zero determinant strategy introduced in~\cite{Stewart2012}\\
        10 & Hard Joss            & \((\frac{9}{10}, 0, \frac{9}{10}, 0)\) & Cooperates
        with probability \(\frac{9}{10}\) when the opponent cooperates, otherwise emulates Tit for Tat.\\
        \bottomrule
    \end{tabular}}
    \end{center}
    \caption{The memory one strategies from~\cite{Stewart2012}.}
    \label{table:list_stewart_plotkin}
\end{table}

% Bibliography
\bibliographystyle{plain}
\bibliography{bibliography.bib}

\end{document}

