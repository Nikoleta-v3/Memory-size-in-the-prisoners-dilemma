\documentclass[10pt]{article}

% Manage page layout
\usepackage[margin=2.5cm, includefoot, footskip=30pt]{geometry}
\pagestyle{plain}
\setlength{\parindent}{0em}
\setlength{\parskip}{1em}
\renewcommand{\baselinestretch}{1}

%%%%%%%PACKAGES HERE%%%%%%%
\usepackage{tikz}
\usepackage{amsmath}
\usepackage{amssymb}
\usepackage{mathtools}
\usepackage{amsthm}
\usepackage{graphicx}
\usepackage{subcaption}
\usepackage{standalone}
\usepackage{booktabs}
\usepackage{setspace}
\usepackage[algoruled,lined]{algorithm2e}
\usepackage[noend]{algpseudocode}
\usepackage{wrapfig}
\usepackage{hyperref}
\usepackage[toc,page]{appendix}
\usetikzlibrary{calc, shapes, patterns, decorations.pathreplacing}

\makeatletter
\def\BState{\State\hskip-\ALG@thistlm}
\makeatother

\newcommand{\R}{\mathbb{R}}
\newtheorem{theorem}{Theorem}
\usetikzlibrary{decorations.pathmorphing, decorations.pathreplacing, angles,
                quotes, calc, er, positioning}

\newtheorem{lemma}[theorem]{Lemma}
\def\arraystretch{1.5}

\title{Stability of defection, optimisation of strategies and the limits of
       memory in the Prisoner's Dilemma.}
\author{Nikoleta E. Glynatsi \and Vincent A. Knight}
\date{}

\begin{document}

\maketitle

\begin{abstract}
    Memory-one strategies are a set of Iterated Prisoner's Dilemma strategies
    that have been praised for their mathematical tractability and performance
    against single opponents. This manuscript investigates \textit{best
    response} memory-one strategies as a multidimensional
    optimisation problem. Though extortionate memory-one strategies have gained
    much attention, we demonstrate that best response memory-one strategies do not
    behave in an extortionate way, and moreover, for memory one strategies to be
    evolutionary robust they need to be able to behave in a forgiving way. We
    also provide evidence that memory-one strategies suffer from their limited
    memory in multi agent interactions and can be out performed by
    longer memory strategies.
\end{abstract}

\section{Introduction}\label{section:introduction}

The Prisoner's Dilemma (PD) is a two player game used in understanding the
evolution of cooperative behaviour, formally introduced in~\cite{Flood1958}.
Each player has two options, to cooperate (C) or to defect (D). The decisions
are made simultaneously and independently. The normal form representation of the
game is given by:

\begin{equation}\label{equ:pd_definition}
    S_p =
    \begin{pmatrix}
        R & S  \\
        T & P
    \end{pmatrix}
    \quad
    S_q =
    \begin{pmatrix}
        R & T  \\
        S & P
    \end{pmatrix}
\end{equation}

where \(S_p\) represents the utilities of the row player and \(S_q\) the
utilities of the column player. The payoffs, \((R, P, S, T)\), are constrained
by equations~(\ref{eq:pd_constrain_one}) and~(\ref{eq:pd_constrain_two}).
Constraint~(\ref{eq:pd_constrain_one}) ensures that
defection dominates cooperation and constraint~(\ref{eq:pd_constrain_two})
ensures that there is a dilemma; the sum of the utilities for both players is
better when both choose to cooperate. The most common values used in the literature are
\((R, P, S, T) = (3, 1, 0, 5)\)~\cite{Axelrod1981}.


\begin{equation}\label{eq:pd_constrain_one}
    T > R > P > S
\end{equation}

\begin{equation}\label{eq:pd_constrain_two}
    2R > T + S
\end{equation}

The PD is a one shot game, however, it is commonly studied in a manner where the
history of the interactions matters. The repeated form of the game is called the
Iterated Prisoner's Dilemma (IPD) and in the 1980s, following the work
of~\cite{Axelrod1980a, Axelrod1980b} it attracted the attention of the
scientific community. In~\cite{Axelrod1980a} and~\cite{Axelrod1980b}, the first
well known computer tournaments of the IPD were performed. A total of 13 and 62
strategies were submitted respectively in the form of computer code. The
contestants competed against each other, a copy of themselves and a random
strategy, and the winner was then decided on the average score achieved (not the
total number of wins). The contestants were given access to the entire history
of a match, however, how many turns of history a strategy would incorporate,
referred to as the \textit{memory size} of a strategy, was a result of the
particular strategic decisions made by the author. The winning strategy of both
tournaments was a strategy called Tit for Tat and its success in both
tournaments came as a surprise. Tit for Tat was a simple, forgiving strategy
that opened each interaction by cooperation, and had won the tournament even
though it never scored higher than that its direct opponent. Tit for Tat provided
evidence that being nice can be advantageous and became the major paradigm for
reciprocal altruism.

Another trait of Tit for Tat is that it considers only the previous move of the
opponent. These type of strategies are called \textit{reactive} \cite{Nowak1989}
and are a subset of so called \textit{memory-one} strategies, which incorporate
both players' latests moves. Memory-one strategies have been
studied thoroughly in the literature~\cite{Nowak1990, Nowak1993}, however, they have gained
most of their attention when a certain subset of memory-one strategies was
introduced in~\cite{Press2012}, the zero-determinants. In~\cite{Stewart2012} it
was stated that ``Press and Dyson have fundamentally changed the viewpoint on
the Prisoner's Dilemma''.
Zero-determinants are a special case of memory-one and extortionate
strategies. They choose their actions so that a linear relationship is forced
between the players' score ensuring that they will always
receive at least as much as their opponents. Zero-determinants are
indeed mathematically unique and are proven to be robust in pairwise
interactions, however, their true effectiveness in tournaments and
evolutionary dynamics has been questioned~\cite{adami2013, Hilbe2013b,
Hilbe2013, Hilbe2015, Knight2018, Harper2015}.

In a similar fashion to~\cite{Press2012} the purpose of this work is to consider
a given memory-one strategy; however, whilst~\cite{Press2012} found a way for a
player to manipulate a given opponent, this work will consider a
multidimensional optimisation approach to identify the best response to a given
group of opponents. In particular, this work presents a compact method of
identifying the best response memory-one strategy against a given set of
opponents, and evaluates whether it behaves extortionately, similar to
zero-determinants. Further theoretical and empirical results of this work
include:

\begin{enumerate}
    \item The conditions that ensure a best response memory-one strategy evolutionary
    robust.
    \item A well designed framework that allows the comparison of an optimal
          memory one strategy and a more complex strategy which has a larger
          memory and was obtained through reinforcement learning
          techniques~\cite{Harper2017}.
    \item An identification of conditions for which defection is known to be
    stable; thus identifying environments where cooperation will not
    occur.
\end{enumerate}

\section{The utility}\label{section:utility}

One specific advantage of memory-one strategies is their mathematical
tractability. They can be represented completely as an element of \(\R^{4}_{[0, 1]}\). This
originates from~\cite{Nowak1989} where it is stated that if a strategy is
concerned with only the outcome of a single turn then there are four possible
`states' the strategy could be in;

\begin{itemize}
    \item both players cooperated, denoted as \(CC\)
    \item first players cooperated whilst the second player defected, denoted as \(CD\)
    \item first players defected whilst the second player cooperated, denoted as \(DC\)
    \item both players defected, denoted as \(DD\)
\end{itemize}

Therefore, a memory-one strategy can be denoted by the probability vector of
cooperating after each of these states; \(p=(p_1, p_2, p_3, p_4) \in \R_{[0,1]}
^ 4\).

In~\cite{Nowak1989} it was shown that it is not necessary to simulate the play
of a strategy $p$ against a memory-one opponent $q$. Rather this exact behaviour
can be modeled as a stochastic process, and more specifically as a Markov chain
(Figure~\ref{fig:markov_chain}) whose corresponding transition matrix \(M\) is
given by (\ref{eq:transition_matrix}). The long run steady state probability
vector \(v\), which is the solution to \(v M = v\), can be
combined with the payoff matrices of (\ref{equ:pd_definition}) to give the expected
payoffs for each player. More specifically, the utility for a memory-one
strategy \(p\) against an opponent \(q\), denoted as \(u_q(p)\), is given by
(\ref{eq:press_dyson_utility}).

\begin{figure}
    \centering
    \includestandalone[width=.35\textwidth]{tex/markov_chain}
    \caption{Markov Chain}
    \label{fig:markov_chain}
\end{figure}

\begin{equation}\label{eq:transition_matrix}
    M = [[p_1*q_1 p_1*(-q_1 + 1) q_1*(-p_1 + 1) (-p_1 + 1)*(-q_1 + 1)]
 [p_2*q_3 p_2*(-q_3 + 1) q_3*(-p_2 + 1) (-p_2 + 1)*(-q_3 + 1)]
 [p_3*q_2 p_3*(-q_2 + 1) q_2*(-p_3 + 1) (-p_3 + 1)*(-q_2 + 1)]
 [p_4*q_4 p_4*(-q_4 + 1) q_4*(-p_4 + 1) (-p_4 + 1)*(-q_4 + 1)]]
\end{equation}


\begin{equation}\label{eq:press_dyson_utility}
    u_q(p) = v \cdot (R, S, T, P).
\end{equation}

This manuscript has explored the form of \(u_q(p)\), to the authors knowledge no
previous work has done this, and it proves that \(u_q(p)\) is given by a ratio
of two quadratic forms~\cite{kepner2011},
Theorem~\ref{theorem:quadratic_form_u}.

\begin{theorem}\label{theorem:quadratic_form_u}
    The expected utility of a memory-one strategy \(p\in\mathbb{R}_{[0,1]}^4\)
    against a memory-one opponent \(q\in\mathbb{R}_{[0,1]}^4\), denoted
    as \(u_q(p)\), can be written as a ratio of two quadratic forms:

    \begin{equation}\label{eq:optimisation_quadratic}
    u_q(p) = \frac{\frac{1}{2}pQp^T + cp + a}
                {\frac{1}{2}p\bar{Q}p^T + \bar{c}p + \bar{a}},
    \end{equation}
    where \(Q, \bar{Q}\) \(\in \R^{4\times4}\) are square matrices defined by the
    transition probabilities of the opponent \(q_1, q_2, q_3, q_4\) as follows:

    \begin{center}
    \begin{equation}
    \resizebox{0.9\linewidth}{!}{\arraycolsep=2.5pt%
    \boldmath\(
    Q = \left[\begin{matrix}0 & - \left(q_{1} - q_{3}\right) \left(P q_{2} - P - T q_{4}\right) & \left(q_{1} - q_{2}\right) \left(P q_{3} - S q_{4}\right) & \left(q_{1} - q_{4}\right) \left(S q_{2} - S - T q_{3}\right)\\- \left(q_{1} - q_{3}\right) \left(P q_{2} - P - T q_{4}\right) & 0 & \left(q_{2} - q_{3}\right) \left(P q_{1} - P - R q_{4}\right) & - \left(q_{3} - q_{4}\right) \left(R q_{2} - R - T q_{1} + T\right)\\\left(q_{1} - q_{2}\right) \left(P q_{3} - S q_{4}\right) & \left(q_{2} - q_{3}\right) \left(P q_{1} - P - R q_{4}\right) & 0 & \left(q_{2} - q_{4}\right) \left(R q_{3} - S q_{1} + S\right)\\\left(q_{1} - q_{4}\right) \left(S q_{2} - S - T q_{3}\right) & - \left(q_{3} - q_{4}\right) \left(R q_{2} - R - T q_{1} + T\right) & \left(q_{2} - q_{4}\right) \left(R q_{3} - S q_{1} + S\right) & 0\end{matrix}\right]\)},
    \end{equation}
    \begin{equation}\label{eq:q_bar_matrix}
    \resizebox{0.8\linewidth}{!}{\arraycolsep=2.5pt%
    \boldmath\(
    \bar{Q} =  \left[\begin{matrix}0 & - \left(q_{1} - q_{3}\right) \left(q_{2} - q_{4} - 1\right) & \left(q_{1} - q_{2}\right) \left(q_{3} - q_{4}\right) & \left(q_{1} - q_{4}\right) \left(q_{2} - q_{3} - 1\right)\\- \left(q_{1} - q_{3}\right) \left(q_{2} - q_{4} - 1\right) & 0 & \left(q_{2} - q_{3}\right) \left(q_{1} - q_{4} - 1\right) & \left(q_{1} - q_{2}\right) \left(q_{3} - q_{4}\right)\\\left(q_{1} - q_{2}\right) \left(q_{3} - q_{4}\right) & \left(q_{2} - q_{3}\right) \left(q_{1} - q_{4} - 1\right) & 0 & - \left(q_{2} - q_{4}\right) \left(q_{1} - q_{3} - 1\right)\\\left(q_{1} - q_{4}\right) \left(q_{2} - q_{3} - 1\right) & \left(q_{1} - q_{2}\right) \left(q_{3} - q_{4}\right) & - \left(q_{2} - q_{4}\right) \left(q_{1} - q_{3} - 1\right) & 0\end{matrix}\right]\)}.
    \end{equation}
    \end{center}

    \(c \text{ and } \bar{c}\) \(\in \R^{4 \times 1}\) are similarly defined by:

    \begin{equation}\label{eq:q_matrix_numerator}
    \resizebox{0.3\linewidth}{!}{\arraycolsep=2.5pt%
    \boldmath\(c = \left[\begin{matrix}- 5 q_{1} q_{4}\\5 q_{4} \left(q_{3} - 1\right)\\q_{4} \left(2 q_{2} + 1\right)\\5 q_{1} q_{4} - 2 q_{2} q_{4} - q_{2} - 5 q_{3} q_{4} + 5 q_{3} - 3 q_{4} + 1\end{matrix}\right]\),}
    \end{equation}
    \begin{equation}\label{eq:q_matrix_denominator}
    \resizebox{0.3\linewidth}{!}{\arraycolsep=2.5pt%
    \boldmath\(\bar{c} = \left[\begin{matrix}q_{1} \left(q_{2} - q_{4} - 1\right)\\- \left(q_{3} - 1\right) \left(q_{2} - q_{4} - 1\right)\\- q_{1} q_{2} + q_{2} q_{3} + q_{2} - q_{3} + q_{4}\\q_{1} q_{4} - q_{2} - q_{3} q_{4} + q_{3} - q_{4} + 1\end{matrix}\right]\),
    }
    \end{equation}
    and the constant terms \(a, \bar{a}\) are defined as \(a = 5 q_{4}\) and
    \(\bar{a} = - q_{2} + q_{4} + 1\).
\end{theorem}

The proof of Theorem~\ref{theorem:quadratic_form_u} is given in
Appendix~\ref{appendix:proof_theorem_one}. Furthermore, numerical simulations
have been carried out to validate the result. The simulated utility, which is
denoted as \(U_q(p)\), has been calculated using~\cite{axelrodproject} an open
source research framework for the study of the IPD (\cite{axelrodproject} is
described in~\cite{Knight2016}). For smoothing the simulated results the utility
has been estimated in a tournament of 500 turns and 200 repetitions.
Figure~\ref{fig:analytical_simulated} shows two examples demonstrating that the
formulation of Theorem~\ref{theorem:quadratic_form_u} successfully captures the
simulated behaviour.

The source code used in this manuscript has been written in a sustainable manner.
It is open source (\url{https://github.com/Nikoleta-v3/Memory-size-in-the-prisoners-dilemma})
and tested which ensures the validity of the results. It has also been archived
and can be found at.
%TODO archive software

\begin{figure}[!htbp]
    \begin{center}
        \begin{subfigure}{0.45\textwidth}
            \includegraphics[width=\linewidth]{img/validation_against_player_one.pdf}
        \end{subfigure}
        \begin{subfigure}{0.45\textwidth}
            \includegraphics[width=\linewidth]{img/validation_against_player_two.pdf}
        \end{subfigure}
    \end{center}
    \caption{Simulated and empirical utilities for \(p = (0, 1, 0, 1)\)
    and \(p = (0, \frac{2}{3}, \frac{1}{3}, 0)\) against \((\frac{1}{3}, \frac{1}{3}, \frac{1}{3}, q_4)\) for
    \(q_4 \in \{0,  \frac{1}{19}, \frac{2}{19}, \dots, \frac{18}{19}, 1\}\).
    \(u_q(p)\) is the theoretic value given in Theorem~\ref{theorem:quadratic_form_u},
    and \(U_q(p)\) is simulated numerically.}
    \label{fig:analytical_simulated}
\end{figure}

Theorem~\ref{theorem:quadratic_form_u} can be extended to consider multiple
opponents. The IPD is commonly studied in tournaments and/or Moran Processes
where a strategy interacts with a number of opponents. The payoff of a player in
such interactions is given by the average payoff the player received against
each opponent. More specifically the expected utility of a memory-one strategy
against a \(N\) number of opponents is given by
Theorem~\ref{theorem:tournament_utility}.

\begin{theorem}\label{theorem:tournament_utility}
    The expected utility of a memory-one strategy \(p\in\mathbb{R}_{[0,1]}^4\)
    against a group of opponents \(q^{(1)}, q^{(2)}, \dots, q^{(N)}\), denoted
    as \(\frac{1}{N} \sum\limits_{i=1} ^ {N} {u_q}^{(i)} (p)\), is given by:

    \begin{equation}\label{eq:tournament_utility}
        \frac{1}{N} \sum\limits_{i=1} ^ {N} {u_q}^{(i)} (p) = \frac{1}{N}
        \frac{\sum\limits_{i=1} ^ {N} (\frac{1}{2} pQ^{(i)} p^T + c^{(i)} p + a^ {(i)})
        \prod\limits_{\tiny\begin{array}{l} j=1 \\ j \neq i \end{array}} ^
        N (\frac{1}{2} p\bar{Q}^{(j)} p^T + \bar{c}^{(j)} p + \bar{a}^ {(j)})}
        {\prod\limits_{i=1} ^ N (\frac{1}{2} p\bar{Q}^{(i)} p^T + \bar{c}^{(i)} p + \bar{a}^ {(i)})}.
    \end{equation}
\end{theorem}

The proof of Theorem~\ref{theorem:tournament_utility} is a straightforward algebraic
manipulation.

Similar to the previous result, the formulation of
Theorem~\ref{theorem:tournament_utility} is validated using numerical
simulations where the 10 memory-one strategies described in~\cite{Stewart2012}
have been used as the opponents. Figure~\ref{fig:stewart_plotkin_results} shows
that the simulated behaviour has been captured successfully.

\begin{figure}[!htbp]
    \begin{center}
    \includegraphics[width=.5\linewidth]{img/Stewart_tournament_results.pdf}
    \caption{The utilities of memory-one strategies \((\frac{1}{3}, \frac{1}{3}, \frac{1}{3}, p_4)\) for
    \(p_4 \in \{0,  \frac{1}{19}, \frac{2}{19}, \dots, \frac{18}{19}, 1\}\)
    against the 10 memory-one strategies described in~\cite{Stewart2012}.
    \(\frac{1}{10} \sum^{10}_{i=1} u_q^{(i)}(p)\) is the theoretic value given in
    Theorem~\ref{theorem:quadratic_form_u},
    and \(\frac{1}{10} \sum^{10}_{i=1} U_q^{(i)}(p)\) is simulated numerically.}
    \label{fig:stewart_plotkin_results}
    \end{center}
\end{figure}

The list of strategies from~\cite{Stewart2012} was also used to check whether
the utility against a group of strategies could be captured by the utility
against the mean opponent. Thus whether condition (\ref{eq:condition}) holds.
However condition~(\ref{eq:condition}) fails, as shown in
Figure~\ref{fig:hypothesis}.

\begin{equation}\label{eq:condition}
    \frac{1}{N} \sum_{i=1} ^ {N} {u_q}^{(i)} (p) = u_{\frac {1}{N} \sum\limits_{i=1} ^ N q^{(i)}}(p),
\end{equation}

\begin{figure}[!htbp]
    \begin{center}
    \includegraphics[width=.5\linewidth]{img/mean_vs_average_heatmap.pdf}
    \end{center}
    \caption{The difference between the average utility against the opponents
    from~\cite{Stewart2012} and the utility against the average player of the
    strategies in~\cite{Stewart2012} of a player \(p=(p_1, p_2, p_1, p_2)\). A
    positive difference indicates that condition (\ref{eq:condition}) does not
    hold.}
    \label{fig:hypothesis}
\end{figure}

Theorem~\ref{theorem:tournament_utility} which allows for the utility of a
memory-one strategy against any number of opponents to be estimated without
simulating the interactions is the main result used in this manuscript. In
Section~\ref{section:best_response_mem_one} it is used to  define best response
memory-one strategies and explore the conditions under which defection dominates
cooperation.

\section{Best responses to memory-one players}\label{section:best_response_mem_one}

This section focuses on best responses and more specifically \textit{memory-one
best response} strategies. A \textit{best response} is a strategy which
corresponds to the most favorable outcome~\cite{Tadelis2013}, thus a memory-one
best response to a set of opponents \(q^{(1)}, q^{(2)}, \dots, q^{(N)}\) corresponds to a strategy \(p^*\) for which
(\ref{eq:tournament_utility}) is maximised. This is considered as a multi
dimensional optimisation problem given by:

\begin{equation}\label{eq:mo_tournament_optimisation}
    \begin{aligned}
    \max_p: & \ \sum_{i=1} ^ {N} {u_q}^{(i)} (p)
    \\
    \text{such that}: & \ p \in \R_{[0, 1]}
    \end{aligned}
\end{equation}

Optimising this particular ratio of quadratic forms is not trivial. It can be
verified empirically for the case of a single opponent that there exists at least
one point for which the definition of concavity does not hold, see Appendix~\ref{appendix:non_concave}
for an example. Some results are
known for non concave ratios of quadratic forms~\cite{Beck2009, Hongyan2014},
however, in these works it is assumed that either both the numerator and the
denominator of the fractional problem are concave or that the denominator is
greater than zero which in this case are not true
(as seen in Theorem~\ref{theorem:concavity}).

\begin{theorem}\label{theorem:concavity}
    The utility of a player \(p\) against an opponent \(q\), \(u_q (p)\), given
    by (\ref{eq:optimisation_quadratic}), is not concave. Furthermore neither
    the numerator or the denominator of (\ref{eq:optimisation_quadratic}), are
    concave or strictly greater than zero.
\end{theorem}

Proof is given in Appendix~\ref{appendix:proof_theorem_three}.

The non concavity of \(u(p)\) indicates multiple local optimal points. The
approach taken here is to introduce a compact way of constructing the candidate
set of all local optimal points, and evaluating which corresponds to the best response
strategy (maximises (\ref{eq:tournament_utility})).

The problem considered is bounded because \(p \in \R^4_{[0, 1]}\).
Therefore, the candidate solutions will exist either at the boundaries of the
feasible solution space, or within that space (the methods of Lagrange
Multipliers~\cite{bertsekas2014} and Karush-Kuhn-Tucker
conditions~\cite{Giorgi2016} are based on this). This approach allow us to
define the best response memory-one strategy to a group of opponents in the
following Lemma:

\begin{lemma}\label{lemma:memone_group_best_response}

    The optimal behaviour of a memory-one strategy player
    \(p^* \in \R_{[0, 1]} ^ 4\)
    against a set of \(N\) opponents \(\{q^{(1)}, q^{(2)}, \dots, q^{(N)} \}\)
    for \(q^{(i)} \in \R_{[0, 1]} ^ 4\) is given by:

    \[p^* = \textnormal{argmax}\sum\limits_{i=1} ^ N  u_q(p), \ p \in S_q.\]

    The set \(S_q\) is defined as all the possible combinations of:

    \begin{equation}\label{eq:s_q_set}
        S_q =
        \left\{p \in \mathbb{R} ^ 4 \left|
            \begin{aligned}
                \bullet\quad p_j \in \{0, 1\} & \quad \text{and} \quad \frac{d}{dp_k} 
                \sum\limits_{i=1} ^ N  u_q^{(i)}(p) = 0
                \quad \text{for all} \quad j \in J \quad \&  \quad k \in K  \quad \text{for all} \quad J, K \\
                & \quad \text{where} \quad J \cap K = \O \quad
                \text{and} \quad J \cup K = \{1, 2, 3, 4\}.\\
                \bullet\quad  p \in \{0, 1\} ^ 4
            \end{aligned}\right.
        \right\}.
    \end{equation}
\end{lemma}

The proof is given in Appendix~\ref{appendix:proof_lemma_four}.

Note that there is no immediate way to find the zeros of \(\frac{d}{dp} \sum\limits_{i=1} ^ N  u_q(p)\);

{\small
\begin{align}\label{eq:mo_tournament_derivative}
    \frac{d}{dp} \sum\limits_{i=1} ^ {N} {u_q}^{(i)} (p) & = \nonumber \\
    & =  \displaystyle\sum\limits_{i=1} ^ {N}
    \frac{\left(pQ^{(i)} + c^{(i)}\right) \left(\frac{1}{2} p\bar{Q}^{(i)} p^T + \bar{c}^{(i)} p + \bar{a}^ {(i)}\right)
    - \left(p\bar{Q}^{(i)} + \bar{c}^{(i)}\right) \left(\frac{1}{2} pQ^{(i)} p^T + c^{(i)} p + a^ {(i)}\right)}
    {\left(\frac{1}{2} p\bar{Q}^{(i)} p^T + \bar{c}^{(i)} p + \bar{a}^ {(i)}\right)^ 2}
\end{align}
}

For \(\frac{d}{dp} \sum\limits_{i=1} ^ N  u_q(p)\) to equal zero then:

{\scriptsize
\begin{align}\label{eq:polynomials_roots}
    \displaystyle\sum\limits_{i=1} ^ {N} \left(
    \left(pQ^{(i)} + c^{(i)}\right) \left(\frac{1}{2} p\bar{Q}^{(i)} p^T + \bar{c}^{(i)} p + \bar{a}^ {(i)}\right)
    - \left(p\bar{Q}^{(i)} + \bar{c}^{(i)}\right) \left(\frac{1}{2} pQ^{(i)} p^T + c^{(i)} p + a^ {(i)}\right)\right)
    &= 0, \quad {while} \\
    \displaystyle\sum\limits_{i=1} ^ {N} \frac{1}{2} p\bar{Q}^{(i)} p^T + \bar{c}^{(i)} p + \bar{a}^ {(i)} &\neq 0.
\end{align}}

Finding best response memory-one strategies, more specifically constructing the
subset \(S_q\), can be done analytically. The points for any or all of \(p_i \in
\{0, 1\}\) for \(i \in \{1, 2, 3, 4\}\) are trivial, and finding the
roots of the partial derivatives which are a set of polynomials of equations
(\ref{eq:polynomials_roots}) is feasible using resultant
theory~\cite{Jonsson2005}; however, for large systems building the resultant quickly becomes
intractable. As a result, a numerical method taking advantage of the structure
will be used for finding best response memory-one strategies. This will be described
in Section~\ref{section:numerical_experiments}. The rest of
the section focuses on an immediate theoretical result from
Lemma~\ref{lemma:memone_group_best_response}.

\subsection{Stability of defection}\label{subsection:stability_of_defection}

An immediate result from Lemma~\ref{lemma:memone_group_best_response} can be
obtained by evaluating the sign of the derivative
(\ref{eq:mo_tournament_derivative}) at \(p=(0, 0, 0, 0)\). If at that point the
derivative is negative, then the utility of a player only decreases if they were
to change their behaviour, and thus defection at that point is stable.

\begin{lemma}\label{lemma:stability_of_defection}
    In a tournament of \(N\) players \(\{q^{(1)}, q^{(2)}, \dots, q^{(N)} \}\)
    for \(q^{(i)} \in \R_{[0, 1]} ^ 4\)
    defection is stable if the transition probabilities of the
    opponents satisfy conditions (\ref{eq:defection_condition_one}) and (\ref{eq:defection_condition_two}).

    \begin{equation}\label{eq:defection_condition_one}
        \sum_{i=1} ^ N (c^{(i)T} \bar{a}^{(i)} - \bar{c}^{(i)T} a^{(i)}) \leq 0
    \end{equation}

    while,

    \begin{equation}\label{eq:defection_condition_two}
        \sum_{i=1} ^ N \bar{a}^{(i)} \neq 0
    \end{equation}
\end{lemma}

\begin{proof}
    For defection to be stable the derivative of the utility
    at the point \(p = (0, 0, 0, 0)\) must be negative. This would indicate that
    the utility function is only declining from that point onwards.

    Substituting \(p = (0, 0, 0, 0)\) in
    equation~(\ref{eq:mo_tournament_derivative}) gives:

    \begin{equation}
    \sum_{i=1} ^ N \frac{(c^{(i)T} \bar{a}^{(i)} - \bar{c}^{(i)T} a^{(i)})}
    {(\bar{a}^{(i)})^2}
    \end{equation}

    The sign of the numerator \( \displaystyle\sum_{i=1} ^ N (c^{(i)T} \bar{a}^{(i)} - \bar{c}^{(i)T} a^{(i)})\)
    can vary based on the transition probabilities of the opponents.
    The denominator can not be negative, and otherwise is always positive.
    Thus the sign of the derivative is negative if and only if
    \( \displaystyle\sum_{i=1} ^ N (c^{(i)T} \bar{a}^{(i)} - \bar{c}^{(i)T} a^{(i)}) \leq 0\).
\end{proof}

Consider a population for which defection is known to be stable. In that
population all the members will over time adopt the same behaviour; thus in such
population cooperation will never take over. This is demonstrated in
Figures~\ref{fig:stable_defection} and~\ref{fig:unstable_defection}.

Lemma~\ref{lemma:stability_of_defection} gives a condition under which cooperation
cannot occur and is the last theoretical result
presented in this manuscript. The following section focuses on numerical
experiments.

\begin{figure}[!htb]
    \begin{subfigure}{0.49\textwidth}
        \centering
        \includegraphics[width=\linewidth]{img/population_defection_takes_over.pdf}
        \caption{For opponents \(q_{1}=(\frac{371}{1250},\frac{4693}{25000},\frac{4037}{50000},\frac{18461}{25000})\),
        $q_{2}=(\frac{48841}{100000},\frac{30587}{50000},\frac{76591}{100000},\frac{25921}{50000})$ and
        $q_{3}=(\frac{22199}{100000},\frac{87073}{100000},\frac{646}{3125},\frac{91861}{100000})$
        conditions (\ref{eq:defection_condition_one}) and
        (\ref{eq:defection_condition_two}) hold and Defector takes over the
        population.}
        \label{fig:stable_defection}
    \end{subfigure}\hfill
    \begin{subfigure}{0.49\textwidth}
        \centering
        \includegraphics[width=\linewidth]{img/population_defection_fails.pdf}
        \caption{For opponents $q_{1}=(\frac{69773}{100000},\frac{21609}{100000},\frac{97627}{100000},\frac{623}{100000})$,
        $q_{2}=(\frac{12649}{50000},\frac{43479}{100000},\frac{38969}{50000},\frac{19769}{100000})$ and
        $q_{3}=(\frac{96703}{100000},\frac{54723}{100000},\frac{24317}{25000},\frac{35741}{50000})$
        (\ref{eq:defection_condition_one}) fails and
        (\ref{eq:defection_condition_two}) holds and Defector does not take over
        the population.}
        \label{fig:unstable_defection}
    \end{subfigure}
\end{figure}

\section{Numerical experiments} \label{section:numerical_experiments}

The results of this section rely on estimating best response memory-one strategies, but as stated in
Section~\ref{section:best_response_mem_one}, estimating best responses
analytically can quickly become an intractable problem. As a result, best
responses will be estimated heuristically using Bayesian
optimisation~\cite{Mokus1978}. Bayesian optimisation is a global optimisation
algorithm that has proven to outperform many other popular
algorithms~\cite{Jones2001}. The algorithm builds a bayesian understanding of
the objective function which is well suited to the potential multiple local optimas in
the described search space of this work. Differential evolution~\cite{Storn1997}
was also considered, however, it was not selected due to Bayesian optimisation being
computationally more efficient.

As an example of the algorithm's usage let us consider the optimisation problem
of (\ref{eq:mo_tournament_optimisation}). Figure~\ref{bayesian_example}
illustrates the change of the utility function over iterations of the algorithm.
The algorithm is set to run for 60 iterations. After 60 iterations if the
utility has changed in the last 10\% iterations then algorithm runs for a
further 20 iterations. This is repeated until there is no change to the utility
in the last 10\% of iterations.


\begin{figure}[!htbp]
    \begin{center}
    \includegraphics[width=.5\linewidth]{img/bayesian_example.pdf}
    \end{center}
    \caption{Utility over time of calls using Bayesian optimisation. The
    opponents are \(q^{(1)} = (\frac{1}{3}, \frac{1}{3}, \frac{1}{3},
    \frac{1}{3})\) and \(q^{(2)} = (\frac{1}{3}, \frac{1}{3},
    \frac{1}{3}, \frac{1}{3})\). The best response obtained is \(p^* = (0, \frac{11}{50}, 0, 0)\)}
    \label{bayesian_example}
\end{figure}

The rest of the section is structured as follows. In
Section~\ref{subsection:best_response_n_2}, Bayesian optimisation is used to
generate a data set containing memory-one best responses against a number of
random opponents. The extortionate behaviour of these best responses is then
evaluated using a method introduced in~\cite{Knight2019}. In Section
\ref{subsection:best_respnse_evolutionary_setting}, a similar data set and
approach is discussed but this time the best responses are memory-one best
responses in an evolutionary setting where they also incorporate self
interactions. This has immediate applications to Moran processes.
Finally, Section~\ref{subsection:longer_memory_best_response}
compares the performances of memory-one and longer-memory best responses against
a number of opponents.

\subsection{Best response memory-one strategies for \(N=2\)}\label{subsection:best_response_n_2}

As briefly discussed in Section~\ref{section:introduction}, zero-determinants
have been praised for their robustness against a single opponent.
Zero-determinants are evidence that extortion works in pairwise interactions,
their behaviour ensures that the strategies will
never lose a game. However, this paper
argues that in multi opponent interactions, where the payoffs matter, strategies
trying to exploit their opponents will suffer.

Compared to zero-determinants, best response memory-one strategies which
have a theory of mind of their opponents, utilise their behaviour in order to
gain the most from their interactions. The question that arises then is whether
best response strategies are optimal because they behave in an extortionate
way. To estimate a strategy's extortionate
behaviour the SSE method as described in~\cite{Knight2019} is used. SSE is
defined as how far a strategy is from behaving extortionate, thus a high
SSE implies a non extortionate behaviour.

%TODO include explanation of SSE

A data set of best response memory-one strategies with \(N=2\) opponents has been
generated which is available at~\cite{glynatsi2019}. The data set contains a total of 1000 trials
corresponding to 1000 different instances of a best response strategy. For each
trial a set of 2 opponents is randomly generated and the memory-one best response
against them is found. The probabilities \(q_i\) of the opponents are
randomly generated and Figures~\ref{fig:first_opponents_probabilities} and
\ref{fig:second_opponents_probabilities}, show that they are uniformly
distributed over the trials. Thus, the full space of possible opponents has been
covered.

\begin{figure}[!htbp]
    \begin{subfigure}{0.49\textwidth}
        \centering
        \includegraphics[width=\linewidth]{img/first_opponent_probabilities.pdf}
        \subcaption{Distributions of first opponents' probabilities.}
        \label{fig:first_opponents_probabilities}
    \end{subfigure}
    \begin{subfigure}{0.49\textwidth}
        \centering
        \includegraphics[width=\linewidth]{img/second_opponent_probabilities.pdf}
        \subcaption{Distributions of second opponents' probabilities.}
        \label{fig:second_opponents_probabilities}
    \end{subfigure}
\end{figure}

The SSE method has been applied to the data set. The distribution of SSE for the best response is given in
Figure~\ref{fig:sserror_mem_one} and a statistics summary in
Table~\ref{table:sserror_stats}. The distribution of SSE is skewed to the left,
indicating that the best response does exhibit extortionate behaviour, however,
the best response is not uniformly extortionate. A positive measure of skewness
and kurtosis indicates a heavy tail to the right. Therefore, in several cases the
strategy is not trying to extort its the opponents.

So although the best response strategy can exhibit extortionate behaviour, its
performance is maximised by behaving in a more adaptable way than zero-determinant
strategies. This is confirms similar results such as~\cite{Knight2019}.
This analysis will now be extended to an evolutionary setting.

\begin{figure}[!htbp]
    \begin{minipage}{0.72\textwidth}
            \begin{center}
                \includegraphics[width=\linewidth]{img/best_respones_sserror.pdf}
            \end{center}
                \caption{Distribution of SSE for memory-one best responses, when \(N=2\).}
                \label{fig:sserror_mem_one}
    \end{minipage}\hspace{1cm}
    \begin{minipage}{0.21\textwidth}
        \centering
        \captionsetup{type=table}
        \resizebox{.85\columnwidth}{!}{%
            \begin{tabular}{lr}
\toprule
{} &     SSE \\
\midrule
count  &  1000.00000 \\
mean   &     0.33762 \\
std    &     0.39667 \\
min    &     0.00000 \\
5\%     &     0.02078 \\
25\%    &     0.07597 \\
50\%    &     0.17407 \\
95\%    &     1.05943 \\
max    &     2.47059 \\
median &     0.17407 \\
skew   &     1.87231 \\
kurt   &     3.60029 \\
\bottomrule
\end{tabular}
}
            \caption{Summary statistics SSE of best response memory one strategies included
            tournaments of \(N=2\).}
            \label{table:sserror_stats}
      \end{minipage}
\end{figure}

\subsection{Memory-one best responses in evolutionary dynamics}\label{subsection:best_respnse_evolutionary_setting}

As mentioned in Section~\ref{section:utility}, the IPD is commonly studied in
Moran processes, and generally, in evolutionary processes. In these settings self
interactions are key. This section extends the formulation of best responses
in evolutionary dynamics, more specifically, the optimisation problem of
(\ref{eq:mo_tournament_optimisation}) is extended to
include self interactions.

Self interactions can be incorporated in the formulation
that has been used so far. The utility is given by,

\begin{equation}
    \frac{1}{N} \sum\limits_{i=1} ^ {N} {u_q}^{(i)} (p) + u_p(p)
\end{equation}

and the optimisation problem of (\ref{eq:mo_tournament_optimisation}) is modified to give:

\begin{equation}\label{eq:mo_evolutionary_optimisation}
    \begin{aligned}
    \max_p: & \ \frac{1}{N} \sum\limits_{i=1} ^ {N} {u_q}^{(i)} (p) + u_p(p)
    \\
    \text{such that}: & \ p \in \R_{[0, 1]}
    \end{aligned}
\end{equation}

% Note that exact formulate are known for given evolutionary processes, however,
% for simplicity 17 is given to optimisation.
For determining the memory-one best response in an evolutionary setting,
an algorithmic approach is considered, called \textit{best
response dynamics}. Best response dynamics are commonly used in evolutionary
game theory. They represent a class of strategy updating rules, where players in
the next round are determined by their best responses to some subset of the
population. The best response dynamics approach used in this manuscript is given by
Algorithm~\ref{algo:best_response_dynamics}.

\begin{minipage}{.6\textwidth}
    \begin{algorithm}[H]
        $p^{(t)}\leftarrow (1, 1, 1, 1)$\;
        \While{$p^{(t)} \neq p ^{(t -1)}$}{
         $p^{(t + 1)} =  \text{argmax} \frac{1}{N} \sum\limits_{i=1} ^ {N} {u_q}^{(i)}
         (p^{(t + 1)}) + u_p^{(t)}(p^{(t + 1)})$\;
        }
        \caption{Best response dynamics Algorithm}
        \label{algo:best_response_dynamics}
    \end{algorithm}
\end{minipage}

The best response dynamics algorithm starts by setting an initial
solution \(p^{(1)}=(1, 1, 1, 1)\), and repeatedly finds a strategy that maximises
(\ref{eq:mo_evolutionary_optimisation}) using Bayesian optimisation. The
algorithm stops once a cycle (a sequence of iterated evaluated points) is
detected. A numerical example of the algorithm is given in Figure~\ref{fig:best_response_dynamics_results}.

\begin{figure}[!htbp]
    \centering
    \includegraphics[width=.6\textwidth]{img/evolution_example_two.pdf}
    \caption{Best response dynamics with \(N=2\). More specifically, for
    \(q ^{(1)}=(\frac{59}{250},
                \frac{1031}{10000},
                \frac{99}{250},
                \frac{1549}{10000})\) and
    \(q ^{(2)}=(\frac{133}{2000},
                \frac{803}{2000},
                \frac{9179}{10000},
                \frac{2001}{2500})\).}
\label{fig:best_response_dynamics_results}
\end{figure}

The algorithm has been used to estimate the best response in an evolutionary
setting for each of the 1000 pairs of opponents described in
Section~\ref{subsection:best_response_n_2}. These are also included in the data
set~\cite{glynatsi2019}, and moreover, the SSE method has also been applied. The
distribution of SSE is given by Figure~\ref{fig:sserror_mem_one} and a
statistical summary by Table~\ref{table:sserror_stats}.

Similarly to the results of Section~\ref{subsection:best_response_n_2}, the
evolutionary best response strategy does not behave uniformly extortionately. A
larger value of both the kurtosis and the skewness of the SSE distribution
indicates that in evolutionary settings a memory-one best response is even more
adaptable.

The difference between best responses in tournaments and in evolutionary
settings are further explored by Figure~\ref{fig:behaviour_violin_plots}.
Though, Table~\ref{table:wilcoxon_tests} details that no statistically
significant differences has been found, from
Figure~\ref{fig:behaviour_violin_plots}, it seems that evolutionary best
response has a higher $p_2$ median. Thus, they more likely to forgive after
being tricked.

\begin{figure}[!htbp]
    \begin{minipage}{0.72\textwidth}
            \begin{center}
            \includegraphics[width=\linewidth]{img/evo_sserror.pdf}
            \end{center}
            \caption{Distribution of SSE of best response memory-one strategies in
            evolutionary settings, when \(N=2\).}
            \label{fig:sserror_mem_one}
    \end{minipage}\hspace{1cm}
    \begin{minipage}{0.21\textwidth}
        \centering
        \captionsetup{type=table}
        \resizebox{.85\columnwidth}{!}{%
            \begin{tabular}{lr}
\toprule
{} &  SSE \\
\midrule
count  &    1000.00000 \\
mean   &       0.17326 \\
std    &       0.23489 \\
min    &       0.00001 \\
5\%     &       0.01497 \\
25\%    &       0.05882 \\
50\%    &       0.12253 \\
95\%    &       0.67429 \\
max    &       1.52941 \\
median &       0.12253 \\
skew   &       3.41839 \\
kurt   &      11.92339 \\
\bottomrule
\end{tabular}
}
            \caption{Summary statistics SSE of best response memory-one strategies in
            evolutionary settings, when when \(N=2\).}
            \label{table:sserror_stats}
      \end{minipage}
\end{figure}

\begin{figure}[!htbp]
    \centering
    \includegraphics[width=.8\textwidth]{img/behaviour_violin_plots.pdf}
    \caption{Distributions of \(p^*\) for both best response and evo memory-one
    strategies.}
    \label{fig:behaviour_violin_plots}
\end{figure}

\begin{table}[!htbp]
    \centering
    \resizebox{.7\columnwidth}{!}{%
    \begin{tabular}{llrrr}
\toprule
{} & Best Response Median in: &  Tournament &  Evolutionary Settings &  p-values \\
\midrule
&       Distribution $p_1$ &         0.0 &                0.00000 &       0.0 \\
&       Distribution $p_2$ &         0.0 &                0.19847 &       0.0 \\
&       Distribution $p_3$ &         0.0 &                0.00000 &       0.0 \\
&       Distribution $p_4$ &         0.0 &                0.00000 &       0.0 \\
\bottomrule
\end{tabular}
}
    \caption{A non parametric test, Wilcoxon Rank Sum, has been performed to
    tests the difference in the median values of the cooperation probabilities
    in tournaments versus evolutionary settings. A non parametric test is used because
    is evident that the data are skewed.}\label{table:wilcoxon_tests}
\end{table}

\subsection{Longer memory best response}\label{subsection:longer_memory_best_response}

This section focuses on the memory size of strategies. The effectiveness of
memory in the IPD has been previously explored in the literature, as
discussed in Section~\ref{section:introduction}, however, none of the
previous works has compared the performance of longer-memory strategies to
memory-one best responses.

In~\cite{Harper2017}, a strategy called \textit{Gambler} which makes
probabilistic decisions based on the opponent's \(n_1\) first moves, the
opponent's \(m_1\) last moves and the player's \(m_2\) last moves was
introduced. In this manuscript Gambler with parameters: $n_1 = 2, m_1 = 1$ and $m_2 = 1$ is used
as a longer-memory strategy.

By considering the opponent's first two moves, the opponents last move and the
player's last move, there are only 16 $(4 \times 2 \times 2)$ possible outcomes
that can occur, furthermore, Gambler also makes a probabilistic decision of
cooperating in the opening move. Thus, Gambler is a function \(f: \{\text{C,
D}\} \rightarrow [0, 1]_{\R}\). This can be hard coded as an element
of \([0, 1]_{\R} ^ {16 + 1}\), one probability for each outcome plus the opening
move. Hence, compared to (\ref{eq:mo_tournament_optimisation}), finding an
optimal Gambler is a 17 dimensional problem given by:

\begin{equation}\label{eq:gambler_optimisation}
    \begin{aligned}
    \max_p: & \ \sum_{i=1} ^ {N} {U_q}^{(i)} (f)
    \\
    \text{such that}: & \ f \in \R_{[0, 1]}^{17}
    \end{aligned}
\end{equation}

Note that (\ref{eq:tournament_utility}) can not be used here for the utility
of Gambler, and actual simulated players are used. This is done using~\cite{axelrodproject}
with 500 turns and 200 repetitions, moreover, (\ref{eq:gambler_optimisation})
is solved numerically using Bayesian optimisation.

Similarly to previous sections, a large data set has been generated with
instances of an optimal Gambler and a memory-one best response, available
at~\cite{glynatsi2019}. Estimating a best response Gambler (17 dimensions) is
computational more expensive compared to a best response memory-one (4
dimensions). As a result, the analysis of this section is based on a total of
130 trials. For each trial two random opponents have been selected. The 130 pair
of opponents are a sub set of the opponents used in
Section~\ref{subsection:best_response_n_2}-
\ref{subsection:best_respnse_evolutionary_setting}. The distributions of their
transition probabilities are given in Figures
\ref{fig:first_opponents_probabilities_with_gambler} and
\ref{fig:first_opponents_probabilities_with_gambler}.

\begin{figure}[!htbp]
    \begin{subfigure}{0.49\textwidth}
        \centering
        \includegraphics[width=\linewidth]{img/first_opponent_probabilities_with_gambler.pdf}
        \subcaption{Distributions of first opponents' probabilities for longer memory experiment.}
        \label{fig:first_opponents_probabilities_with_gambler}
    \end{subfigure}
    \begin{subfigure}{0.49\textwidth}
        \centering
        \includegraphics[width=\linewidth]{img/second_opponent_probabilities_with_gambler.pdf}
        \subcaption{Distributions of second opponents' probabilities for longer memory experiment.}
        \label{fig:second_opponents_probabilities_with_gambler}
    \end{subfigure}
\end{figure}

The utilities of both strategies are plotted against each other in
Figure~\ref{fig:utilities_gambler_mem_one}. Although Gambler has an infinite
memory (in order to remember the opening moves of the opponent) the information
the strategy considers is not significantly larger than memory-one strategies.
Even so, it is evident from Figure~\ref{fig:utilities_gambler_mem_one} that
Gambler always performs as well as the best response memory-one or better. This seems to be at odd with the
result of~\cite{Press2012} that against a memory-one opponent having a longer memory
will not give a strategy any
advantage. However, against two memory-one opponents Gambler's performance is better than
the optimal memory-one strategy. This is evidence that in the case of two opponents having a
shorter memory is limiting.

\begin{figure}[!htbp]
    \centering
    \includegraphics[width=.55\textwidth]{img/gambler_performance_against_mem_one.pdf}
    \caption{Utilities of Gambler and best response memory-one strategies for
    130 different pair of opponents.}\label{fig:utilities_gambler_mem_one}
\end{figure}

\section{Conclusion}

This manuscript has considered \textit{best response} strategies in the IPD game, and
more specifically, \textit{memory-one best responses}. It has proven that there is
a compact way of identifying a memory-one best response to a group of opponents,
and moreover, that there exists a condition for which in an
environment of memory-one opponents defection is the stable choice.
The later parts of this paper focused on a series of empirical results, where it
was shown that the performance and the evolutionary stability of memory-one
strategies rely not on extortion but on adaptability. Finally, it was shown that
memory-one strategies' performance is limited by their memory in cases where
they interact with multiple opponents.

Following the work described in~\cite{Nowak1989}, where it was shown that the
utility between two memory-one strategies can be estimated by a Markov
stationary state, we proved that the utilities can be written as a ration of two
quadratic forms in $R^4$, Theorem~\ref{theorem:quadratic_form_u}. This was
extended to include multiple opponents, as the IPD is commonly studied in such
situations, Theorem~\ref{theorem:tournament_utility}.
The formulation of Theorem~\ref{theorem:tournament_utility} allowed us to introduce an approach for identifying
memory-one best responses to any number of opponents;
Lemma~\ref{lemma:memone_group_best_response}. This does not only have game
theoretic novelty, but also a mathematical novelty of solving quadratic ratio
optimisation problem where the quadratics are non concave. The results of
Lemma~\ref{lemma:memone_group_best_response} were also used to define a
condition for which defection is known to be stable.

This manuscript presented several experimental results. These results were mainly to
investigate the behaviour of memory-one strategies and their limitations. In
Sections~\ref{subsection:best_response_n_2}
and~\ref{subsection:best_respnse_evolutionary_setting}, a large data set which
contained best responses in tournaments and in evolutionary settings for $N=2$
was generated. This allowed us to investigate their respective behaviours, and
whether it was extortionate acts that made them the most favorable strategies.
However, it was shown that it was not extortion but adaptability that allowed
the strategies to gain the most from their interactions.
In evolutionary settings it was specifically shown that being adaptable and being
able to forgive after being tricked were key factors. In Section~\ref{subsection:longer_memory_best_response}, the performance of
memory-one strategies was put against the performance of a longer memory
strategy called Gambler. There were several cases where Gambler would outperform
the memory-one strategy, however, a memory-one strategy never managed to outperform
a Gambler. This result occurred whilst considering a Gambler with a sufficiently
larger memory but not a sufficiently larger amount of information regarding
the game.

All the empirical results presented in this manuscript have been for the
case of $N=2$. In future work we would consider larger values of $N$, however, we
believe that for larger values of $N$ the results that have been presented here would
only be more evident.

\section{Acknowledgements}

A variety of software libraries have been used in this work:

\begin{itemize}
    \item The Axelrod library for IPD simulations~\cite{axelrodproject}.
    \item The Scikit-optimize library for an implementation of Bayesian optimisation~\cite{tim_head_2018_1207017}.
    \item The Matplotlib library for visualisation~\cite{hunter2007matplotlib}.
    \item The SymPy library for symbolic mathematics~\cite{sympy}.
    \item The Numpy library for data manipulation~\cite{walt2011numpy}.
\end{itemize}

% Bibliography
\bibliographystyle{plain}
\bibliography{bibliography.bib}

\begin{appendices}

\section{Proofs of the Theorems}\label{section:appendix_a}

\subsection{Proof of Theorem~\ref{theorem:quadratic_form_u}}\label{appendix:proof_theorem_one}
\begin{proof} Utility \(u_q(p)\) can be written as a ratio of two quadratic forms.

The utility of a memory-one player \(p\) against an opponent \(q\) its the product of the steady
states and the PD payoffs,

\[u_q(p) = v \cdot (R, S, T, P)\]

where

\[v = \left[\frac{v_1}{\bar{v_1}}, \frac{v_2}{\bar{v_2}}, \frac{v_3}{\bar{v_3}}, \frac{v_4}{\bar{v_4}}\right],\]

where

\scalebox{0.9}{\parbox{\linewidth}{%
\begin{align*}
    v_1 & = p_{2} p_{3} \left(q_{2} q_{4} - q_{3} q_{4}\right) + p_{2} p_{4} \left(q_{2} q_{3} - q_{2} q_{4} - q_{3} + q_{4}\right) + p_{3} p_{4} \left(- q_{2} q_{3} + q_{3} q_{4}\right) - p_{3} q_{2} q_{4} + p_{4} q_{2} q_{4} - p_{4} q_{4}\\
    \bar{v_1} & = p_{1} p_{2} \left(q_{1} q_{2} - q_{1} q_{4} - q_{1} - q_{2} q_{3} + q_{3} q_{4} + q_{3}\right) +
    p_{1} p_{3} \left(- q_{1} q_{3} + q_{1} q_{4} + q_{2} q_{3} - q_{2} q_{4}\right)
    + p_{1} p_{4} \left(- q_{1} q_{2} + q_{1} q_{3} + q_{1} + q_{2} q_{4} - q_{3} q_{4} - q_{4}\right)
    \\
    & \phantom{ {}= } - p_{1} \left(q_{1} q_{2} + q_{1} q_{4} + q_{1} \right) +
     p_{2} p_{3} \left(- q_{1} q_{2} + q_{1} q_{3} + q_{2} q_{4} + q_{2} - q_{3} q_{4} - q_{3}\right)
    + p_{2} p_{4} \left(- q_{1} q_{3} + q_{1} q_{4} + q_{2} q_{3} - q_{2} q_{4}\right) \\
    &\phantom{ {}= } + p_{2} \left(q_{2} q_{3} - q_{2} - q_{3} q_{4} - q_{3} + q_{4} + 1\right) + p_{3} p_{4} \left(q_{1} q_{2} - q_{1} q_{4} - q_{2} q_{3} - q_{2} + q_{3} q_{4} + q_{4}\right) \\
    & \phantom{ {}= } + p_{3} \left(q_{1} q_{2} - q_{2} q_{3} - q_{2} + q_{3} - q_{4}\right) 
    - p_{4} \left( q_{1} q_{4} + q_{2} + q_{3} q_{4} - q_{3} + q_{4} - 1\right) + q_{2} - q_{4} - 1
\end{align*}
}}

\end{proof}

%- p_{1}\left(q_{1} q_{2} + q_{1} q_{4} + q_{1}\right) +

\subsection{Proof of Theorem~\ref{theorem:concavity}}\label{appendix:proof_theorem_three}
The utility \(u_q(p)\) is non concave and neither are it's numerator or
denominator. Furthermore, the denominator is not always strictly positive.

\begin{proof}

The utility \(u_q(p)\) is non concave because the concavity condition fails for at
least one pair of points see Appendix~\ref{appendix:non_concave}.

Furthermore, regarding the numerator and denominator of \(u_q(p)\)
in~\cite{Anton2014} it is stated that a quadratic form will be concave if and
only if it's symmetric matrix is semi-negative definite. A matrix \(A\) is
semi-negative definite if:

\begin{equation}\label{def:semi_negative}
|A|_i \leq 0 \text{ for } i \text{ is odd and } |A|_i \geq 0  \text{ for } i
\text{ is even,}
\end{equation}

where \(|A|_i\) is the eigenvalues of the submatrix \(A_i\).

For (\ref{eq:optimisation_quadratic}), neither \(\frac{1}{2}pQp^T + cp + a\)
or \(\frac{1}{2}p\bar{Q}p^T + \bar{c}p + \bar{a}\) are concave because for an even \(i=2\):

\[|Q|_2 = - \left(q_{1} - q_{3}\right)^{2} \left(q_{2} - 5 q_{4} - 1\right)^{2} \text{and}\]
\[|\bar{Q}|_2 =- \left(q_{1} - q_{3}\right)^{2} \left(q_{2} - q_{4} - 1\right)^{2}\]

are negative.

Moreover, for a quadratic to be strictly positive it has to be positive definite.
A quadratic form is positive definite iff every eigenvalue of is positive,
however, \(\frac{1}{2}p\bar{Q}p^T + \bar{c}p + \bar{a}\) is not positive definite
because:

\[|\bar{Q}|_2 =- \left(q_{1} - q_{3}\right)^{2} \left(q_{2} - q_{4} - 1\right)^{2}\]

is negative.
\end{proof}

\subsection{Proof of Lemma~\ref{lemma:memone_group_best_response}}\label{appendix:proof_lemma_four}
\begin{proof}
The optimal behaviour of a memory-one strategy player
\(p^* \in \R_{[0, 1]} ^ 4\)
against a set of \(N\) opponents \(\{q^{(1)}, q^{(2)}, \dots, q^{(N)} \}\)
for \(q^{(i)} \in \R_{[0, 1]} ^ 4\) is established by:

\[p^* = \textnormal{argmax}\left(\sum\limits_{i=1} ^ N  u_q(p)\right), \ p \in S_q,\]

where \(S_q\) is given by (\ref{eq:s_q_set}).

The optimisation problem of (\ref{eq:mo_tournament_optimisation}) can be
written as:

\begin{equation}\label{eq:mo_tournament_optimisation_standard}
    \begin{aligned}
    \max_p: & \ \sum_{i=1} ^ {N} {u_q}^{(i)} (p)
    \\
    \text{such that}: p_i & \leq 1 \text{ for } \in \{1, 2, 3, 4\} \\
    - p_i & \leq 0 \text{ for } \in \{1, 2, 3, 4\} \\
    \end{aligned}
\end{equation}

The optimisation problem has two inequality constraints and regarding the optimality
this means that:

\begin{itemize}
    \item either the optimum is away from the boundary of the optimization domain, and so the constraints plays no role;
    \item or the optimum is on the constraint boundary.
\end{itemize}

Thus, the following three cases must be considered:

\textbf{Case 1:} The solution is on the boundary and any of the possible
combinations for $p_i \in \{0, 1\}$ for $i \in \{1, 2, 3, 4\}$ are candidate
optimal solutions.

\textbf{Case 2:} The optimum is away from the boundary of the optimization domain
and the interior solution $p^*$ necessarily satisfies the condition
\(\frac{d}{dp} \sum\limits_{i=1} ^ N  u_q(p^*) = 0\).

\textbf{Case 3:} The optimum is away from the boundary of the optimization domain
but some constraints are equalities. The candidate solutions in this case
are any combinations of $p_j \in \{0, 1\} \quad \text{and} \quad \frac{d}{dp_k} 
\sum\limits_{i=1} ^ N  u_q^{(i)}(p) = 0$ 
forall $ j \in J \text{ \& } k \in K \text{ forall } J, K
\text{ where } J \cap K = \O \text{ and } J \cup K = \{1, 2, 3, 4\}.$

Combining cases 1-3 a set of candidate solution is constructed as:

\begin{equation*}
    S_q =
    \left\{p \in \mathbb{R} ^ 4 \left|
        \begin{aligned}
            \bullet\quad p_j \in \{0, 1\} & \quad \text{and} \quad \frac{d}{dp_k} 
            \sum\limits_{i=1} ^ N  u_q^{(i)}(p) = 0
            \quad \text{for all} \quad j \in J \quad \&  \quad k \in K  \quad \text{for all} \quad J, K \\
            & \quad \text{where} \quad J \cap K = \O \quad
            \text{and} \quad J \cup K = \{1, 2, 3, 4\}.\\
            \bullet\quad  p \in \{0, 1\} ^ 4
        \end{aligned}\right.
    \right\}.
\end{equation*}

This set is denoted as $S_q$ and the optimal solution to
(\ref{eq:mo_tournament_optimisation}) is the point from $S_q$ for which the
utility is maximised.

% The Lagrangian\cite{bertsekas2014} of (\ref{eq:mo_tournament_optimisation_standard})
% is then given by,

% \begin{align}
% L(p_1, p_2, p_3, p_4, \lambda_1, \lambda_2, \dots, \lambda_8) = \sum\limits_{i=1} ^ N  u_q(p)
% + \lambda_1 (p_1 - 1) + \lambda_2 (p_2 - 1) + \lambda_3 (p_3 - 1) + \lambda_4 (p_4 - 1) + \\
% \lambda_5( - p_1) + \lambda_6 (- p_2) + \lambda_7 (- p_3) + \lambda_8 (- p_4)
% \end{align}

% This gives the following Karush-Kuhn-Tucker~\cite{Giorgi2016} conditions:

% \begin{align}
% \frac{d\sum\limits_{i=1} ^ N  u_q(p)}{dp_1} + \lambda_1 -\lambda_5 = 0 \\
% \frac{d\sum\limits_{i=1} ^ N  u_q(p)}{dp_2} + \lambda_2 -\lambda_6 = 0 \\
% \frac{d\sum\limits_{i=1} ^ N  u_q(p)}{dp_3} + \lambda_3 -\lambda_7 = 0 \\
% \frac{d\sum\limits_{i=1} ^ N  u_q(p)}{dp_4} + \lambda_4 -\lambda_8 = 0 \\
% \lambda_i (p_i - 1) = 0 \text{ for } i \in \{1, 2, 3, 4\} \\
% -\lambda_5 p_1 = 0 \\
% -\lambda_6 p_2 = 0 \\
% -\lambda_7 p_3 = 0 \\
% -\lambda_8 p_4 = 0
%     \end{align}

% There are eight complementarity conditions (37) - (41), thus a total of 16 cases
% to be checked.

% \textbf{Case 1:} \(\lambda_1 = \lambda_2 = \dots = \lambda_8 = 0\). The best response
% is given by the roots of the partial derivatives
% \(\frac{d\sum\limits_{i=1} ^ N  u_q(p)}{dp} = 0\).

% \textbf{Case 2:} \(\lambda_1 = \lambda_2 = \lambda_3 = \lambda_4 = 0\) and
% \(\lambda_5 \neq 0, \lambda_6 \neq 0,  \lambda_7 \neq 0, \lambda_8 \neq 0\). The
% best response is given by \(p_1 = p_2 = p_3 = p_4 = 0\).

% \textbf{Case 3:} \(\lambda_5 = \lambda_6 = \lambda_7 = \lambda_8 = 0\) and
% \(\lambda_1 \neq 0, \lambda_2 \neq 3,  \lambda_4 \neq 0, \lambda_5 \neq 0\). The
% best response is given by \(p_1 = p_2 = p_3 = p_4 = 1\).


\end{proof}

\section{Further Examples}\label{section:appendix_b}

\subsection{Example of non concavity for \(u(p)\) }\label{appendix:non_concave}
A function \(f(x)\) is concave on an interval \([a, b]\) if, for any two
points \(x_1, x_2 \in [a, b]\) and any \(\lambda \in [0, 1]\), 

\begin{equation}\label{eq:concave}
f (\lambda x_1 + (1 - \lambda )x_2 ) \geq \lambda f (x_1 ) + (1 - \lambda )f (x_2 ).
\end{equation}

Let \(f\) be \(u_{(\frac{1}{3}, \frac{1}{3}, \frac{1}{3}, \frac{1}{3})}\).
For \(x_1 = (\frac{1}{4}, \frac{1}{2}, \frac{1}{5} , \frac{1}{2}),
x_2 = (\frac{8}{10}, \frac{1}{2}, \frac{9}{10} , \frac{7}{10})\) and
\(\lambda=0.1\), direct substitution in (\ref{eq:concave}) gives:

\scalebox{0.85}{\parbox{\linewidth}{%
\begin{align*}
    u_{(\frac{1}{3}, \frac{1}{3}, \frac{1}{3}, \frac{1}{3})}
    \left( 0.1 \left(\frac{1}{4}, \frac{1}{2}, \frac{1}{5} , \frac{1}{2}\right)
    + 0.9 \left(\frac{8}{10}, \frac{1}{2}, \frac{9}{10} , \frac{7}{10}\right) \right) & \geq
    0.1 \times u_{(\frac{1}{3}, \frac{1}{3}, \frac{1}{3}, \frac{1}{3})}
    \left(\left(\frac{1}{4}, \frac{1}{2}, \frac{1}{5} , \frac{1}{2}\right) \right) 
    + 0.9 \times u_{(\frac{1}{3}, \frac{1}{3}, \frac{1}{3}, \frac{1}{3})}
    \left(\left(\frac{8}{10}, \frac{1}{2}, \frac{9}{10} , \frac{7}{10}\right) \right) \Rightarrow\\
    1.485 & \geq 0.1 \times 1.790 + 0.9 \times 1.457 \Rightarrow \\
    1.485 & \geq 1.490
\end{align*}
}}

which can not hold. Thus \(u_{(\frac{1}{3}, \frac{1}{3}, \frac{1}{3}, \frac{1}{3})}\)
is not concave.

\end{appendices}

\end{document}

