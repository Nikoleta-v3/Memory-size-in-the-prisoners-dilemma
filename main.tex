\documentclass[10pt]{article}
% Package to manage page layout
\usepackage[margin=2.5cm, includefoot, footskip=30pt]{geometry}

\title{Memory size in the Prisoner's Dilemma}
\author{Nikoleta E. Glynatsi \and Vincent Knight}
\date{}

\begin{document}

\maketitle

\begin{abstract}

The two player Iterated Prisoner's Dilemma is a fundamental iterated game used
for studying the emergence of cooperation. The two players interact repeatedly
and they have the ability to adopt strategies. A strategy allows a player to map
the outcomes of the previous interactions to an action. A set of strategies that
consider only the outcome of the previous round are called memory one. These
players gain attention after a publication in 2012 that showed that a memory one
strategy can manipulate its opponent.

In this manuscript we build upon a framework provided in 1989 for the study of
these strategies and identify the best responses of memory one players. The aim
of this work is to show the limitations of memory one strategies in multi-opponent
interactions. A number of theoretic results are presented.
%TODO: Expand when we get results
\end{abstract}

\section{Introduction}

\subsection{Background}

\section{Utility}

\subsection{Validation}

\section{Proximate of best responses}
\subsection{In purely random strategies}
\subsection{In memory one strategies}

\section{Stability of defection}

\section{Numerical experiments}

% Bibliography
\bibliographystyle{plain}
\bibliography{bibliography.bib}

\end{document}