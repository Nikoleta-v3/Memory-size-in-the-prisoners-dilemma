\documentclass{article}
\usepackage[margin=2cm, includefoot, footskip=30pt]{geometry}
\setlength\parindent{0pt}
\setlength{\parskip}{1em}
\usepackage{authblk}

\title{Cover Letter: Stability of defection, optimisation of strategies and the
limits of memory in the Prisoner's Dilemma..}

\author[1]{Nikoleta E. Glynatsi}
\author[1]{Vincent A. Knight}

\affil[1]{Cardiff University, School of Mathematics, Cardiff, United Kingdom}
\date{}
\setcounter{Maxaffil}{0}
\renewcommand\Affilfont{\itshape\small}

\begin{document}

\maketitle

To the editors,

This paper presents a framework for exploring the space of memory-one strategies
in the Iterated Prisoner's Dilemma (IPD), and identifying the best response
strategy to a variety of situations. It is demonstrated how best response
memory-one strategies can be found explicitly by considering a multidimensional
optimisation of a ratio of two quadratic functions.

This paper offers a strong contribution on three fronts:

\begin{itemize}
    \item It presents a compact method of identifying the best memory-one
    strategy against a given set of opponents in tournaments and evolutionary
    settings.
    \item It presents a well designed framework that allows the comparison of an
          optimal memory-one strategy, and a more complex strategy that has a
          larger memory.
    \item It presents an identification of conditions for which defection is
          known to be a best response; thus identifying environments where
          cooperation can not occur.
\end{itemize}

The results of this work contribute to the discussion regarding the
effectiveness of extortionate strategies in the IPD, it highlights the importance
of adaptability in multi player interactions and the limitations of short memory.
The results are novel not only from the point of view of game theory as
a field, but also the mathematical novelty of solving quadratic ratio
optimisation problems where the quadratics are non-concave.

This work has been carried out with the highest standard of reproducibility: the
source code for generating data and for the analysis are not only well described
but they are also all open source, archived and made available online.

Thank you for taking the time to consider our work,

The Authors.
\end{document}