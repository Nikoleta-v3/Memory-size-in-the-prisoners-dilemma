\documentclass[9pt,twoside,lineno]{pnas-new}
% Use the lineno option to display guide line numbers if required.

\templatetype{pnassupportinginfo}

\usepackage{amsthm}
\newcommand{\R}{\mathbb{R}}
\newtheorem{theorem}{Theorem}
\newtheorem{lemma}[theorem]{Lemma}

\title{Stability of defection, optimisation of strategies and the limits of
memory in the Prisoner's Dilemma.}
\author{Nikoleta E. Glynatsi and Vince A. Knight}
\correspondingauthor{Nikoleta E. Glynatsi.\\E-mail: glynatsine@cardiff.ac.uk}

\begin{document}

\maketitle

%% Adds the main heading for the SI text. Comment out this line if you do not have any supporting information text.
\SItext

\subsection*{Proof of Theorem 2}
The utility of a memory one player \(p\) against an opponent \(q\),
\(u_q(p)\), can be written as a ratio of two quadratic forms on \(R^4\).

\begin{proof}

It was discussed that \(u_q(p)\) it is the product of the steady states \(v\) and
the PD payoffs,

\[u_q(p) = v \cdot (R, S, T, P).\]

More specifically, with \((R, P, S, T) = (3, 1, 0, 5)\)

\begingroup
\footnotesize
\begin{equation}
    u_q(p) =
    \left(
      \frac
        {\parbox{6in}{$
            p_{1} p_{2} (q_{1} q_{2} - 5 q_{1} q_{4} - q_{1} - q_{2} q_{3} + 5 q_{3} q_{4} + q_{3}) + p_{1} p_{3} (- q_{1} q_{3} + q_{2} q_{3}) + p_{1} p_{4} (5 q_{1} q_{3} - 5 q_{3} q_{4}) + p_{3} p_{4} (- 3 q_{2} q_{3} + 3 q_{3} q_{4}) +$ \\
            \hspace*{1cm} $ p_{2} p_{3} (- q_{1} q_{2} + q_{1} q_{3} + 3 q_{2} q_{4} + q_{2} - 3 q_{3} q_{4} - q_{3}) + p_{2} p_{4} (- 5 q_{1} q_{3} + 5 q_{1} q_{4} + 3 q_{2} q_{3} - 3 q_{2} q_{4} + 2 q_{3} - 2 q_{4}) + $ \\
            \hspace*{1cm} $ p_{1} (- q_{1} q_{2} + 5 q_{1} q_{4} + q_{1}) + p_{2} (q_{2} q_{3} - q_{2} - 5 q_{3} q_{4} - q_{3} + 5 q_{4} + 1) + p_{3} (q_{1} q_{2} - q_{2} q_{3} - 3 q_{2} q_{4} - q_{2} + q_{3}) +$ \\
            \hspace*{4cm} $ p_{4} (- 5 q_{1} q_{4} + 3 q_{2} q_{4} + 5 q_{3} q_{4} - 5 q_{3} + 2 q_{4}) + q_{2} - 5 q_{4} - 1$
        }}
        {\parbox{6in}{$
        p_{1} p_{2} (q_{1} q_{2} - q_{1} q_{4} - q_{1} - q_{2} q_{3} + q_{3} q_{4} + q_{3}) + p_{1} p_{3} (- q_{1} q_{3} + q_{1} q_{4} + q_{2} q_{3} - q_{2} q_{4}) + p_{1} p_{4} (- q_{1} q_{2} + q_{1} q_{3} + q_{1} + q_{2} q_{4} - q_{3} q_{4} - q_{4}) +$ \\
        $ p_{2} p_{3} (- q_{1} q_{2} + q_{1} q_{3} + q_{2} q_{4} + q_{2} - q_{3} q_{4} - q_{3}) + p_{2} p_{4} (- q_{1} q_{3} + q_{1} q_{4} + q_{2} q_{3} - q_{2} q_{4}) + p_{3} p_{4} (q_{1} q_{2} - q_{1} q_{4} - q_{2} q_{3} - q_{2} + q_{3} q_{4} + q_{4}) + $ \\
        $ p_{1} (- q_{1} q_{2} + q_{1} q_{4} + q_{1}) + p_{2} (q_{2} q_{3} - q_{2} - q_{3} q_{4} - q_{3} + q_{4} + 1) + p_{3} (q_{1} q_{2} - q_{2} q_{3} - q_{2} + q_{3} - q_{4}) + p_{4} (- q_{1} q_{4} + q_{2} + q_{3} q_{4} - q_{3} + q_{4} - 1) + $ \\
        \hspace*{7cm} $q_{2} - q_{4} - 1$
      }}
    \right).
\end{equation}
\endgroup

Let us consider the numerator of \(u_q(p)\). The cross product terms \(p_ip_j\)
are given by,

\begingroup
\footnotesize
\begin{align*}
p_{1} p_{2} (q_{1} q_{2} - 5 q_{1} q_{4} - q_{1} - q_{2} q_{3} + 5 q_{3} q_{4}
+ q_{3}) + p_{1} p_{3} (- q_{1} q_{3} + q_{2} q_{3}) + p_{1} p_{4} (5 q_{1} q_{3} -
5 q_{3} q_{4}) + p_{3} p_{4} (- 3 q_{2} q_{3} + 3 q_{3} q_{4}) +  \\
p_{2} p_{3} (- q_{1} q_{2} + q_{1} q_{3} + 3 q_{2} q_{4} + q_{2} - 3 q_{3} q_{4} - q_{3}) +
p_{2} p_{4} (- 5 q_{1} q_{3} + 5 q_{1} q_{4} + 3 q_{2} q_{3} - 3 q_{2} q_{4} +
2 q_{3} - 2 q_{4}).
\end{align*}
\endgroup

This can be re written in a matrix format given by Eq.~\ref{eq:cross_product_coeffs}.

\begin{equation}\label{eq:cross_product_coeffs}
    \resizebox{0.8\linewidth}{!}{\arraycolsep=2.5pt%
    \boldmath\( 
    (p_1, p_2, p_3, p_4) \frac{1}{2} \left[\begin{matrix}0 & 5 q_{4} \left(q_{1} - q_{3}\right) & - q_{4} \left(q_{1} - q_{2}\right) & \left(q_{1} - q_{4}\right) \left(q_{2} - 5 q_{3} - 1\right)\\5 q_{4} \left(q_{1} - q_{3}\right) & 0 & - 3 q_{4} \left(q_{2} - q_{3}\right) & \left(q_{3} - q_{4}\right) \left(5 q_{1} - 3 q_{2} - 2\right)\\- q_{4} \left(q_{1} - q_{2}\right) & - 3 q_{4} \left(q_{2} - q_{3}\right) & 0 & - \left(q_{2} - q_{4}\right) \left(q_{1} - 3 q_{3} - 1\right)\\\left(q_{1} - q_{4}\right) \left(q_{2} - 5 q_{3} - 1\right) & \left(q_{3} - q_{4}\right) \left(5 q_{1} - 3 q_{2} - 2\right) & - \left(q_{2} - q_{4}\right) \left(q_{1} - 3 q_{3} - 1\right) & 0\end{matrix}\right] \begin{pmatrix} 
    p_1 \\
    p_2 \\
    p_3 \\
    p_4 \end{pmatrix}
    \) }
\end{equation}

Similarly, the linear terms are given by,

\begingroup
\footnotesize
\begin{align*}
p_{1} (- q_{1} q_{2} + 5 q_{1} q_{4} + q_{1}) + p_{2} (q_{2} q_{3} - q_{2} - 5 q_{3} q_{4} - q_{3} + 5 q_{4} + 1) + p_{3} (q_{1} q_{2} - q_{2} q_{3} - 3 q_{2} q_{4} - q_{2} + q_{3}) + \\
p_{4} (- 5 q_{1} q_{4} + 3 q_{2} q_{4} + 5 q_{3} q_{4} - 5 q_{3} + 2 q_{4}).
\end{align*}
\endgroup

and the expression can be written using a matrix format as Eq.~\ref{eq:linear_coeffs}.

\begin{equation}\label{eq:linear_coeffs}
    \resizebox{0.38\linewidth}{!}{\arraycolsep=2.5pt%
    \boldmath\(
    (p_1, p_2, p_3, p_4) \left[\begin{matrix}- 5 q_{1} q_{4}\\5 q_{4} \left(q_{3} - 1\right)\\q_{4} \left(2 q_{2} + 1\right)\\5 q_{1} q_{4} - 2 q_{2} q_{4} - q_{2} - 5 q_{3} q_{4} + 5 q_{3} - 3 q_{4} + 1\end{matrix}\right]\)}
\end{equation}

Finally, the constant term of the numerator, which is obtained by substituting
$p=(0, 0, 0, 0)$, is given by Eq.~\ref{eq:constant}.

\begin{equation}\label{eq:constant}
q_{2} - 5 q_{4} - 1
\end{equation}

Combining Eq.~\ref{eq:cross_product_coeffs}, Eq.~\ref{eq:linear_coeffs} and Eq.~\ref{eq:constant}
gives that the numerator of \(u_q(p)\) can be written as,

\begingroup
\tiny\boldmath
\begin{align*}
    \frac{1}{2}p & \left[\begin{matrix}0 & 5 q_{4} \left(q_{1} - q_{3}\right) & - q_{4} \left(q_{1} - q_{2}\right) & \left(q_{1} - q_{4}\right) \left(q_{2} - 5 q_{3} - 1\right)\\5 q_{4} \left(q_{1} - q_{3}\right) & 0 & - 3 q_{4} \left(q_{2} - q_{3}\right) & \left(q_{3} - q_{4}\right) \left(5 q_{1} - 3 q_{2} - 2\right)\\- q_{4} \left(q_{1} - q_{2}\right) & - 3 q_{4} \left(q_{2} - q_{3}\right) & 0 & - \left(q_{2} - q_{4}\right) \left(q_{1} - 3 q_{3} - 1\right)\\\left(q_{1} - q_{4}\right) \left(q_{2} - 5 q_{3} - 1\right) & \left(q_{3} - q_{4}\right) \left(5 q_{1} - 3 q_{2} - 2\right) & - \left(q_{2} - q_{4}\right) \left(q_{1} - 3 q_{3} - 1\right) & 0\end{matrix}\right] p^T +  \\
    & \left[\begin{matrix}0 & 5 q_{4} \left(q_{1} - q_{3}\right) & - q_{4} \left(q_{1} - q_{2}\right) & \left(q_{1} - q_{4}\right) \left(q_{2} - 5 q_{3} - 1\right)\\5 q_{4} \left(q_{1} - q_{3}\right) & 0 & - 3 q_{4} \left(q_{2} - q_{3}\right) & \left(q_{3} - q_{4}\right) \left(5 q_{1} - 3 q_{2} - 2\right)\\- q_{4} \left(q_{1} - q_{2}\right) & - 3 q_{4} \left(q_{2} - q_{3}\right) & 0 & - \left(q_{2} - q_{4}\right) \left(q_{1} - 3 q_{3} - 1\right)\\\left(q_{1} - q_{4}\right) \left(q_{2} - 5 q_{3} - 1\right) & \left(q_{3} - q_{4}\right) \left(5 q_{1} - 3 q_{2} - 2\right) & - \left(q_{2} - q_{4}\right) \left(q_{1} - 3 q_{3} - 1\right) & 0\end{matrix}\right] p + q_{2} - 5 q_{4} - 1
\end{align*}
\endgroup

and equivalently as,

\[\frac{1}{2}pQp^T + cp + a\]

where \(Q\) \(\in \R^{4\times4}\) is a square matrix defined by the
transition probabilities of the opponent \(q_1, q_2, q_3, q_4\) as follows:

\begin{equation*}
    \resizebox{0.9\linewidth}{!}{\arraycolsep=2.5pt%
    \boldmath\(
    Q = \left[\begin{matrix}0 & 5 q_{4} \left(q_{1} - q_{3}\right) & - q_{4} \left(q_{1} - q_{2}\right) & \left(q_{1} - q_{4}\right) \left(q_{2} - 5 q_{3} - 1\right)\\5 q_{4} \left(q_{1} - q_{3}\right) & 0 & - 3 q_{4} \left(q_{2} - q_{3}\right) & \left(q_{3} - q_{4}\right) \left(5 q_{1} - 3 q_{2} - 2\right)\\- q_{4} \left(q_{1} - q_{2}\right) & - 3 q_{4} \left(q_{2} - q_{3}\right) & 0 & - \left(q_{2} - q_{4}\right) \left(q_{1} - 3 q_{3} - 1\right)\\\left(q_{1} - q_{4}\right) \left(q_{2} - 5 q_{3} - 1\right) & \left(q_{3} - q_{4}\right) \left(5 q_{1} - 3 q_{2} - 2\right) & - \left(q_{2} - q_{4}\right) \left(q_{1} - 3 q_{3} - 1\right) & 0\end{matrix}\right]\)},
\end{equation*}

\(c\) \(\in \R^{4 \times 1}\) is similarly defined by:

\begin{equation*}
    \resizebox{0.3\linewidth}{!}{\arraycolsep=2.5pt%
    \boldmath\(c = \left[\begin{matrix}- 5 q_{1} q_{4}\\5 q_{4} \left(q_{3} - 1\right)\\q_{4} \left(2 q_{2} + 1\right)\\5 q_{1} q_{4} - 2 q_{2} q_{4} - q_{2} - 5 q_{3} q_{4} + 5 q_{3} - 3 q_{4} + 1\end{matrix}\right]\),}
\end{equation*}

and \(a = 5 q_{4}\).

The same process is done for the denominator.
\end{proof}

\subsection*{Proof of Theorem 3}
\begin{proof}
    The optimal behaviour of a memory-one strategy player
    \(p^* \in \R_{[0, 1]} ^ 4\)
    against a set of \(N\) opponents \(\{q^{(1)}, q^{(2)}, \dots, q^{(N)} \}\)
    for \(q^{(i)} \in \R_{[0, 1]} ^ 4\) is established by:
    
    \[p^* = \textnormal{argmax}\left(\sum\limits_{i=1} ^ N  u_q(p)\right), \ p \in S_q,\]
    
    where \(S_q\) is given by:
    \begin{equation}\label{eq:s_q_set}
        S_q =
        \left\{p \in \mathbb{R} ^ 4 \left|
            \begin{aligned}
                \bullet\quad p_j \in \{0, 1\} & \quad \text{and} \quad \frac{d}{dp_k} 
                \sum\limits_{i=1} ^ N  u_q^{(i)}(p) = 0 \\
                & \quad \text{for all} \quad j \in J \quad \&  \quad k \in K  \quad \text{for all} \quad J, K \\
                & \quad \text{where} \quad J \cap K = \O \quad
                \text{and} \quad J \cup K = \{1, 2, 3, 4\}.\\
                \bullet\quad  p \in \{0, 1\} ^ 4
            \end{aligned}\right.
        \right\}.
    \end{equation}

    The optimisation problem of Eq.~\ref{eq:mo_tournament_optimisation} 

    \begin{equation}\label{eq:mo_tournament_optimisation}
        \begin{aligned}
        \max_p: & \ \sum_{i=1} ^ {N} {u_q}^{(i)} (p)
        \\
        \text{such that}: & \ p \in \R_{[0, 1]}
        \end{aligned}
    \end{equation}

    can be written as:

    \begin{equation}\label{eq:mo_tournament_optimisation_standard}
        \begin{aligned}
        \max_p: & \ \sum_{i=1} ^ {N} {u_q}^{(i)} (p)
        \\
        \text{such that}: p_i & \leq 1 \text{ for } \in \{1, 2, 3, 4\} \\
        - p_i & \leq 0 \text{ for } \in \{1, 2, 3, 4\} \\
        \end{aligned}
    \end{equation}
    
    The optimisation problem has two inequality constraints and regarding the optimality
    this means that:
    
    \begin{itemize}
        \item either the optimum is away from the boundary of the optimization domain, and so the constraints plays no role;
        \item or the optimum is on the constraint boundary.
    \end{itemize}
    
    Thus, the following three cases must be considered:
    
    \textbf{Case 1:} The solution is on the boundary and any of the possible
    combinations for $p_i \in \{0, 1\}$ for $i \in \{1, 2, 3, 4\}$ are candidate
    optimal solutions.
    
    \textbf{Case 2:} The optimum is away from the boundary of the optimization domain
    and the interior solution $p^*$ necessarily satisfies the condition
    \(\frac{d}{dp} \sum\limits_{i=1} ^ N  u_q(p^*) = 0\).
    
    \textbf{Case 3:} The optimum is away from the boundary of the optimization domain
    but some constraints are equalities. The candidate solutions in this case
    are any combinations of $p_j \in \{0, 1\} \quad \text{and} \quad \frac{d}{dp_k} 
    \sum\limits_{i=1} ^ N  u_q^{(i)}(p) = 0$ 
    forall $ j \in J \text{ \& } k \in K \text{ forall } J, K
    \text{ where } J \cap K = \O \text{ and } J \cup K = \{1, 2, 3, 4\}.$
    
    Combining cases 1-3 a set of candidate solution is constructed as:
    
    \begin{equation*}
        S_q =
        \left\{p \in \mathbb{R} ^ 4 \left|
            \begin{aligned}
                \bullet\quad p_j \in \{0, 1\} & \quad \text{and} \quad \frac{d}{dp_k} 
                \sum\limits_{i=1} ^ N  u_q^{(i)}(p) = 0
                \quad \text{for all} \quad j \in J \quad \&  \quad k \in K  \quad \text{for all} \quad J, K \\
                & \quad \text{where} \quad J \cap K = \O \quad
                \text{and} \quad J \cup K = \{1, 2, 3, 4\}.\\
                \bullet\quad  p \in \{0, 1\} ^ 4
            \end{aligned}\right.
        \right\}.
    \end{equation*}
    
    This set is denoted as $S_q$ and the optimal solution to
    Eq.~\ref{eq:mo_tournament_optimisation} is the point from $S_q$ for which the
    utility is maximised.
    \end{proof}

\end{document}
