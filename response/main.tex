\documentclass{article}
\usepackage[margin=2cm, includefoot, footskip=30pt]{geometry}
\setlength\parindent{0pt}
\setlength{\parskip}{1em}
\usepackage{authblk}
 
\title{Stability of defection, optimisation of strategies and the limits of
memory in the Prisoner's Dilemma RESPONSE TO REVIEWS}

\author{Nikoleta E. Glynatsi, Vincent A. Knight}
 
\begin{document}

\maketitle

We would like to open this response by thanking the reviewers for their
thoughtful comments and suggestions. We have fully taken their comments on board
and made significant modifications to both the mathematical arguments and the narrative. We feel this has greatly improved the work.

The main modifications include the improvement of the theorems' proofs and the
generalisation of our results regardless the PD payoff matrix. Moreover, we
included more details of our methodology, such as the mathematical explanation
of the SSE method, and a discussion based on the results of Hilbe et al from
``Partners and rivals in direct reciprocity'' (Nature Human Behaviour, 2018).
Finally, following the suggestions of Reviewer 2 we improved the narrative of
the manuscript which involved changing the title of the manuscript to 
``Best response memory-one strategies with a theory of mind: the limitations
of extortion and restricted memory''.

We hope that our manuscript will be considered for publication following our
modifications.

\section{Response}

\textbf{Regarding the comments of Reviewer 1:}

\begin{verbatim}
     ``Theorem 2: First, it is not clear how equation 17 is obtained.''
\end{verbatim}

We have added further details on how to obtain the equation in the proof of Theorem 2
(in the current version Theorem 1).

\begin{verbatim}
    ``Theorem 2: Moreover, even if that's correct, it is only for a
    specific PD payoff matrix. One cannot state the general results in (4) from
    just a specific case. To obtain (4), authors need to establish it for the
    general PD payoff matrix.''
\end{verbatim}

We agree with the comment of the reviewer. The utility was
specific only for a given payoff matrix. We have generalised our result by
considering the generalised payoffs \((R, S, T, P)\).

\begin{verbatim}
     ``Theorem 3: The proof in the Appendix did not show how to obtain equations
    (22) - (24), the key results of the Theorem''
\end{verbatim}

We have addressed this comment by making additions to the proof.

\begin{verbatim}
     ``Theorem 3: and what was stated in (5)''
\end{verbatim}

We have addressed this comment by specifying that it is an algebraic manipulation
following equation (4).

\textbf{Regarding the comments of Reviewer 2:}

\begin{verbatim}
    ``what exactly is the evolutionary process used, what are the parameters?
    This is not mentioned anywhere in the paper or appendix, and makes it then
    hard to evaluate the impact of the findings, or give any hope for reproducing
    the results. I realize that there is a repository containing raw data, but
    the lack of explanation of the setup really harms any expectation of the
    paper being self-contained. The issue of the evolutionary process not being
    explained also prevented me from fully following the argument the authors
    make for why best responses in the evolutionary process would be less
    extortionate. ''
\end{verbatim}

We agree with the comment of the reviewer. We are not running a evolutionary
tournament and now understand that our wording was misleading. We study best
response memory-one strategies that include self interactions which in a sense are
approximating evolutionary stable best response strategies. We have reworded
the analysis of best response memory-one strategies that include self interactions
and it's results.

\begin{verbatim}
    ``Another related problem is that the authors don't go into more detail about
    their SSE method. A more expansive explanation (also of Table 1) would be of
    critical importance, seeing that this is one of the central points of the
    results section.''
\end{verbatim}

We have address this comment by including details of the SSE method and a better
discussion of the results based on Table 1. We have also included plots of the
SSE distributions that accompany Table 1.

\begin{verbatim}
    ``Another point I want to raise is how this work relates to the findings in Hilbe
    et al.: Partners and rivals in direct reciprocity (Nature Human Behaviour,
    2018). That previous review article shows, among other results, the conditions
    under which "partner" (aiming for mutual cooperation, fighting back against
    defection) or "rival" (defecting/extortionate) strategies evolve in the IPD. It
    seems that the authors of the current work investigate a similar distinction
    (however, they should explain what exactly they mean with "adaptable", as it
    features as a contrast to "extortionate"), and thus would profit from discussing
    the Hilbe et al. article as well.
    ''
\end{verbatim}

We thank the reviewer for their article recommendation. The findings of Hilbe et
 al and how our results relate to them, are discussed in the ``Results'' section.

\begin{verbatim}
    ``As a last point, I think that the flow of the paper could be improved. As it is
    now, the paper reads a bit like a list of results that are more or less loosely
    tied together (especially Section 1.2 suffers from this problem in comparison to
    the others). It would be good to motivate single sections in the broader context
    of the message the paper wants to tell. ''
\end{verbatim}

Thank you for this comment which encouraged us to reflect on the narrative of the
paper, as a result we have modified the title of the manuscript and the titles
of the sections. Moreover, the results of Section 1.2 have been to the Appendix,
and the Theorem 1 (old version Theorem 2) has been moved in the main body.

\begin{verbatim}
    ``The figures should be larger, such that the labels are easier to read''
\end{verbatim}

We have made this suggested change.

\begin{verbatim}
    ``Please make a reference to the Appendix when you mention the Theorems, and
    maybe summarize their core statement in the main text, which would make
    the paper flow better.''
\end{verbatim}

We have made this suggested change.

\end{document}
