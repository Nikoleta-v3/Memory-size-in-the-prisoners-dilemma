\section{Appendix Proofs}

\subsection{Theorem 3}




\subsection{Lemma 4}

\begin{proof}

    where,
    \begin{equation}\label{eq:group_derivative_numerator_condition}
    F(p)=(\sum\limits_{i=1} ^ {N} Q_{N}^{(i)'} \prod_{\substack{j=1 \\ j \neq i}} ^ N Q_{D}^{(i)}
     + \sum\limits_{i=1} ^ {N} Q_{D}^{(i)'} \sum_{\substack{j=1 \\ j \neq i}} ^ {N} Q_{N}^{(i)}
    \prod_{\substack{j=1 \\ j \neq \{i, j\}}} ^ N Q_{D}^{(i)}) \times
    \prod\limits_{i=1} ^ N Q_{D}^{(i)} - (\sum\limits_{i=1} ^ {N} Q_{D}^{(i)'}
    \prod_{\substack{j=1 \\ j \neq i}} ^ N Q_{D}^{(i)}) \times
    (\sum\limits_{i=1} ^ {N} Q_{N}^{(i)} \prod_{\substack{j=1 \\ j \neq i}} ^ N Q_{D}^{(i)})
    \end{equation}

    and,
    \begin{align*}
        Q_{N}^{(i) } & = \frac{1}{2} pQ^{(i)} p^T + c^{(i)} p + a^ {(i)}, \\
        Q_{N}^{(i)'} & =  pQ^{(i)} + c^{(i)}, \\
        Q_{D}^{(i) } & = \frac{1}{2} p\bar{Q}^{(i)} p^T + \bar{c}^{(i)} p + \bar{a}^ {(i)}, \\
        Q_{D}^{(i)'} & =  p\bar{Q}^{(i)} + \bar{c}^{(i)}. \\
    \end{align*}


    The best response of a memory one strategy against a group of memory one
    strategies can captured by a candidate set of behaviours. This candidate set
    is constructed by considering behaviours where any or all of \(p_1, p_2,
    p_3, p_4\) are \(\in \{0, 1\}\) and the rest or all of \(p_1, p_2, p_3,
    p_4\) are given by roots of the partial derivatives.
    
    Note that for \(p_i \in
    \{0, 1\}\) we consider the roots of the partial derivatives for \(p_j \neq
    p_i\) for \(i,j \in [1, 4]\). The derivatives, \(\frac{d\sum u}{dp}\), are
    given by,

    {\scriptsize
    \begin{align}\label{eq:mo_tournament_derivative}
        \frac{d}{dp} \sum\limits_{i=1} ^ {N} {u_q}^{(i)} (p) & = \nonumber \\
        & =\frac{
        (\sum\limits_{i=1} ^ {N} Q_{N}^{(i)'} \prod_{\substack{j=1 \\ j \neq i}} ^ N Q_{D}^{(i)}
        + \sum\limits_{i=1} ^ {N} Q_{D}^{(i)'} \sum_{\substack{j=1 \\ j \neq i}} ^ {N} Q_{N}^{(i)}
       \prod_{\substack{j=1 \\ j \neq \{i, j\}}} ^ N Q_{D}^{(i)}) \times
       \prod\limits_{i=1} ^ N Q_{D}^{(i)} - (\sum\limits_{i=1} ^ {N} Q_{D}^{(i)'}y-vk
       \prod_{\substack{j=1 \\ j \neq i}} ^ N Q_{D}^{(i)}) \times
       (\sum\limits_{i=1} ^ {N} Q_{N}^{(i)} \prod_{\substack{j=1 \\ j \neq i}} ^ N Q_{D}^{(i)})}
        {(\prod\limits_{i=1} ^ N Q_{D}^{(i)})^{2}}
    \end{align}
    }

    For equation~\ref{eq:mo_tournament_derivative} to be zero, the numerator
    must fall to zero and the denominator can not nullified. One the candidate
    set is constructed each point is evaluated using equation
    (\ref{eq:tournament_utility}). The point with the maximum utility is
    selected.
\end{proof}


\subsection{Lemma 5}

A special case of Lemma~\ref{lemma:memone_group_best_response} is for \(N=1\),
thus when a strategy plays against a single opponent. In this case the formulation
of Theorem~\ref{theorem:quadratic_form_u} is used and the best response is captured
by Lemma~\ref{lemma:memone_best_response}.

% TODO I wonder if moving this long list of specific results to an appendix
% would help focus the paper. (They could be an entire separate chapter in your
% thesis).
\begin{lemma}\label{lemma:memone_best_response}
    The optimal behaviour of a memory one strategy player \(p^* \in \R_{[0, 1]} ^ 4\)
    against a given opponent \(q \in \R_{[0, 1]} ^ 4\) is given by:

    \[p^* = \textnormal{argmax}(u_q(p)), \ p \in S_q,\]

    where the set \(S_q\) is defined as

    \[S_q = \{0, \bar{p}_i, 1 \}^4 \text{ for } i \in \R,\]

    where any \(\bar{p}\) satisfy conditions:

    {\small
    \begin{equation}\label{eq:derivative_numerator_condition}
        (\bar{p}Q + c) ( \frac{1}{2} \bar{p}  \bar{Q}  \bar{p}^T + \bar{c}  \bar{p} + \bar{a})
        - (\bar{p}\bar{Q} + \bar{c})( \frac{1}{2} \bar{p}  Q  \bar{p}^T + c \bar{p} + a) = 0
    \end{equation}}

    and

    {\small
    \begin{equation}\label{eq:derivative_denominator_condition}
        \frac{1}{2} \bar{p}  \bar{Q}  \bar{p}^T + \bar{c}  \bar{p} + \bar{a} \neq 0
    \end{equation}}
\end{lemma}

\begin{proof} The best response of a memory one strategy against another memory
    one strategy can captured by a candidate set of behaviours. This candidate
    set is constructed by considering behaviours where any or all of \(p_1, p_2,
    p_3, p_4\) are \(\in \{0, 1\}\) and the rest or all of \(p_1, p_2, p_3,
    p_4\) are given by roots of the partial derivatives.

    Note that for \(p_i \in \{0, 1\}\) we consider the roots of the partial
    derivatives for \(p_j \neq p_i\) for \(i,j \in [1, 4]\). The derivatives,
    \(\frac{du}{dp}\), are given by,

    \begin{equation}\label{eq:mo_derivative}
        \frac{du_q (p)}{dp}  = \frac{(pQ + c) ( \frac{1}{2} p  \bar{Q}  p^T + \bar{c}  p + \bar{a})
        - (p\bar{Q} + \bar{c})( \frac{1}{2} p  Q  p^T + c p + a)}
          {( \frac{1}{2} p  \bar{Q}  p^T + \bar{c}  p + \bar{a})^2} \\
    \end{equation}

    For equation~\ref{eq:mo_tournament_derivative} to be zero, the numerator must fall
    to zero and the denominator can not be zero.
\end{proof}

Equation (\ref{eq:derivative_numerator_condition}) is systems of at most \(4\)
polynomials and the degree of the polynomials is gradually increasing every time
an extra opponent is taken into account.
%TODO add sentence and (see Appendix) on how this is solved for a constrainted
% version of the reactive case. Briefly comment on resultants. Though severals
% for larger system does exist
Solving system of polynomials corresponds to the calculation of a resultant and
for large systems these quickly become intractable.
% TODO Add a reference here about resultants.
Because of that no further analytical consideration is given to problems
described here

\subsection{Reactive case stability of defection.}

A further constrained version of Lemma~\ref{lemma:stability_of_defection}, is
for single interactions while the opponent is a reactive player. Defection is
known to be stable in such interactions by the condition given in
Lemma~\ref{lemma:stability_of_defection_reactive}.

\begin{lemma}\label{lemma:stability_of_defection_reactive}
    Defection is the best responses of a memory one player \(p\) against a reactive
    player \(q\) if the transition probabilities of the opponent satisfy the
    condition:
    
    \begin{equation}
        4 q_{1} - 5 q_{2} - 1 > 0
    \end{equation}
    \end{lemma}

\begin{proof}
Initially, consider equation~(\ref{eq:mo_derivative}) for \(p = (0, 0, 0,
0)\),

\begin{equation}\label{eq:derivative_of_quadratic_zero}
    \begin{aligned}
     \frac{du}{dp_{| p=(0, 0, 0, 0)}} & = && \frac{c \bar{a} - \bar{c}a}
      {\bar{a}^2} .\\
    \end{aligned}
\end{equation}

The numerator \(\bar{c}a - c\bar{a}\) is given by,

\[\input{tex/defection_matrix.txt}\]

and the denominator \(\bar{a} ^ 2 = (-q_2 + q_4 + 1) ^ 2\), which is always positive. In order
for defection to be the best response the derivative must have a negative
sign at the point \(p = (0, 0, 0, 0)\). That means that the utility is only
decreasing after \(p = (0, 0, 0, 0)\).

Because \(\bar{a} ^ 2\) is always positive the sign of the derivative is given by \(\bar{c}a - c\bar{a}\).
More specifically from equations,

\begin{equation}\label{eq:defection_condition_one}
    \input{tex/defection_condition_one.txt}
\end{equation}
\begin{equation}\label{eq:defection_condition_two}
    \input{tex/defection_condition_two.txt}
\end{equation}

Both signs of the partial derivatives must be negative in order for the overall
function to be decreasing ensuring defection is a best response.
The signs of equations (\ref{eq:defection_condition_one}) and (\ref{eq:defection_condition_two})
vary. There are cases that they have the same sign and cases that they do not,

Moreover lets us consider a constrained version of the problem once again. Lets us
assume that in an pairwise interaction the opponent is a reactive player \(q=(q_1, q_2, q_1, q_2)\).
By substituting \(q_3=q_1\) and \(q_4=q_2\) equations (\ref{eq:defection_condition_one})
and (\ref{eq:defection_condition_two}) are now re written as follow,

\[\left[\begin{matrix}- q_{2} \left(4 q_{1} - 5 q_{2} - 1\right)\\
\left(q_{2} - 1\right) \left(4 q_{1} - 5 q_{2} - 1\right)\end{matrix}\right]\]

\end{proof}
