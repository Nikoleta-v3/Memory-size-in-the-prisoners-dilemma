\section{Appendix Tables}\label{appendix:tables}

The memory one strategies used in the computer tournament described in~\cite{Stewart2012}
are given by Table~\ref{table:list_stewart_plotkin}.

\begin{table}[htbp]
    \begin{center}
        \resizebox{\textwidth}{!}{\begin{tabular}{clclll}
        \toprule
        {}&  Name & Memory one representation & Explanation \\
        \midrule
        1  & Cooperator           & \((1, 1, 1, 1)\) & Always chooses \(C\).\\
        2  & Defector             & \((0, 0, 0, 0)\) & Always chooses \(D\).\\
        3  & Random               & \((\frac{1}{2}, \frac{1}{2}, \frac{1}{2}, \frac{1}{2})\) & 
        Randomly chooses between \(C\) and \(D\) with a probability of 0.5.\\
        4  & Tit for Tat          & \((1, 0, 1, 0)\) & Start with a \(C\) and then mimics
        the opponent's last move.\\
        5  & Grudger              & \((1, 0, 0, 0)\) & Starts by cooperating however
        will defect if at any point the opponent has defected.\\
        6  & Generous Tit for Tat & \((1, \frac{1}{3}, 1, \frac{1}{3})\) & A more generous
        version of Tit for Tat.\\
        7  & Win Stay Lose Shift  & \((1, 0, 0, 1)\) & Starts with a \(C\) and then
        repeats it's previous move only if it was awarded with a payoff of \(R\) or \(T\).\\
        8  & ZDGTFT2              & \((1, \frac{1}{8}, 1, \frac{1}{4})\) & A generous
        zero determinant strategy introduced in~\cite{Stewart2012}\\
        9  & ZDExtort2            & \((\frac{8}{9}, \frac{1}{2}, \frac{1}{3}, 0)\) & An
        extortionate zero determinant strategy introduced in~\cite{Stewart2012}\\
        10 & Hard Joss            & \((\frac{9}{10}, 0, \frac{9}{10}, 0)\) & Cooperates
        with probability \(\frac{9}{10}\) when the opponent cooperates, otherwise emulates Tit for Tat.\\
        \bottomrule
    \end{tabular}}
    \end{center}
    \caption{The memory one strategies from~\cite{Stewart2012}.}
    \label{table:list_stewart_plotkin}
\end{table}