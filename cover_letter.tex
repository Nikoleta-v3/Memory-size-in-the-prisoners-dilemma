\documentclass{article}
\usepackage[margin=2cm, includefoot, footskip=30pt]{geometry}
\setlength\parindent{0pt}
\setlength{\parskip}{1em}
\usepackage{authblk}
 
\title{Cover Letter: A bibliometric study of research topics, collaboration and
influence in the Iterated Prisoner’s Dilemma.}
 
\author[1, *]{Nikoleta E. Glynatsi}
\author[1]{Vincent A. Knight}
 
\affil[1]{Cardiff University, School of Mathematics, Cardiff, United Kingdom}
\affil[*]{Corresponding author: Nikoleta E. Glynatsi, glynatsine@cardiff.ac.uk}
\date{}
\setcounter{Maxaffil}{0}
\renewcommand\Affilfont{\itshape\small}
 
\begin{document}
 
\maketitle

To the editors,
 
The Prisoner's Dilemma has been used for decades to explain human interactions
and the emergence of cooperation amongst us. A paper published in
2012~\cite{Press2012} stated that the best action one can take when playing a
Prisoner's Dilemma game is to manipulate their opponent and ensure victory
whilst not needing to remember much of their past interactions. A rapidly
expanding literature following~\cite{Press2012} questioned these results and
have since then established that memory-one strategies must be forgiving to be
evolutionarily stable~\cite{Stewart2013} and that longer-memory strategies have
a certain advantage over short memory strategies~\cite{Hilbe2017}.

This manuscript presents further evidence downplaying the findings
of~\cite{Press2012}. It reinforces the discussion that the best action is
adaptability not manipulation and that short memory can be limiting. This is
done by exploring the full space of memory-one strategies, and more specifically, the best
responses of memory-one strategies. We introduce a well designed framework which
allow us to identify the best response memory-one strategy against a given set
of opponents and to compare an optimal memory-one strategy and a more complex
strategy which has a larger memory. It is also used to identify conditions for
which defection is stable; thus identifying environments where cooperation will
not occur.
 
The framework which is introduced in this manuscript can be used for the
continued understanding of best responses in the Prisoner’s Dilemma and the
emergence (or not) of cooperative behaviour in complex dynamics, and thus we
believe that our paper would be a great addition to the journal.
 
The Authors.

\textbf{Declarations}

The authors did not have any prior discussions with a Nature Communications editor
about the work described in the manuscript.

\textbf{Potential Reviewers}

\begin{enumerate}
    \item Dr Christian Hilbe; hilbe@evolbio.mpg.de
    \item Dr Alexander J. Stewart; astewar6@CENTRAL.uh.edu
\end{enumerate}

 
\bibliographystyle{plain}
\bibliography{bibliography.bib}
\end{document}

